\documentclass[a4paper]{bsuir-tor}


%\usepackage[sorting=none,backend=biber]{biblatex}

\begin{document}

%%% Титульный лист
\begin{titlepage}
\begin{center}
Министерство образования Республики Беларусь\\
Учреждение образования\\
«Белорусский государственный университет \\
информатики и радиоэлектроники»\\[1.2em]

Факультет компьютерного проектирования\\
Кафедра проектирования информационно-компьютерных систем\\
\end{center}

\vfill


% В титульнике оригинального листа задания эта часть вообще была
% свёрстана таблицей в ворде.

\begin{minipage}{8cm}
  \begin{flushleft}
    \center{СОГЛАСОВАНО}\\
    \raggedright % выравнивание
    Руководитель проекта\\
    доцент\\
    \underline{\hspace*{2cm}} ~В.~С. Колбун\\
    \underline{\hspace*{0.5cm}}.\underline{\hspace*{0.5cm}}.2025\\
  \end{flushleft}
\end{minipage}
\hfill
\begin{minipage}{8cm}
  \begin{flushright}
    \center{УТВЕРЖДАЮ}\\
    \raggedright
    Заведующий кафедрой ПИКС\\
    канд.техн.наук, доцент\\
    \underline{\hspace*{2cm}} ~В.~В. Хорошко\\
    \underline{\hspace*{0.5cm}}.\underline{\hspace*{0.5cm}}.2025\\
  \end{flushright}
\end{minipage}


\vfill
\begin{center}
  ПЛАН ПРОСПЕКТ\\
  по дипломному проекту студента\\
  «Система автоматического управления беспилотным летательным аппаратом мультироторного типа»
\end{center}

\vfill
\begin{flushright}
  \begin{minipage}{6cm}
    \center{Исполнитель}\\
    \raggedright
    Студент  группы 112601\\
    \underline{\hspace*{2cm}} ~А.~Л.Корякин\\
    \underline{\hspace*{0.5cm}}.\underline{\hspace*{0.5cm}}.2025\\
  \end{minipage}
\end{flushright}


\vfill
\begin{center}
    {\normalsize Минск 2025}
\end{center}

\end{titlepage}

%%% Local Variables:
%%% mode: LaTeX
%%% TeX-master: "main"
%%% End:


\section{Общие сведения о разработке}

\subsection{На что направлена разработка \newline}

  Опытно-конструкторская разработка «Система автоматического
  управления беспилотным летательным аппаратом мультироторного типа»
  направлена на создание системы, способной осуществлять стабилизацию и
  управление полётом беспилотного летательным аппарата мультироторного
  типа, обеспечивая устойчивость и выполнение команд пилота или
  автономных систем.

  
  \subsection{Сведения о мировом уровне данного вида продукции. \newline}
  
  В обиходе принято более краткое название этого устройства — полётный
  контроллер.  По состоянии на март 2025 года, такие изделия
  изготавливаются промышленностью и доступны для приобретения.
    
  Основные функции этих устройств: стабилизация полёта, управление
  двигателями, обработка данных с датчиков, выполнении команд пилота или
  автономных систем, а также обеспечении безопасности и интеграции с
  дополнительными системами, такими как GPS и камеры.
  
  \subsection{Основные аспекты разработки: }
  
  \subsubsection{Технологические достижения: }
  c развитием микроконтроллерных устройств, многими компаниями были
  разработаны компактные и высокоточные полётные контроллеры.
  
  \subsubsection{Функциональные возможности: }
  Основной функциональной возможностью полётного контроллера является
  стабилизация полёта, обеспечивающая устойчивость и контроль движения
  мультироторного летательного аппарата в воздухе.

  \subsubsection{Применение: }
  Эти устройства применяются в различных беспилотных летательных
  аппаратах мультироторного типа.
  Они используются для обеспечения стабильного и управляемого
  полёта, что делает их незаменимыми в таких областях, как
  аэрофотосъёмка, сельское хозяйство, спасательные операции, доставка
  грузов и развлечения.

  \subsubsection{Стандарты и соответствие: }
  разработка подобных устройств также сопровождается необходимостью
  соблюдения различных международных стандартов, таких как ISO и IEC,
  что гарантирует их безопасность и надежность при эксплуатации.
  

  
  \subsubsection{Будущие тенденции: }
  ожидается, что с ростом номенклатуры различных беспилотных
  летательный аппаратов, полётный контроллеры будут более
  востребованы. Рост спроса на этом рынке потребует внесения улучшений
  на уровне схемотехники изделия, для сохранения конкурентоспособности
  на рынке.
  

  \subsection{К компаниям,
    занимающимся разработкой полётных контроллеров, относятся:}
  \subsubsection*{ModalAI} \ - компания разрабатывает и производит
  высокоинтегрированные системы управления полётом для мультироторных
  летательных аппаратов, используя технологии искусственного интеллекта.

  \subsubsection*{Airbus SAS, The Boeing Company, Leonardo SpA}
  \ — эти крупные аэрокосмические компании занимаются разработкой
  эффективных и точных систем управления полётом для различных типов
  воздушных судов.
  
   \subsubsection*{NXP Semiconductors} \ — компания предоставляет
   решения для разработки полётных контроллеров, включая процессоры и
   беспроводные продукты для интеграции в системы мультироторных
   летательных аппаратов.

   \subsubsection*{Honeywell International Inc., Raytheon Technologies
     Corporation} \
   — эти компании разрабатывают системы управления полётом для различных
   авиационных платформ, включая гражданские и военные самолёты .

  Эти компании играют ключевую роль в развитии технологий управления
  полётом, обеспечивая надёжность и функциональность для различных
  применений в авиации и мультироторных летательных аппаратах.
  
  
  \subsection{Ожидаемые результаты}
  
  \subsubsection{Повышение надёжности: }
  Новые полётные контроллеры могут включать улучшенные алгоритмы и
  датчики, что повышает стабильность полёта и снижает риск аварий.
  
  \subsubsection{Увеличение автономности: }
  Современные полётные контроллеры могут обеспечивать более высокую
  степень автономности, позволяя мультироторным летательным аппаратам
  выполнять сложные задачи без постоянного вмешательства оператора.

  \subsubsection{Интеграция с современными технологиями: }
  Новые полётные контроллеры могут поддерживать интеграцию с системами
  компьютерного зрения и другими передовыми технологиями, улучшая
  навигацию и возможности мультироторных летательных аппаратов.
  
  \subsubsection{Снижение энергопотребления: }
  Оптимизация аппаратного и программного обеспечения может привести к
  уменьшению энергопотребления, что увеличивает время полёта.

  \subsubsection{Улучшение производительности: }
  Усовершенствованные процессоры и алгоритмы могут обеспечить более
  быструю и точную обработку данных, что улучшает общую
  производительность системы.

  \subsubsection{Снижение затрат на разработку и эксплуатацию: }
  Использование современных технологий и материалов может снизить
  затраты на производство и обслуживание полётных контроллеров.

  \subsubsection{Расширение функциональных возможностей: }
  Новые полётные контроллеры могут поддерживать дополнительные
  функции, такие как управление несколькими мультироторными
  летательными аппаратами одновременно или выполнение специализированных
  задач.

  \subsubsection{Соответствие современным стандартам и требованиям: }
  Разработка новых контроллеров позволяет соответствовать последним
  стандартам безопасности и регулирования, что важно для коммерческого
  использования мультироторных летательных аппаратов.

  Эти улучшения могут значительно расширить области применения
  мультироторных летательных аппаратов и повысить их эффективность в
  различных отраслях.

  \subsection{Возможность дальнейшего развития}
  Разрабатываемая система автоматического управления беспилотным
  летательным аппаратом мультироторного типа
  предполагает возможность дальнейшего развития, обеспечивая высокую
  надёжность и функциональность изделий использующих полётные
  контроллеры.



\section{Наименование и область применения}
\subsection{Наименование: \newline}

Система автоматического управления беспилотным летательным аппаратом
мультироторного типа (в дальнейшем устройство).



\subsection{Область применения}

Полётные контроллеры для беспилотных летательных аппаратов находят
применение в различных областях благодаря своей способности
обеспечивать стабильный и управляемый полёт. Вот основные области
применения:

\subsubsection{Аэрофотосъёмка и видеосъемка: }
Мультироторные летательные аппараты с качественными камерами
используются для создания фотографий и видео с высоты, что популярно в
кинематографе, рекламе и недвижимости.

\subsubsection{Сельское хозяйство: }
Мультироторные летательные аппараты помогают в мониторинге состояния
полей, распылении удобрений и пестицидов, а также в оценке
урожайности.

\subsubsection{Инспекция инфраструктуры: }
Мультироторные летательные аппараты применяются для осмотра
труднодоступных объектов, таких как мосты, электростанции и ветряные
турбины, что позволяет выявлять дефекты и повреждения.

\subsubsection{Спасательные операции и мониторинг чрезвычайных ситуаций: }
Мультироторные летательные аппараты используются для поиска пропавших
людей, оценки ущерба после стихийных бедствий и мониторинга пожаров.

\subsubsection{Доставка грузов: }
В некоторых регионах мультироторные летательные аппараты уже
используются для доставки медицинских препаратов и других небольших
грузов в труднодоступные районы.

\subsubsection{Научные исследования и экологический мониторинг: }
Учёные применяют мультироторные летательные аппараты для сбора данных
о состоянии окружающей среды, изучения дикой природы и проведения
геологических исследований.

\subsubsection{Развлечения и хобби: }
Многие энтузиасты используют мультироторные летательные аппараты для
гонок и выполнения трюков, а также для обучения основам пилотирования.

\subsubsection{Строительство и картография: }
Мультироторные летательные аппараты помогают в создании
топографических карт и трёхмерных моделей местности, что полезно для
планирования строительства и управления земельными ресурсами.



\subsection{Предусматривается использование изделия для экспорта.}

\section{Основание для разработки}

\subsection{Разработка системы автоматического управления обоснована}
несколькими факторами связанными с современными требованиями к
качеству и надёжности мультироторных летательных аппаратов, завязанных
на использование сервоприводов, основными из которых являются:

\subsubsection{Повышение безопасности: }
Автоматические системы управления могут быстро реагировать на
изменения в окружающей среде, такие как препятствия или
неблагоприятные погодные условия, что снижает риск аварий.

  \subsubsection{Увеличение эффективности выполнения задач: }
  Автоматизация позволяет БПЛА выполнять рутинные задачи, такие как
  патрулирование или мониторинг, без необходимости постоянного
  вмешательства оператора, что повышает производительность.

  \subsubsection{Снижение человеческого фактора: }
  Автоматические системы исключают ошибки, связанные с усталостью или
  невнимательностью оператора, что особенно важно в критических миссиях.

  \subsubsection{Расширение возможностей применения: }
  Автономные БПЛА могут использоваться в условиях, опасных для
  человека, например, при ликвидации последствий стихийных бедствий или
  в зонах радиационного загрязнения.

  \subsubsection{Оптимизация ресурсов: }
  Автоматические системы управления могут планировать маршруты и
  распределять задачи между несколькими БПЛА, что позволяет более
  эффективно использовать энергию и время.

  \subsubsection{Соблюдение нормативных требований: }
  Современные системы автоматического управления могут быть настроены
  для соблюдения местных и международных норм и правил, что важно для
  коммерческого использования БПЛА.

  \subsubsection{Интеграция с другими технологиями: }
  Автоматические системы могут быть легко интегрированы с системами
  компьютерного зрения, лидарами и другими датчиками, что расширяет
  функциональные возможности БПЛА.




  \section{Цель и назначение разработки}
  \subsection{Целью разработки является }
  Цель разработки системы автоматического управления беспилотным
  летательным аппаратом мультироторного типа заключается в повышении
  безопасности, эффективности и надёжности выполнения задач
  летательного аппарата.
  

  \subsection{Основные задачи которые должны быть решены: }
  
  \subsubsection{Создать систему автоматического управления}

  \subsubsection{Проектирование схемы: }
  Разработка электрической схемы и печатной платы, обеспечивающих
  корректную работу всех компонентов устройства.

  \subsubsection{Подбор подходящих электронных компонентов, }
  микроконтроллера, датчиков и других элементов, необходимых для
  реализации функций полётного контроллера.

  \subsubsection{Оптимизировать процессы}
  управления летательным аппаратом для повышения точности и
  эффективности.
  
  \subsubsection{Обеспечить возможность сбора и анализа}
  данных о работе полётного контроллера в реальном времени.

  \subsubsection{Улучшить надежность}
  и качество производимых летательных аппаратов.

  В результате опытно-конструкторской работы планируется создать
  инструмент, который обеспечит точный контроль работы сервоприводов
  мультироторного летательного аппарата, что очень важно сразу для
  нескольких областей промышленности.

  \subsection{ Назначение разработки создание }
  конструктивно законченного
  устройства на базе современных изделий электронной техники.

  \subsection{ Разработка должна обеспечить }
  создание базовой модели устройства полётного контроллера.

  \subsection{ Дальнейшее развитие }
  разработки должно выполняться путем создания модификации базовой
  модели, отличающихся конфигурацией и изменениями функций на основе
  частных технических заданий.

  \subsection{Изделие предназначено для серийного изготовления.}

\section{Источники разработки}

\subsection{Основные источники разработки }
должны включать широкий спектр информации и технологий: научные
журналы и конференции, технических документы и стандарты, технические
документы от производителей оборудования, специализированные
источники, патенты, анализ технических характеристик и возможностей
оборудования ведущих производителей, программного обеспечения,
лабораторные эксперименты, практические исследования и опытное
применение.

\subsection{Источниками разработки}
в рамках дипломного проекта должна являться схема электрическая
принципиальная автоматической системы управления, выданная заказчиком,
а также другая документация.

\section{Технические требования}

\subsection{Состав изделия} и требования к конструктивному исполнению
устройства:

\subsubsection{Устройство системы автоматического управления должно содержать}

\begin{itemize}
\item микроконтроллер;
\item четыре потенциометра;
\item разъёмы для подключения четырех сервоприводов;
\item разъёмы для подключения по интерфейсу SPI;
\item разъёмы для подключения по интерфейсу I$^2$C
  
\item кварц для тактирования микроконтроллера;  
\item микросхему регулятора напряжения.
\end{itemize}

  \subsubsection{Устройство системы автоматического управления должно
    изготавливаться}
  в соответствии с требованиями настоящего ТЗ, ТУ и комплекта
  конструкторской документации.

  \subsubsection{По внешнему виду устройство должно}
  соответствовать опломбированному и утвержденному образцу.
  
  \subsubsection{Корпус, передняя панель, }
  органы управления и другие детали наружной поверхности устройства не
  должны иметь дефектов, портящих внешний вид устройства (вмятин, следов
  коррозии, царапин, трещин и других механических повреждений).
  
  \subsubsection{На устройстве должны быть нанесены условные функциональные 
  обозначения (символы), поясняющие назначен органов управления
  по ГОСТ 25874-83.}
  
  \subsubsection{Тумблеры управления
  должны обеспечивать не менее 9000 переключений.}
  
  \subsubsection{Усилие переключения тумблера — не более 15Н.}
  
  \subsubsection{Задняя стенка должна быть надежно прикреплена к корпусу.}

  \subsubsection{Сетевой шнур должен соответствовать требованиям ГОСТ 7399-90}
  и ГОСТ 12.2.007.0-75. Длина сетевого шнура от места выхода из корпуса
  (стенки) до основания контактной вилки должна быть не менее 1 м.
  
  \subsubsection{Масса блока, измеренная с погрешностью плюс-минус 0,1 кг,}
  должна быть не более 1 кг.
  
  \subsubsection{Габаритные размеры устройства},
  измеренные с погрешностью плюс-минус 1 мм, должны быть не более, мм:
  высота - 95 мм, ширина 200 мм, длина 200мм.

  \subsubsection{Защита от электрических разрядов}
  должна определяться по ГОСТ 12.1.030-81.

  \subsubsection{Структура устройства и его конструктивное выполнение}
  должны обеспечивать объедение составных частей в единый базовый
  конструктив.

  \subsubsection{Устройство должно быть оснащено}
  тумблером для включения или выключения питающего напряжения.

  \subsubsection{Конструкция устройства}
  должна обеспечивать свободный доступ к составным элементами изделия
  при проведении наладочных и ремонтных работ.

  \subsubsection{Питание устройства} должно осуществляться от
  производственной сети кабелем в трехжильном исполнении, оснащенной
  трехжильной вилкой, либо батареей с напряжением 5 В.

  \subsubsection{По устойчивости к воздействию температуры}
  и влажности окружающей среды устройство должно соответствовать
  климатическому исполнению к  категории размещения УХЛ 2.1 по
  ГОСТ 15150-69.

  \subsubsection{Конструкция устройства по средствам защиты}
  от механических воздействий должна соответствовать ГОСТ 26568-85.
  
  \subsubsection{Для антикоррозийной защиты }
  поверхностей блоков, деталей и изделий в целом применить
  гальваническое и лакокрасочные покрытия.
  
  \subsubsection{Конструктив устройства должен быть обеспечен}
  винтом общего заземления устройства.
  
  \subsubsection{Винт заземления по предыдущему пункту должен }
  располагаться в месте доступном для монтажа и визуального контроля.

\subsection{Показатели назначения:}

\subsubsection{Максимальная потребляемая мощность}
не более 100ВТ.


\subsubsection{Частота подключаемых сервоприводов } устанавливается
потенциометрами.


\subsection{Требования к надёжности: }
\subsubsection{Показатели надежности (ГОСТ 27.003-2016) должны}
соответствовать заданным значениям при нормальных климатических
условиях (температура окружающей среды плюс-минус 20ºC, относительная
влажность 60 \%, атмосферное давление $(958...1037) \cdot  10^2$ ПА;
c отклонениями напряжения сети 220В от плюс 10\% до минут 15\%
от номинального значения, частотой (49...51) Гц.

\subsubsection{Средний срок службы блока }
 должен быть не менее 30 лет.

 \subsubsection{ По нормам надежности блок должен }
 удовлетворять требованиям ГОСТ 27.003-2016. Время наработки на отказ
 должно быть не менее 10000 часов. Среднее время восстановления
 устройства должно быть не более 60 минут.
 
 \subsubsection{После восстановления работоспособности, }
 по окончании ремонтных работ при его отказе, изделие
 должно сохранять показатели назначения, изложенные в настоящем
 техническом задании.

 \subsubsection{Устройство должно выдерживать}
 воздействия внешних
 механических и климатических факторов в соответствии с ГОСТ 22261-94
 для 3 группы аппаратуры.

 \subsection{Требования к технологичности и
   метрологическому обеспечению разработки,
   производства и эксплуатации.}

 \subsubsection{Показатели технологичности конструкции изделия согласно ГОСТ 14.201-83.}
 
 \subsubsection{Трудоемкость изготовления блока – не более 12 часов.}
 
 \subsubsection{Параметры устройства должны контролироваться с
   помощью стандартных измерительных приборов
   обслуживающим персоналом средней квалификации.}

 \subsubsection{Конструкция устройства
   должна обеспечивать возможность
   выполнения монтажных работ
   с соблюдением требований технических условий
   на установку и пайку комплектующих изделий.}

 \subsubsection{Конструкция устройства в целом
   и отдельных сложных узлов
   должна обеспечивать сборку при изготовлении без создания и применения
   специального оборудования.}

 \subsubsection {При изготовлении устройства должны применяться стандартные
   методы и универсальные средства измерений,
   серийное испытательное оборудование.
   Допускается для проведения климатических проверок
   при технологическом прогоне применять специально
   приготовленную камеру или специально оборудованное оборудование.}

 \subsubsection{ Конструкция устройства должна соответствовать
   требованиям ремонтопригодности по рекомендациям Р 50-84-88.}

 \subsection{Требования к уровню стандартизации и унификации}

 \subsubsection{В качестве комплектующих единиц и деталей
   (коммутационные, изделия электроники, крепежные, установочные)
   должны применяться серийно выпускаемые изделия.}

 \subsubsection{Сборочные единицы типа монтажных плат,
   панелей, крепежных и установочных узлов должны быть унифицированными.}

 \subsubsection{В конструкции устройства должны быть заимствованы
   сборочные единицы, узлы и детали из ранее разработанных изделий.}

 \subsubsection{ Коэффициент унификации стандартных и заимствованных
   деталей должен быть не менее 0,5.}

 \subsection{Требования безопасности и
   требования по охране природы}

 \subsubsection{Общие требования безопасности к конструкции устройства
   должны соответствовать ГОСТ 12.2.007.0-75.}

 \subsubsection{Устройство по способу защиты человека от поражения
   электрическим током относится к классу 01 согласно ГОСТ 12.2.007.0-75.}

 \subsubsection{Конструкция устройства должна исключать
   возможность неправильного присоединения
   его сочленяемых токоведущих и составных частей.}
 
 \subsubsection{Органы управления должны быть снабжены надписями,
   соответствующими их принадлежности и назначению}

 \subsubsection{Присоединительные разъемы электрических цепей
   должны быть снабжены надписями,
   соответствующими их принадлежности и назначению.}

 \subsubsection{Коммутационные изделия,
   устанавливаемые в цепях повышенного напряжения,
   должны быть конструктивно выделены
   и не должны одновременно коммутировать другие цепи.}

 \subsubsection{Конструкция устройства должна исключать попадание
   внутрь посторонних предметов.}

\subsubsection{В эксплуатационных документах по требованиям
  техники безопасности должны быть соблюдены правила технической
  эксплуатации электроустановок потребителем и правила техники
  безопасности при эксплуатации электроустановок потребителем.}

\subsection{Эстетические и эргономические требования}

\subsubsection{Лицевая панель должна быть темной.}

\subsubsection{Устройство по своим эргономическим показателям
  должно обеспечивать удобство эксплуатации в производственных условиях.}

\subsubsection{Органы управления и индикации должны быть
  расположены с достаточным обзором и удобством использования.}

\subsection{Требования к патентной чистоте}

\subsubsection{Патентная чистота стенда должна быть обеспечена
  в отношении стран СНГ и стран возможных импортеров изделия.}

\subsection{Условия эксплуатации,
  требования к техническому обслуживанию и ремонту}

\subsubsection{Устройство должно быть выполнено для климатического
  исполнения УХЛ 2.1 согласно ГОСТ 15150-69
  и нормально функционировать при следующих климатических условиях:}

\begin{itemize}
\item верхнее значение температуры окружающей среды плюс 40ºC
\item нижнее значение температуры окружающей среды плюс -60ºC 
\item среднее (рабочее) значение температуры окружающей среды плюс 0ºС;  
\item атмосферное давление 650-800 мм рт. ст. (86-106 кПа);
\item относительная влажность воздуха не более 75\% при температуре плюс 15С.
\end{itemize}

\subsubsection{Время подготовки устройства к эксплуатации после
  транспортировки и хранения не должно превышать 45 минут.}

\subsubsection{Рабочий режим в устройстве должен устанавливаться не
более чем через 5 минуту после включения.}

\subsubsection{Ремонт устройства должен производиться в
  специализированной ремонтной организации или по месту эксплуатации
  высококвалифицированным радиомехаником.}

\subsection{Требования к маркировке и упаковке}

\subsubsection{Маркировка устройства должна соответствовать
  требованиям ГОСТ 26828-86.}

\subsubsection{Маркировка устройства и входящих составных частей должна содержать:}
\begin{itemize}
\item товарный знак или наименование предприятия-изготовителя;
\item полное торговое наименование;  
\item порядковый номер изделия и составных частей;
\item необходимые поясняющие и предупреждающие надписи,
  выполненные по ГОСТ 12.2.007.0-75;
\item дату изготовления.
\end{itemize}

\subsubsection{Упаковка должна быть выполнена в виде картонного ящика
  с пенопластовыми вкладышами.}

\subsubsection{На таре должны быть нанесены манипуляционные знаки
  «Осторожно, хрупкое»,
  «Боится сырости»,
  «Соблюдение интервала температур», «Верх, не кантовать» по ГОСТ 14192-77}


\subsubsection{При поставке изделия на экспорт все надписи
  выполняются на языке, оговоренном в договоре на поставку.}

\subsection{Требования к транспортированию и хранению}

\subsubsection{Упакованные изделия перевозить только в закрытом транспорте.}

\subsubsection{ Требования к виду транспорта не предъявляются.}

\subsubsection{Условия транспортирования изделия должны
  соответствовать следующим требованиям: }
\begin{itemize}
\item температура воздуха от минус 50 С до плюс 50 ºС;
\item относительная влажность воздуха 95\% при температуре плюс 30 ºС;
\item атмосферное давление от 84 до 107 кПа (от 630 до 800 мм рт. ст.).  
\end{itemize}


\subsubsection{Размещение и крепление упакованных изделий в
транспортных средствах должно обеспечивать их устойчивое положение,
исключить возможность ударов их друг о друга.}


\subsubsection{Устройство должно храниться в предназначенной для него таре
  в закрытых складских помещениях
  при температуре от плюс 5 ºС до плюс 35 ºС и влажности
  85\% на подставках.}

\subsubsection{Расстояние между стенами, полом хранилища
  и изделием должно быть не менее 100 мм, а между отопительными
  устройствами не менее 0,5 м.}

\subsection{Экономические показатели}

\subsubsection{Сравнительные характеристики разрабатываемого
  изделия на основе источников разработки.}

\subsubsection{Предполагаемый годовой выпуск до 1200 устройств в год. }

\subsection{Порядок контроля и приемки}

\subsubsection{Для приемки работы на этапе проведения испытаний
  необходимо представить минимум три стенда.}

\subsubsection{Испытания должны проводиться по программе и методике испытаний,
  утвержденной заказчиком.}

\subsubsection{Для приемки представляются следующие документы:}
\begin{itemize}
\item техническое задание;  
\item комплект конструкторской документации;  
\item ведомость покупных изделий;  
\item программа и методика испытаний;  
\item эксплуатационные документы;  
\item методики проверки.  
\end{itemize}



\subsubsection{Приемочные испытания проводит разработчик,
  приемосдатчик, изготовитель.}

\subsubsection{Приемочные испытания опытного образца производятся в сроки,
  согласованные с заказчиком.}


\subsubsection{Аттестацию опытного образца проводит разработчик
  с участием заказчика,
  а остальных образцов – изготовитель.}

\end{document}


%%% Local Variables:
%%% mode: LaTeX
%%% TeX-master: t
%%% End:
