\documentclass[a4paper]{bsuir-tor}


%\usepackage[sorting=none,backend=biber]{biblatex}

\begin{document}

%%% Титульный лист
\begin{titlepage}
\begin{center}
Министерство образования Республики Беларусь\\
Учреждение образования\\
«Белорусский государственный университет \\
информатики и радиоэлектроники»\\[1.2em]

Факультет компьютерного проектирования\\
Кафедра проектирования информационно-компьютерных систем\\
\end{center}

\vfill

% They done it using table in original document btw
\begin{minipage}{8cm}
  \begin{flushleft}
    \center{СОГЛАСОВАНО}\\
    \raggedright % выравнивание
    Руководитель проекта\\
    доцент\\
    \underline{\hspace*{2cm}} ~В.~C. Колбун\\
    \underline{\hspace*{0.5cm}}.\underline{\hspace*{0.5cm}}.2025\\
  \end{flushleft}
\end{minipage}
\hfill
\begin{minipage}{8cm}
  \begin{flushright}
    \center{УТВЕРЖДАЮ}\\
    \raggedright
    Заведующий кафедрой ПИКС\\
    канд.техн.наук, доцент\\
    \underline{\hspace*{2cm}} ~В.~В. Хорошко\\
    \underline{\hspace*{0.5cm}}.\underline{\hspace*{0.5cm}}.2025\\
  \end{flushright}
\end{minipage}


\vfill
\begin{center}
  ТЕХНИЧЕСКОЕ ЗАДАНИЕ\\
  на опытно-конструкторскую разработку
  «Система тестирования сервоприводов квадрокоптера»
\end{center}

\vfill
\begin{flushright}
  \begin{minipage}{6cm}
    \center{Исполнитель}\\
    \raggedright
    Студент  группы 112601\\
    \underline{\hspace*{2cm}} ~А.~Л.Корякин\\
    \underline{\hspace*{0.5cm}}.\underline{\hspace*{0.5cm}}.2025\\
  \end{minipage}
\end{flushright}


\vfill
\begin{center}
    {\normalsize Минск 2025}
\end{center}

\end{titlepage}

%%% Local Variables: 
%%% coding: utf-8
%%% mode: latex
%%% TeX-engine: xetex
%%% End:


\section{Общие сведения о разработке}

\subsection{На что направлена разработка \newline}

  Опытно-конструкторская разработка
  «Система тестирования сервоприводов квадрокоптера»
  направлена на создание системы,
  способной проводить тестирование работы одновременно
  четырёх сервоприводов и, опционально, полётного контроллера
  квадрокоптера.

  \subsection{Сведения о мировом уровне данного вида продукции. \newline}
  
  В обиходе принято более краткое название этого устройства — сервотестер.
  По состоянии на сентябрь 2024 года, такие изделия изготавливаются
  промышленностью и доступны для приобретения.
  
  Основные функции этих устройств: тестирование сервоприводов,
  управление широтой импульсов широтноимпульсной модуляции,
  тестирование и настройка регуляторов оборотов электродвигателя и
  полетных контроллеров, а также измерение питающего напряжения.
  \subsection{Основные аспекты разработки: }
  \subsubsection{Технологические достижения: }
  c развитием микроконтроллерных устройств, многими компаниями были
  разработаны компактные и высокоточные сервотестеры.
  \subsubsection{Функциональные возможности: }
  Устройства могут обеспечивать как тестирование непосредственно
  сервоприводов, так и полетных контроллеров и регуляторов оборотов
  электродвигателя, подключенных к ним.
  \subsubsection{Применение: }
  Эти устройства находят применение во всех тех областях, в которых уже
  так или иначе применяются устройства с сервоприводами, осуществляется и
  их сборка или ремонт. Например, в промышленных роботах и манипуляторах,
  автоматизированных станках, бытовых роботах, квадрокоптера.
  \subsubsection{Стандарты и соответствие: }
  разработка подобных устройств также
  сопровождается необходимостью соблюдения различных международных
  стандартов, таких как ISO и IEC, что гарантирует их безопасность и
  надежность при эксплуатации.
  \subsubsection{Будущие тенденции: }
  ожидается, что с ростом устройств содержащих в себе сервоприводы,
  сервотестеры будут более востребованы. Рост спроса на этом рынке
  потребует внесения улучшений на уровне схемотехники изделия, для
  сохранения конкурентоспособности на рынке.
  \subsection{К компаниям,
  которые занимаются разработкой систем тестирования сервоприводов
  квадрокоптера можно отнести: }
  \subsubsection*{GSMIN} \  — компания производит сервотестеры для
  использования совместно со средой разработки arduino.
  \subsubsection*{PIMNARA} \   — компания производит сервотестеры для
  тестирования одновременно трех сервомоторов, с возможностью
  переключения частоты широко импульсной модуляции.
  \subsubsection*{G.T.Power} \  — компания производит сервотестеры для
  тестирования одновременно четырех сервомоторов, с возможностью
  вывода информации о прохождении тестирования на семисегментный дисплей.

  Эти компании предоставляют широкий спектр оборудования технология,
  предназначенный для проведения тестирования сервоприводов на
  различных этапах производства электронных устройств содержащих их.
  
  \subsection{Ожидаемые результаты}
  
  \subsubsection{Повышение надёжности: }
  снижение количества вышедших из строя сервоприводов за счет
  тщательного контроля.

  \subsubsection{Экономия ресурсов: }
  оптимизация затрат на производства за счет выявления дефектов и
  снижение численности отбраковки.
  
  \subsubsection{Расширение функциональности: }
  возможность использования устройства для различных типов изделий
  использующих сервоприводы, что увеличивает его универсальность.

  Данная разработка может сыграть ключевую роль в улучшении качества и
  надёжности продукции завязанной на использование сервоприводов.

  \subsection{Возможность дальнейшего развития}
  Разрабатываемая система тестирования сервоприводов микроконтроллера.
  предполагает возможность дальнейшего развития, обеспечивая высокую
  надёжности и функциональности изделий использующих сервоприводы.

\section{Наименование и область применения}
\subsection{Наименование: \newline}
Система тестирования сервоприводов квадрокоптера (в дальнейшем устройство).

\subsection{Область применения}


  \subsubsection{Производственные линии: }
  При тестировании и калибровке сервоприводов в промышленных роботах.
  Проверке и настройке сервоприводов в станках с численно-программным
  управлением. Диагностики и обслуживании сервоприводов в промышленных
  линиях.
  
  \subsubsection{Научные исследования: }
  Тестирование сервоприводов в научных приборах и установках.
  Проверка сервоприводов в роботах, используемых для обучения и
  исследований.
 
  \subsubsection{Сельское хозяйство: }
  Тестирование сервоприводов в системах управления поливом.
  Проверка сервоприводов в роботах для сбора урожая и других операций.
  
  \subsubsection{Электроника и бытовая техника: }
  Проверка сервоприводов в бытовых роботах.
  
  \subsubsection{Логистика: }
  Тестирование сервоприводов в системах автоматического складирования
  и извлечения товаров.
  Проверка сервоприводов в сортировочных конвейерах и роботах.


  \subsubsection{Аэрокосмическая и оборонная промышленность: }
  Тестирование сервоприводов в системах управления полетом.

  \subsubsection{Медицинское оборудование: }
  Проверка сервоприводов в медицинских сканерах и других
  диагностических устройствах.


  \subsubsection{Сервисные центры: }
  для диагностики и ремонта устройств с дефектными сервомоторами.
  

\subsection{Предусматривается использование изделия для экспорта.}

\section{Основание для разработки}

\subsection{Разработка системы тестирования сервоприводов обоснована}
несколькими
факторами связанными с современными требованиями к качеству и
надёжности изделий, завязанных на использование сервоприводов,
основными из которых являются:

  \subsubsection{Повышение надежности оборудования: }
  сервотестер позволяет проводить регулярные проверки и диагностику
  сервоприводов, что помогает выявлять и устранять потенциальные
  проблемы до того, как они приведут к отказам.
  
  \subsubsection{Снижение затрат на обслуживание: }
  Раннее выявление неисправностей позволяет избежать дорогостоящих
  ремонтов и замены оборудования.
  
  \subsubsection{Использование сложных изделий: }
  современные устройства достаточно сложны и потому используют
  одновременно не один, а несколько сервоприводов в своей конструкции.
  Поэтому требуется сервотестер с возможность одновременного
  тестирования сразу нескольких подключенных сервоприводов.

  \subsubsection{Улучшение производительности: }
  Регулярное тестирование и калибровка сервоприводов обеспечивают их
  оптимальную работу, что повышает общую производительность системы.
  
  \subsubsection{Упрощение диагностики: }
  Автоматизированные тесты и диагностические функции сервотестера
  позволяют быстро и точно определять причины неисправностей.

  \subsubsection{Снижение времени простоя: }
  Быстрое выявление и устранение проблем сокращает время простоя
  оборудования, что особенно важно для производственных процессов.

  \subsubsection{Обучение и подготовка персонала: }
  Сервотестер может использоваться для обучения сотрудников, что
  повышает их квалификацию и способность к самостоятельной диагностике и
  ремонту.

  \subsubsection{Интеграция с другими системами: }
  Современные сервотестеры могут интегрироваться с системами
  управления и мониторинга, что позволяет централизованно управлять
  состоянием оборудования.
  
  \subsubsection{Снижение человеческого фактора: }
  Автоматизация процессов тестирования и диагностики снижает вероятность
  ошибок, связанных с человеческим фактором.

  \subsubsection{Увеличение срока службы оборудования: }
  Регулярное обслуживание и тестирование продлевают срок службы
  сервоприводов и других компонентов.

  \subsubsection{Оптимизация запасов: }
  Знание состояния сервоприводов позволяет более эффективно управлять
  запасами запчастей и расходных материалов.

  \subsubsection{Снижение эксплуатационных расходов: }
  Оптимальная работа сервоприводов снижает энергопотребление и другие
  эксплуатационные расходы.


  \subsubsection{Минимизация времени на тестирование: }
  устройство, способное проводить испытание сразу нескольких
  сервоприводов, позволяет сократить время на тестирование и отладку
  изделий, что важно в условиях быстрого выхода продукции на рынок.

  \subsubsection{Соответствие стандартам и нормативам: }
  Сервотестер помогает обеспечивать соответствие оборудования
  различным стандартам и нормативам, что особенно важно для
  регулируемых отраслей.

  \subsubsection{Улучшение клиентского сервиса: }
  Быстрая диагностика и ремонт оборудования повышают удовлетворенность
  клиентов и улучшают их взаимодействие с компанией.

  \section{Цель и назначение разработки}
  \subsection{Целью разработки является }
  создание системы тестирования сервоприводов квадрокоптера, позволяющую
  проводить диагностику, тестирование и калибровку сервоприводов для
  обеспечения их надежной и эффективной работы.

  \subsection{Основные задачи которые должны быть решены: }
  
  \subsubsection{Создать устройство для тестирования сервоприводов}

  \subsubsection{Проектирование схемы: }
  Разработка электрической схемы и печатной платы, обеспечивающих
  корректную работу всех компонентов устройства.

  \subsubsection{Подбор подходящих электронных компонентов, }
  микроконтроллер, датчиков и других элементов, необходимых для
  реализации функций сервотестера.

  \subsubsection{Оптимизировать процессы}
  контроля и диагностики для повышения точности и эффективности.
  
  \subsubsection{Обеспечить возможность сбора и анализа}
  данных о работе сервотестера в реальном времени.

  \subsubsection{Улучшить надежность}
  и качество производимых изделий содержащих сервоприводы,
  через ранее выявлении проблем.

  В результате опытно-конструкторской работы планируется
  создать инструмент, который обеспечит точный контроль
  работы сервоприводов, что очень важно
  сразу для нескольких областей промышленности.

  \subsection{ Назначение разработки создание }
  конструктивно законченного
  устройства на базе современных изделий электронной техники.

  \subsection{ Разработка должна обеспечить }
  создание базовой модели устройства сервотестера.

  \subsection{ Дальнейшее развитие}
  разработки должно выполняться путем создания модификации базовой
  модели, отличающихся конфигурацией и изменениями функций на
  основе частных технических заданий.

  \subsection{Изделие предназначено для серийного изготовления.}

\section{Источники разработки}


\subsection{Основные источники разработки}
Должны включать широкий спектр информации и технологий: научные
журналы и конференции, технических документы и стандарты, технические
документы от производителей оборудования, специализированные
источники, патенты, анализ технических характеристик и возможностей
оборудования ведущих производителей, программного обеспечения,
лабораторные эксперименты, практические исследования и опытное
применение.

\subsection{Источниками разработки}
в рамках дипломного проекта должна являться схема электрическая
принципиальная системы тестирования сервоприводов, выданная
заказчиком, а также другая документация.

\section{Технические требования}

\subsection{Состав изделия} и требования к конструктивному исполнению
устройства:

\subsubsection{Устройство системы тестирования сервоприводов должно содержать}

\begin{itemize}
\item микроконтроллер;
\item четыре потенциометра;
\item разъёмы для подключения четырех сервоприводов;
\item разъёмы для подключения по интерфейсу SPI;
\item разъёмы для подключения по интерфейсу I$^2$C
  
\item кварц для тактирования микроконтроллера;  
\item микросхему регулятора напряжения.
\end{itemize}

  \subsubsection{Устройство системы тестирования сервоприводов должно
    изготавливаться}
  в соответствии с требованиями настоящего ТЗ, ТУ и комплекта конструкторской
  документации.

  \subsubsection{По внешнему виду устройство должно}
  соответствовать опломбированному и утвержденному образцу.
  
  \subsubsection{Корпус, передняя панель, }
  органы управления и другие детали наружной поверхности устройства не
  должны иметь дефектов, портящих внешний вид устройства (вмятин, следов
  коррозии, царапин, трещин и других механических повреждений).
  
  \subsubsection{На устройстве должны быть нанесены условные функциональные 
  обозначения (символы), поясняющие назначен органов управления
  по ГОСТ 25874-83.}
  
  \subsubsection{Тумблеры управления
  должны обеспечивать не менее 9000 переключений.}
  
  \subsubsection{Усилие переключения тумблера — не более 15Н.}
  
  \subsubsection{Задняя стенка должна быть надежно прикреплена к корпусу.}

  \subsubsection{Сетевой шнур должен соотвествовать требованиям ГОСТ 7399-90}
  и ГОСТ 12.2.007.0-75. Длина сетевого шнура от места выхода из корпуса
  (стенки) до основания контактной вилки должна быть не менее 1 м.
  
  \subsubsection{Масса блока, измеренная с погрешностью плюс-минус 0,1 кг,}
  должна быть не более 4 кг.
  
  \subsubsection{Габаритные размеры устройства},
  измерненные с погрешностью плюс-минус 1 мм, должны быть не более, мм:
  высота - 150 мм, ширина 200 мм, длина 300мм.

  \subsubsection{Защита от электрических разрядов}
  должна определяться по ГОСТ 12.1.030-81.

  \subsubsection{Структура устройства и его конструктивное выполнение}
  должны обеспечивать объедение составных частей в единый базовый
  конструктив.

  \subsubsection{Устройство должно быть оснащено}
  тумблером для включения или выключения питающего напряжения.

  \subsubsection{Конструкция устройства}
  должна обеспечивать свободный доступ к составным элементами изделия
  при проведении наладочных и ремонтных работ.

  \subsubsection{Питание устройства} должно осуществляться от
  производственной сети кабелем в трехжильном исполнении, оснащенной
  трехжильной вилкой, либо батареей с напряжением 5 В.

  \subsubsection{По устойчивости к воздействию температуры}
  и влажности окружающей среды устройство должно соотвествовать
  климатическому исполнению к  категории размещения УХЛ 4.2 по
  ГОСТ 15150-69ю

  \subsubsection{Конструкция устройства по средствам защиты}
  от механических воздействий должна соотвествовать ГОСТ 26568-85.
  
  \subsubsection{Для антикоррозийной защиты }
  поверхностей блоков, деталей и изделий в целом применить
  гальваническое и лакокрасочные покрытия.
  
  \subsubsection{Конструктив устройства должен быть обеспечен}
  винтом общего заземления устройства.
  
  \subsubsection{Винт заземления по предыдущему пункту должен }
  располагаться в месте доступном для монтажа и визуального контроля.

\subsection{Показатели назначения:}

\subsubsection{Максимальная потребляемая мощность}
не более 100ВТ.

\subsubsection{Устройство должно позволять производить проверку сервоприводов
  в ручном режиме.}

\subsubsection{Частота подключаемых сервоприводов } устанавливается
потенциометрами.


\subsection{Требования к надёжности: }
\subsubsection{Показатели надежности (ГОСТ 27.003-2016) должны}
соотвествовать заданным значениям при нормальных климатических
условиях (температура окружающей среды плюс-минус 20ºC, относительная
влажность 60 \%, атмосферное давление $(958...1037) \cdot  10^2$ ПА;
c отклонениями напряжения сети 220В от плюс 10\% до минут 15\%
от номинального значения, частотой (49...51) Гц.

\subsubsection{Средний срок службы блока }
 должен быть не менее 30 лет.

 \subsubsection{ По нормам надежности блок должен }
 удовлетворять требованиям ГОСТ 27.003-2016. Время наработки на отказ
 должно быть не менее 10000 часов. Среднее время восстановления
 устройства должно быть не более 60 минут.
 
 \subsubsection{После восстановления работоспособности, }
 по окончании ремонтных работ при его отказе, изделие
 должно сохранять показатели назначения, изложенные в настоящем
 техническом задании.

 \subsubsection{Устройство должно выдерживать}
 воздействия внешних
 механических и климатических факторов в соответствии с ГОСТ 22261-94
 для 3 группы аппаратуры.

 \subsection{Требования к технологичности и
   метрологическому обеспечению разработки,
   производства и эксплуатации.}

 \subsubsection{Показатели технологичности конструкции изделия согласно ГОСТ 14.201-83.}
 
\subsubsection{Трудоемкость изготовления блока – не более 12 часов.}

\subsubsection{Параметры устройства должны контролироваться с
  помощью стандартных измерительных приборов
  обслуживающим персоналом средней квалификации.}

\subsubsection{Конструкция устройства
  должна обеспечивать возможность
  выполнения монтажных работ
  с соблюдением требований технических условий
  на установку и пайку комплектующих изделий.}

\subsubsection{Конструкция устройства в целом
  и отдельных сложных узлов
  должна обеспечивать сборку при изготовлении без создания и применения
  специального оборудования.}

\subsubsection {При изготовлении устройства должны применяться стандартные
  методы и универсальные средства измерений,
  серийное испытательное оборудование.
  Допускается для проведения климатических проверок
  при технологическом прогоне применять специально
  приготовленную камеру или специально оборудованное оборудование.}

\subsubsection{ Конструкция устройства должна соответствовать
  требованиям ремонтопригодности по рекомендациям Р 50-84-88.}

\subsection{Требования к уровню стандартизации и унификации}

\subsubsection{В качестве комплектующих единиц и деталей
  (коммутационные, изделия электроники, крепежные, установочные)
  должны применяться серийно выпускаемые изделия.}

\subsubsection{Сборочные единицы типа монтажных плат,
  панелей, крепежных и установочных узлов должны быть унифицированными.}

\subsubsection{В конструкции устройства должны быть заимствованы
  сборочные единицы, узлы и детали из ранее разработанных изделий.}

\subsubsection{ Коэффициент унификации стандартных и заимствованных
  деталей должен быть не менее 0,5.}

\subsection{Требования безопасности и
  требования по охране природы}

\subsubsection{Общие требования безопасности к конструкции устройства
  должны соответствовать ГОСТ 12.2.007.0-75.}

\subsubsection{Устройство по способу защиты человека от поражения
  электрическим током относится к классу 01 согласно ГОСТ 12.2.007.0-75.}

\subsubsection{Конструкция устройства должна исключать
  возможность неправильного присоединения
  его сочленяемых токоведущих и составных частей.}

\subsubsection{В качестве источника питания должна применяться
  сеть переменного тока частотой 50 Гц и напряжением 220 В.}

\subsubsection{Органы управления должны быть снабжены надписями,
  соответствующими их принадлежности и назначению}

\subsubsection{Присоединительные разъемы электрических цепей
  должны быть снабжены надписями,
  соответствующими их принадлежности и назначению.}

\subsubsection{Коммутационные изделия,
  устанавливаемые в цепях повышенного напряжения,
  должны быть конструктивно выделены
  и не должны одновременно коммутировать другие цепи.}

\subsubsection{Конструкция устройства должна исключать попадание
  внутрь посторонних предметов.}

\subsubsection{В эксплуатационных документах по требованиям
  техники безопасности должны быть соблюдены правила технической
  эксплуатации электроустановок потребителем и правила техники
  безопасности при эксплуатации электроустановок потребителем.}

\subsection{Эстетические и эргономические требования}

\subsubsection{Лицевая панель должна быть темной.}

\subsubsection{Устройство по своим эргономическим показателям
  должно обеспечивать удобство эксплуатации в производственных условиях.}

\subsubsection{Органы управления и индикации должны быть
  расположены с достаточным обзором и удобством использования.}

\subsection{Требования к патентной чистоте}

\subsubsection{Патентная чистота стенда должна быть обеспечена
  в отношении стран СНГ и стран возможных импортеров изделия.}

\subsection{Условия эксплуатации,
  требования к техническому обслуживанию и ремонту}

\subsubsection{Устройство должно быть выполнено для климатического
  исполнения УХЛ 4.2 согласно ГОСТ 15150-69
  и нормально функционировать при следующих климатических условиях:}

\begin{itemize}
\item верхнее значение температуры окружающей среды плюс 40ºC
\item нижнее значение температуры окружающей среды плюс 1ºC 
\item среднее (рабочее) значение температуры окружающей среды плюс 20ºС;  
\item атмосферное давление 650-800 мм рт. ст. (86-106 кПа);
\item относительная влажность воздуха не более 80\% при темпера­туре плюс 25 0С.
\end{itemize}

\subsubsection{Время подготовки устройства к эксплуатации после
  транспортировки и хранения не должно превышать 45 минут.}

\subsubsection{Рабочий режим в устройстве должен устанавливаться не более чем через 5 минуту после включения.}

\subsubsection{Ремонт устройства должен производиться в
  специализированной ремонтной организации или по месту эксплуатации
  высококвалифицированным радиомехаником.}

\subsection{Требования к маркировке и упаковке}

\subsubsection{Маркировка устройства должна соответствовать
  требованиям ГОСТ 26828-86.}

\subsubsection{Маркировка устройства и входящих составных частей должна содержать:}
\begin{itemize}
\item товарный знак или наименование предприятия-изготовителя;
\item полное торговое наименование;  
\item порядковый номер изделия и составных частей;
\item необходимые поясняющие и предупреждающие надписи,
  выполненные по ГОСТ 12.2.007.0-75;
\item дату изготовления.
\end{itemize}

\subsubsection{Упаковка должна быть выполнена в виде картонного ящика
  с пенопластовыми вкладышами.}

\subsubsection{На таре должны быть нанесены манипуляционные знаки
  «Осторожно, хрупкое»,
  «Боится сырости»,
  «Соблюдение интервала температур», «Верх, не кантовать» по ГОСТ 14192-77}


\subsubsection{При поставке изделия на экспорт все надписи
  выполняются на языке, оговоренном в договоре на поставку.}

\subsection{Требования к транспортированию и хранению}

\subsubsection{Упакованные изделия перевозить только в закрытом транспорте.}

\subsubsection{ Требования к виду транспорта не предъявляются.}

\subsubsection{Условия транспортирования изделия должны
  соответствовать следующим требованиям: }
\begin{itemize}
\item температура воздуха от минус 50 С до плюс 50 ºС;
\item относительная влажность воздуха 95\% при температуре плюс 30 ºС;
\item атмосферное давление от 84 до 107 кПа (от 630 до 800 мм рт. ст.).  
\end{itemize}


\subsubsection{Размещение и крепление упакованных изделий
  в транспортных средствах должно обеспечивать их устойчивое положение,
  исключить возможность ударов их друг о друга.}


\subsubsection{Устройство должно храниться в предназначенной для него таре
  в закрытых складских помещениях
  при температуре от плюс 5 ºС до плюс 35 ºС и влажности
  85\% на подставках.}

\subsubsection{Расстояние между стенами, полом хранилища
  и изделием должно быть не менее 100 мм, а между отопительными
  устройствами не менее 0,5 м.}

\subsection{Экономические показатели}

\subsubsection{Сравнительные характеристики разрабатываемого
  изделия на основе источников разработки.}

\subsubsection{Предполагаемый годовой выпуск до 2000 устройств в год. }

\subsection{Порядок контроля и приемки}

\subsubsection{Для приемки работы на этапе проведения испытаний
  необходимо представить минимум три стенда.}

\subsubsection{Испытания должны проводиться по программе и методике испытаний,
  утвержденной заказчиком.}

\subsubsection{Для приемки представляются следующие документы:}
\begin{itemize}
\item техническое задание;  
\item комплект конструкторской документации;  
\item ведомость покупных изделий;  
\item программа и методика испытаний;  
\item эксплуатационные документы;  
\item методики проверки.  
\end{itemize}



\subsubsection{Приемочные испытания проводит разработчик,
  приемосдатчик, изготовитель.}

\subsubsection{Приемочные испытания опытного образца производятся в сроки,
  согласованные с заказчиком.}


\subsubsection{Аттестацию опытного образца проводит разработчик
  с участием заказчика,
  а остальных образцов – изготовитель.}


\end{document}
%%% Local Variables:
%%% mode: LaTeX
%%% TeX-master: t
%%% End:
