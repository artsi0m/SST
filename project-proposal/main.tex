\documentclass[a4paper]{bsuir-std}

\usepackage[sorting=none,backend=biber]{biblatex}
\addbibresource{Karakin7semCoursework.bib}

\begin{document}

%%% Титульный лист
\begin{titlepage}
\begin{center}
Министерство образования Республики Беларусь\\
Учреждение образования\\
«Белорусский государственный университет \\
информатики и радиоэлектроники»\\[1.2em]

Факультет компьютерного проектирования\\
Кафедра проектирования информационно-компьютерных систем\\
\end{center}

\vfill

% They done it using table in original document btw
\begin{minipage}{8cm}
  \begin{flushleft}
    \center{СОГЛАСОВАНО}\\
    \raggedright % выравнивание
    Руководитель проекта\\
    доцент\\
    \underline{\hspace*{2cm}} ~В.~C. Колбун\\
    \underline{\hspace*{0.5cm}}.\underline{\hspace*{0.5cm}}.2025\\
  \end{flushleft}
\end{minipage}
\hfill
\begin{minipage}{8cm}
  \begin{flushright}
    \center{УТВЕРЖДАЮ}\\
    \raggedright
    Заведующий кафедрой ПИКС\\
    канд.техн.наук, доцент\\
    \underline{\hspace*{2cm}} ~В.~В. Хорошко\\
    \underline{\hspace*{0.5cm}}.\underline{\hspace*{0.5cm}}.2025\\
  \end{flushright}
\end{minipage}


\vfill
\begin{center}
  ТЕХНИЧЕСКОЕ ЗАДАНИЕ\\
  на опытно-конструкторскую разработку
  «Система тестирования сервоприводов квадрокоптера»
\end{center}

\vfill
\begin{flushright}
  \begin{minipage}{6cm}
    \center{Исполнитель}\\
    \raggedright
    Студент  группы 112601\\
    \underline{\hspace*{2cm}} ~А.~Л.Корякин\\
    \underline{\hspace*{0.5cm}}.\underline{\hspace*{0.5cm}}.2025\\
  \end{minipage}
\end{flushright}


\vfill
\begin{center}
    {\normalsize Минск 2025}
\end{center}

\end{titlepage}

%%% Local Variables: 
%%% coding: utf-8
%%% mode: latex
%%% TeX-engine: xetex
%%% End:


\section{Анализ литературно \\
  патентных исследований}

В данном разделе будет приведена информация про то что именно из
себя представляет разрабатываемое устройство, а именно то,
что оно является полётным контроллером.

Также пять патентов схожих по параметрам
устройств, выполняющих функции полётных контроллеров.

\section{Общетехническое обоснование \\
  разработки устройства}

Приведена информация про режимы работы устройства, то для чего именно
оно будет использоваться, его соответствие категории из ГОСТ 151150-69.


\section{Схемотехнический анализ \\
  радиоэлектронного средства}

Рассказывается про то как функционирует устройство, на основании того
как объяснено функционирование принципиальной схемы, взятой из
журнала. Приведен расчёт электрических параметров потенциометров и
транзистора из схемы. Расчёт именно этих параметров обоснован тем, что
они будут необходимы в дальнейшем для расчёта выделяемой теплоты.
Возможно, будет использован САПР \textit{SimulIDE} вместо расчёта
схемы в ручную.


\section{Разработка конструкции \\
  проектируемого изделия} % 4

\subsection{Выбор и обоснование элементной базы,\\
  конструктивных элементов, установочных изделий, \\
  материалов корпуса и защитных покрытий,\\
  маркировки деталей и сборочных единиц. }

Приведены виды электрорадиоизделий, согласно ~\cite{Belyanin2008}.
Приведены основные параметры электрорадиоэлементов, согласно
~\cite{Alexeev2011}.
Рассказано на основании каких соображения осуществлялся выбор тех или
иных элементов.

\subsection{Выбор типа электрического монтажа,\\
  элементов крепления и фиксации. }

В этом подразделе рассказано то почему выбран именно THT монтаж в
данном изделии.

\subsection{Выбор способов обеспечения \\
  нормального теплового режима устройства \\
  (выбор способа охлаждения \\
  на ранней стадии проектирования;\\
  выбор наименее теплостойких элементов, \\
  для которых необходимо проведение расчёта). }

Говорится, что выбрано пассивное охлаждение из-за своей простоты.

Рассказано почему в качестве наименее теплостойкого элемента будет
выбран единственный транзистор на плате, либо микроконтроллер. На
данный момент склоняюсь к выбору транзистора.

\subsection{Выбор и обоснование \\
  метода изготовления печатной платы. }

В данном разделе будет приведена информация про несколько оптимальных
с разных точек зрения методов изготовления печатный платы. Будет
указано почему выбран один конкретный метода. На данный момент таким
методом выбран метод фрезерования ~\cite{PirogovaEngineering}.

\subsection{Выбор конструкторских решений, \\
  обеспечивающих удобство ремонта \\
  и эксплуатации устройства.}

Рассказано о том, как  может быть осуществлено перепрограммирование устройства с
помощью разъёма ISP, без выпаивания микроконтроллера из печатной
платы. Обосновывается выбор THT компонентов, как облегающих ремонт.
Возможно, обосновывается добавление резистора, находящегося
между транзистором и микроконтроллером, как меры безопасности.

\subsection{Технология разработки чертежа детали в среде KiCAD.}

Рассказано, как разработан чертёж принципиальной схемы и печатные
платы в САПР KiCAD. Обоснован выбор именно свободного ПО, типа САПР
KiCAD.  Рассказано как проходит разработка в данной САПР, а также как
может осуществляться контроль разрабатываемых файлов с помощью \textit{git}.

\subsection{Обеспечение требований стандартизации, \\
  унификации и технологичности конструкции устройства.}

Приведены ГОСТы из листа задания. Рассказано о том, что именно
предпринималось, чтобы изделия соответствовало данным нормативным
документам.

\section{Расчет параметров проектируемого изделия}

\subsection{Расчет теплового режима \\
  (выбор способа охлаждения; \\
  описание тепловых моделей; \\
  оценка теплового режима). }

Будет приведены аргументы в пользу выбора пассивного охлаждения в
данном изделии, касающиеся простоты изделия и отсутствия каких-либо
устройств выделяющих большой объём теплоты.

Приведены расчёты соответствующие пассивному охлаждению в герметичном
корпусе и в перфорированном корпусе.

\subsection{Расчёт на механические воздействия. }

По толщине, длине, ширине и массе печатной платы будет определена
собственная частота печатной платы согласно ~\cite{Kalenkovich2012}
года издания.

\subsection{Расчет конструктивно-технологических параметров\\
  печатных плат. }

Основываясь на~\cite{Kostukevich2012}, приведен расчёт
объёмно-компоновочных характеристик устройства на основе данных о
имеющихся на ней компонентах и коэффициенте заполнения.

\subsection{Расчёт электромагнитной совместимости}

Согласно одному из актуальных пособий будет выбрана методика расчёта
электромагнитной совместимости о осуществлён расчёт на основании этой
методики.

\subsection{Полный расчёт надёжности. }

Будет приведен полный расчёт надёжности на основании данных о
компонентах из перечня элементов и данных полученных в ходе расчетов
электрических параметров из главы 3.

\section{Моделирование \\
  физических процессов, \\
  протекающих в проектируемом\\
  радиоэлектронном средстве}

В данном разделе будет приведены данные о моделировании устройства в
следующих программах:
\begin{itemize}
\item \textit{Ansys},
  
\item \textit{COMSOL Multiphysics},
\item \textit{Solidworks Simulation},
  
\item \textit{Solidworks Flow Simulation.}
\end{itemize}

Будет приведена инфомарция касательно результатов моделирования данных
программах.

\section{Экономическое обоснование}

В этой главе будет приведён расчёт экономического обоснование
внедрения в эксплуатацию программно-аппаратного комплекса.

Пользователем данной разработки будут разработчики и производители
беспилотных летательных аппаратов мультироторного типа.

Данная разработка решает проблему полуавтоматического управления
двигателями мультироторного летательного аппарата.

Разработка окажется для пользователя более предпочтительных из-за
простоты производства и эксплуатации.

\newpage


\renewcommand{\refname}{\textbf{Cписок использованных источников}}
\DeclareFieldFormat{url}{Режим доступа\addcolon\space\url{#1}}
\DeclareFieldFormat{title}{{#1}}
\DeclareFieldFormat{labelnumberwidth}{[{#1}]\adddot  }
\printbibliography[title={Cписок использованных источников}]

\newpage
%%% Local Variables:
%%% mode: LaTeX
%%% TeX-master: "main"
%%% End:


\end{document}

%%% Local Variables:
%%% mode: LaTeX
%%% TeX-master: t
%%% LaTeX-biblatex-use-Biber: t
%%% End:
