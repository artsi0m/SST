\section{Разработка конструкции \\
  проектируемого изделия}

\subsection{Выбор конструкторских решений, \\
  обеспечивающих удобство ремонта\\
  и эксплуатации устройства }

По ГОСТ 23751-86 предусмотрены следующие типы конструкции ПП
~\cite{PirogovaEngineering}:
\begin{enumerate}

\item Односторонние ПП. Применяются в бытовой технике, технике связи
  и в блоках питания на ЭРЭ. Имеют низкую стоимость,
  высокую надежность, низкую плотность компоновки;

\item двусторонние ПП. Применяются в измерительной, вычислительной
  технике, технике управления и автоматического регулирования,
  технике связи, высокочастотной технике;

\item многослойные ПП. Применяются в технике управления и
  автоматического регулирования,
  вычислительной и бортовой аппаратуры для
  коммутации ИМС, БИС, СБИС, МСБ, в ЭА с высокими требованиями по быстродействию,
  плотности монтажа,  волновому сопротивлению, времени задержки сигнала и т.д.;

\item гибкие ПП, ГЖП и ГПК. Применяются в ЭА и высокой надежности
  при реализации уникальных и сложных технических решений,
  конструкция которых исключает применение жестких ПП.
\end{enumerate}

Чем больше слоев, тем легче трассировать печатную плату.
Однако большее число слоёв затрудняет производство печатной платы
в условиях лаборатории, в которых может использоваться разрабатываемое
устройств.

Оптимальным с точки зрения трассировки было бы проектирование
многослойной печатной платы, с четырьмя слоями, при этом на первом
внутреннем слое полигон земли, а на втором полигоны питания.

Оптимальным же с точки зрения изготовления было бы проектирование
двухсторонней ПП с использованием метода фрезерования.

Этот механический метод применяют при единичном изготовлении ДПП
полностью на одном универсальном станке фирм, например, LPFK, Bungard,
VHF, Mitronic~\cite{PirogovaEngineering}.
Он включает следующие этапы:
\begin{enumerate}
\item подготовка управляющего файла для станка;
  
\item сверление монтажных и переходных отверстий по программе;
\item фрезерование (высвобождение) мест от фольги твердосплавными коническими фрезами
  с углом при вершине 60 или 30 град.
  Файл оконтуривания генерируется в одной из программ \textit{CAD-CAM}
  (\textit{InstantCAM, CircuitCAM, САМ 350});
  
\item металлизация монтажных и переходных отверстий.

\end{enumerate}

Применяя фрезу с рабочим диаметром 100 мкм,
изготавливают ДПП по 4-му классу точности,
т.е. между выводами ЭРИ со стандартным шагом 2,54 мм можно провести пять проводников.
При этом основным требованием, предъявляемым к материалу ДПП, является плоскостность.
Для исключения разброса ширины реза при фрезеровании применяют
специальные прижимные головки.
При этом контролируют глубину врезания фрезы
в заготовку и равномерность прижима заготовки к рабочему
столу.

Металлизацию переходных отверстий
осуществляют пустотелыми
заклепками; обслуженными пустотелыми заклепками, содержащими припой с флюсом,
которые вставляют в отверстие, и с помощью паяльника расплавляют припой
(\textit{LPFK}); специальными пастами,
которые разогревают в печах при температуре $160 \pm 10$°C.

Основными преимуществами механического способа являются высокая
оперативность и простота реализации, а недостатками — низкая
производительность и высокая стоимость оборудования~\cite{PirogovaEngineering}.


В итоге был выбран вариант проектирования с двухсторонней печатной платой,
ввиду доступности изготовления ПП таким способом в условиях её
эксплуатации.




\subsection{Выбор типа электрического монтажа,\\
  элементов крепления и фиксации }

% Тут обосновать использование THT компонентов и зажима устройства в корпус.

Во время проектирования был сделан выбор в пользу того, чтобы размещать
THT компоненты только с одной стороны печатной платы.
Такое решение позволяет ~\cite{yadro-habr-764056}:
% https://habr.com/ru/companies/yadro/articles/764056/
\begin{enumerate}
\item в полтора-два раза сократить время монтажа и расходы на трафареты;
\item реже перезаряжать питатели и корректировать термопрофиль;  
\item снизить нагрузку на выходной контроль.
\end{enumerate}

Таким образом была повышена простота производства и ремонтопригодности.

Существуют следующие опции при выборе типа монтажа на печатной плате:
\begin{itemize}
\item поверхностный монтаж;
\item выводной монтаж;
\item смешанный монтаж.
\end{itemize}

Появление компонентов с большим числом выводов привело к появлению
планарных компонентов и использованию поверхностного монтажа для их
установки.

В данном проекте отсутствуют компоненты со столь большим числом
выводов. У самого большого по числу выводов компонентов
микроконтроллера \textit{ATmega328p} всего лишь 28 выводов, а на рынке
представлены \textit{DIP} корпуса для установки микроконтроллера
монтажом в отверстия.

Широкое распространение поверхностный монтаж получил к концу
восьмидесятых годов. Новизна заключается в использовании вместо
компонентов на выводах, вставляемых в отверстия платы, применение
компонентов припаиваемых к контактным площадкам, сформированным
проводящим рисунком. Планарные компоненты не имеют выводов совсем или
редко имеют короткие выводы. Отсутствие отверстий для установки
компонентов снижает затраты на изготовление платы. Планарные
компоненты унифицированы, в несколько раз меньше, вдвое дешевле
выводных аналогов. Модули, собранные по технологии поверхностного
монтажа имеют плотное размещение компонентов, малое расстояние между
компонентами и контактными площадками. Уменьшение длины проводников
улучшает передачу высокочастотных и слабых сигналов, уменьшается
нежелательная индуктивность и емкость. Планарные радиоэлементы имеют
низкую цену. Поверхностный монтаж сегодня распространен намного шире
монтажа в отверстия. Постоянно снижается себестоимость сборки
~\cite{платы.рф-монтаж}.

Поверхностный монтаж обладает рядом недостатков. Жесткое крепление
компонента за корпус к проводящему рисунку приводит к разрушению
компонентов, подвергающихся воздействию перепадам температур. Модули,
собранные из планарных компонентов боятся перегрева при пайке, сгибов
и ударов. Эти воздействия приводят к трещинам компонентов. Разработчик
печатных плат должен проектировать проводящий рисунок, обеспечивающий
равную скорость нагревания контактов компонента благодаря
симметричности тепловых полей. Технология групповой пайки включает в
себя режим работы оборудования и технологическую оснастку
обеспечивающие одинаковую скорость нагревания контактов каждого
компонента для исключения брака. Требуется точно соблюдать требования
переноса пасты на плату и режим работы паяльного
оборудования. Повышаются требования к транспортировке и хранению
планарных компонентов и материалов для монтажа. Отработка трассировки
проводящего рисунка требует больше средств. Возрастают затраты на
технологическую оснастку при выпуске опытных партий. Ремонт модулей
собранных поверхностным монтажом требует специализированного
инструмента ~\cite{платы.рф-монтаж}.

Данный тип монтажа не был выбран из соображений ремонтопригодности
получаемого изделия.

Ручной выводной монтаж модулей целесообразно использовать в следующих
случаях: применение автоматического оборудования невыгодно из-за
малого объема заказа или сборки нескольких макетных образцов модулей,
платы не подходят для автоматизированного монтажа, при окончательном
монтаже выводных элементов после автоматического монтажа. Сегодня
электроника находится на уровне развития не позволяющем полностью
отказаться от ручных операций монтажа. Монтажник тщательно проверяет
внешний вид каждого компонента перед установкой. При необходимости
выполняется очистка выводов от окислов и лужение выводов. Есть
возможность придания выводам каждого компонента, формы наиболее
оптимальной для установки на плате, обусловленной конструкцией
электронного модуля. Ручная формовка позволяет придать форму выводам
компонентов, облегчающую чтение маркировки~\cite{платы.рф-монтаж}.

Выбор данного типа монтажа в данной плате обусловлен
стремлением повысить простоту производства платы в условиях эксплуатации.

\subsection{Выбор способов защиты устройства от внешних воздействий.}

Отталкиваясь от выбранной схемы, можно сделать вывод, что для защиты
от помех возникающих в сети питания необходим ферритовый фильтр,
устанавливаемый на провод блока питания используемого устройством.

Для того чтобы предотвратить повреждение и деформацию плат или
защитить их от внешних воздействий применяются различные покрытия:
лаки, смолы, различные жидкости. Качественное дополнительное покрытие
надежно защищает оборудование от пыли и влаги, что особенно важно для
обеспечения бесперебойной работы производства~\cite{energota-varnish-or-compaund}.

Компаунд, является менее известное средство защиты, однако по
качеству не уступает лаку.

Преимуществ лаков таковы ~\cite{energota-varnish-or-compaund}:
\begin{enumerate}
\item  доступность: как правило, лак на 50\% дешевле других защитных материалов;
\item вариативность: можно выбрать конкретный вид лака, в зависимости оттого, какие свойства нужны для защиты;
\item специальный контроль: после нанесения лак просвечивают под УФ-лампой, чтобы было видно, насколько хорошо покрыта плата. Это гарантирует надежность и качество покрытия.
\end{enumerate}

К преимуществам компаунда относятся~\cite{energota-varnish-or-compaund}:

\begin{enumerate}
\item повышенная механическая устойчивость: эффективно защищает от
вибраций, возникающих в процессе работы;
\item высокая теплоотдача: радиоэлементы не перегреваются в процессе
эксплуатации;
\item универсальность: подходит для всех плат;
\item визуальный контроль: для того, чтобы понять, насколько хорошо
покрыта плата, не требуется специальное оборудование.
\end{enumerate}


Следует отметить, что средства также различаются и по области
применения: покрытие лаком лучше использовать в сухих помещениях,
покрытие компаундом – в помещениях с повышенной влажностью и
агрессивной средой (среда, в которой благодаря воздействию внешних
факторов образуется коррозия на печатных платах)~\cite{energota-varnish-or-compaund}.


Использования компаунда противоречит интенции повысить простоту
изготовления печатной платы, из-за которой уже были сделаны несколько
инженерных решений, таких как изготовление именно двухслойной печатной
платы, монтаж компонентов на одной стороне платы, использование монтажа
ручного выводного монтажа.

Таким образом был сделан выбор в пользу использования лаков, потому что эта
опция позволяет повысить ремонтопригодность изделия.

\subsection{
  Выбор способов обеспечения \\
  нормального теплового режима устройства \\
  (выбор способа охлаждения \\
  на ранней стадии проектирования; \\
  выбор наименее теплостойких элементов, \\
  для которых необходимо проведение теплового расчета).}

Защита от тепловых воздействий это одна из важных задач решаемых в
обеспечении надёжности РЭС.
Насыщение современных технических устройств РЭА различного назначения
заставляет конструкторов уменьшать ее габариты и увеличивать удельные
мощности рассеивания, т.е. мощности, приходящиеся на единицу
поверхности или объема РЭА. Одним из основных направлений в
конструировании РЭА стала комплексная микроминиатюризация, что
приводит к ещё большему увеличению удельной мощности рассеивания
~\cite{Rotkop1976}.

Различают внутренние и внешние тепловые воздействия на РЭА.
Внутренние тепловые воздействия на РЭА в основном зависят от мощности
рассеиваемой её элементами ~\cite{Rotkop1976}.

Защита РЭА от тепловых воздействий осуществляется при помощи ряда
мероприятий. Одним из основных является использование систем
обеспечения теплового режима РЭА (СОТР). СОТР обычно предназначена для
поддержания заданного в технических условиях (ТУ) диапазона температур
на элементах РЭА, чтобы обеспечить ее надежность при определенных
тепловых воздействиях и других специальных требованиях ~\cite{Rotkop1976}.

В радиоэлектронных комплексах СОТР, как правило, являются сложными
системами, состоящими из многих элементов, коммуникаций и несущих
конструктор. В некоторых случаях регулирование температуры в РЭА может
быть достигнуто за счет простейших конструктивных решений,
осуществляющих теплопередачу между элементами РЭА, элементами несущей
конструкции и окружающей средой. Тогда нет смысла рассматривать СОТР
как отдельное изделие и мы будем пользоваться терминами «методы (или
способы) охлаждения РЭА». Этими же терминами будем пользоваться и при
исследовании температурного поля элементов РЭА, в результате действия
некоторых гипотетических СОТР, когда конкретная конструкция СОТР не
рассматривается ~\cite{Rotkop1976}.

Для выбора и обоснование системы охлаждения важно иметь представление
о тепловом режиме РЭС.  Тепловой режим есть совокупность значений
температур в различных точках всей РЭС, её корпуса и СОТР.

Тепловой режим РЭА характеризуется, прежде всего, двумя факторами:
электрическим режимом работы и условиями эксплуатации ~\cite{Rotkop1976}.

Энергетический коэффициент полезного действия радиоэлементов, как
правило, невелик, и значительная доля энергии питания превращается в
тепловую энергию с сопутствующим перегревом элементов и аппаратуры
~\cite{Rotkop1976}.


Электрический режим работы РЭА в данном случае интересует нас только в
связи с изменением внутренних тепловых воздействий во времени и
пространстве и задается обычно в виде графиков зависимости
рассеиваемой мощности от времени для различных узлов РЭА~\cite{Rotkop1976}.

Из технического задания известно, что режим работы данного РЭС —
длительный. То есть устройство работает в течении достаточно большого
периода времени рассеивая постоянную по величине мощность.

Такой электрический режим не позволит использовать радиаторы в виде
так называемых тепловых аккумуляторов.

Данные касаемо условий эксплуатации были приведены ранее и
соответствуют условиям категории УХЛ 4.2 указанным в ГОСТ
~\cite{GOST-15150-69}.

Понятию «достаточно большой промежуток времени» в контексте охлаждения
РЭС соответствует такой промежуток времени, за который полностью
устанавливается тепловой режим этого РЭС.

Учитывая тип и состояние теплоносителя, также причину, вызвавшую его
движение, способы охлаждения РЭА можно разделить на следующие основные
классы: газовое (воздушное), жидкостное, испарительное, а также
естественное и принудительное ~\cite{Rotkop1976}.

Здесь некоторый способ охлаждения может относиться сразу к двум
классам: по агрегатному состоянию теплоносителя и тому принудительно
ли осуществляется охлаждение.

Естественное воздушное охлаждение РЭА является наиболее простым,
надежным и дешевым способом охлаждения и осуществляется без затраты
дополнительной энергии. Однако интенсивность такого охлаждения
невелика, поэтому использование этого способа возможно при небольших
удельных мощностях рассеивания (мощностях, рассеиваемых
единицей поверхности или объема), т.е. в РЭА, работающей в облегченном
тепловом режиме. При естественном воздушном охлаждении конвективный
теплообмен осуществляется за счет энергии,
рассеиваемой элементами РЭА ~\cite{Rotkop1976}.

Естественное воздушное охлаждение РЭА с перфорированным кожухом
позволяет обеспечить тепловой режим при более высоких удельных
мощностях рассеивания, чем при герметичном кожухе ~\cite{Rotkop1976}.

Однако в таком случае, ухудшаются показатели защищенности устройства
от пыли.

Расчет теплового режима радиоэлектронных аппаратов рекомендуется
проводить в три этапа~\cite{Rotkop1976}:
\begin{enumerate}[label={\arabic*.}]
  \item Определение среднеповерхностной температуры платы с
расположенными ней деталями, корпуса и температуры воздуха внутри
радиоэлектронного аппарата.
  \item Определить среднеповерхностные температуры корпусов элементов
  используя результаты первого этапа.
  \item Определить максимальные температуры критических зон элементов и
их функциональные связи со среднеповерхностной температурой как
корпусов, так и и плат.
\end{enumerate}

Первый и второй этапы расчета позволяют получить значения основных
параметров, связанных с выбором системы охлаждения, т.е. первых двух
этапов хватает для принятия конструкторского решения касаемо выбора
системы охлаждения.

Полную систему уравнений теплообмена для реального аппарата часто
невозможно не только решить аналитически, но и строго записать. В
связи с этим процессы, протекающие в реальном радиоэлектронном
аппарате, схематизируют, принимают ряд упрощающих предпосылок и в
результате получают тепловую модель аппарата, для которой и проводят
расчет теплового режима ~\cite{Rotkop1976}.

Наибольшее распространение получила весьма плодотворная схематизация
процессов теплообмена в РЭА, предложенная Г.Н.Дульневым
~\cite{Dulnev1968}.

Суть метода заключается в том, что печатная плата с её элементами
принимается за одно тело с изотермической поверхностью (нагретую
зону), для которого и проводится расчет теплового режима.

Таким образом производится расчет среднеповерхностной температуры
нагретой зоны.

Под понятием нагретая зона понимается поверхность того элемента на
печатной плате, который рассеивает больше всего мощности.

Во всей схеме практически не встречаются компоненты с большим
тепловыделением. Так например найти коэффициент \textit{TDP} для
микроконтроллера не представляется возможным, потому что среди
производителей микроконтроллеров, в отличие от производителей
микропроцессоров не принято не то что рассчитывать этот коэффициент
рассеивания теплоты, а даже закладывать в этот компонент схемы
возможности какого-либо его нагрева.

Однако это ни в коем случае не означает то, что ни один из элементов
не будет нагреваться. Напротив при работе любого электронного прибора
какая-то часть потребляемой мощности рассеивается как тепло.

Но если рассматривать отношение мощности потребляемой и рассеиваемой
как тепло, то можно понять, что самым простым оно будет у резисторов.

По видимому, можно взять информацию о мощности резистора и принять её
за мощность рассеиваемую в виде теплоты
~\cite{HeatDissipatedResistors}.

Примерно таким же образом было сделано допущение о схожести в вопросе
рассеиваемой мощности между резисторами и потенциометрами.

В принципе задача теплоотвода — удерживать переходы транзисторов или
других устройств при температуре, не превышающей указанной для них
максимальной рабочей температуре. Для кремниевых транзисторов в
металлических корпусах максимальная температура переходов обычно равна
200°C, а для транзисторов в пластмассовых корпусах 150°С
~\cite{ArtOfElectronics2010}.

Зная нужные параметры,
проектировать теплоотвод просто~\cite{ArtOfElectronics2010}:
Зная мощность, которую прибор будет рассеивать в данной схеме,
подсчитываем температуру переходов с учетом теплопроводности
транзистора, радиатора и максимальной рабочей температуры окружающей
транзистор среды. Затем выбираем такой радиатор, чтобы температура
переходов была максимально ниже указанной изготовителем
максимальной. Здесь разумно перестраховаться, так как при температурах
близких к максимальной, транзисторы быстро выходят
из строя~\cite{ArtOfElectronics2010}.

Однако, стоит заметить, что установка дополнительного радиатора
это увеличение количества компонентов, которые могут отказать, то есть,
если возможно, то такого искусственного увеличения количества компонентов
следует избегать.

Из вышеизложенного можно сделать вывод, что в качестве наименее
теплостойких элементов для которых нужно производиться расчёт следует
выбрать транзистор. Если хочется сделать расчёт более простым, то
резисторы, взяв за основу потенциометры.


% Обоснование почему охлаждать необходимо транзистор, а не микроконтроллер.


\subsection{Выбор и обоснование элементной базы, \\
  конструктивных элементов, \\
  установочных изделий, \\
  материалов конструкции и защитных покрытий, \\
  маркировки деталей и сборочных единиц.}


По конструктивному оформлению все ЭРИ (ИЭТ, ЭРЭ и ПМК)
различают~\cite{Belyanin2008}:
\begin{enumerate}
\item корпусные с планарными выводами, лежащими в плоскости основания
корпуса, с осевыми (отформованными) и штыревыми, перпендикулярными
основанию (традиционная элементная база);
\item  корпусные без выводов, с укороченными планарными или j-образными
выводами, уходящими под корпус; в виде матрицы шариковых вводов из
припоя и пр.; их называют микрокорпуса или поверхностно-монтируемые
компоненты (ПМК);
\item  бескорпусные ЭРЭ и ИС.
\end{enumerate}

По конструктивно-технологическому признаку в настоящее время
различают следующие корпуса~\cite{Belyanin2008}:
\begin{enumerate}
\item  металлостеклянные – стеклянное или металлическое основание,
соединенное с металлической крышкой с помощью сварки; выводы
изолированы стеклом;
\item  керамические – керамическое основание и крышка, соединенные
между собой пайкой;
\item  пластмассовые – пластмассовое основание и крышка, соединены
опрессовкой;
\item  металлополимерные – подложка с компонентами и выводами
помещаются в металлическую крышку и герметизируются заливкой
компаундом. Металлическая крышка обеспечивает эффективную влагозащиту,
отвод тепла от кристалла, снижает уровень помех.
\end{enumerate}

Выбор элементной базы производится на основе схемы электрической
принципиальной с учетом изложенных в ТЗ условий и требований.
Эксплуатационная надежность элементной базы в основном определяется
правильным выбором типа элементов при проектировании для использования
в режимах, которые не превышают предельно допустимые~\cite{Alexeev2011}.

Критерием отбора электрорадиоэлементов (ЭРЭ) является соответствие
технических и эксплуатационных характеристик ЭРЭ заданным условиям
работы и эксплуатации ~\cite{Alexeev2011}.

Основными параметрами при выборе ЭРЭ являются ~\cite{Alexeev2011}:
\begin{enumerate}
\item Технические параметры:
  
  \begin{itemize}
  \item номинальное значение параметров ЭРЭ согласно схеме
    электрической принципиальной прибора;
  \item допустимые отклонения величины параметра ЭРЭ от номинального
    значения;
    
  \item допустимое рабочее напряжение ЭРЭ;
    
  \item допустимая мощность рассеивания ЭРЭ; 
  \item диапазон рабочих частот ЭРЭ;
  \item коэффициент электрической нагрузки ЭРЭ;
  \end{itemize}

\item Эксплуатационные параметры:
  \begin{itemize}
  \item диапазон рабочих температур;
  \item относительная влажность воздуха;
  \item атмосферное давление;
  \item вибрационные нагрузки;    
  \item другие показатели.
  \end{itemize}  
\end{enumerate}

Кроме всех выше обозначенных критериев выбора ЭРЭ, активно
применялись принципы стандартизации и унификации ЭРЭ, с
целью получить такие преимущества как улучшение эксплуатационной и
производственной технологичности, снижение себестоимости выпуска
проектируемого изделия и уменьшение сроков проектирования изделия.

\newpage
%%% Local Variables:
%%% mode: LaTeX
%%% TeX-master: "main"
%%% End:
