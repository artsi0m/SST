
\section{Расчет \\
  конструктивно-технологических \\
  параметров проектируемого изделия}

\subsection{Расчет объемно-компоновочных характеристик устройства.}

Под компоновкой электронной аппаратуры понимается процесс размещения
комплектующих модулей, изделий электронной техники (ИЭТ) и деталей ЭА
на плоскости или в пространстве с определением основных геометрических
форм и размеров, а также ориентировочное определение массы изделия.
Напрактике задача компоновки чаще всего решается путем размещения
готовых элементов с заданными формами, размером и весом на плоскости с
учетом электрических, магнитных, механических, тепловых и других видов
связи. При компоновке нужно стремиться к тому, чтобы ~\cite{Kostukevich2012}:
\begin{enumerate}
\item обеспечивалось отсутствие заметных паразитных электрических и магнитных взаимосвязей,
  влияющих на технические характеристики изделия;
  
\item взаимное расположение элементов обеспечивало технологичность сборки и монтажа,
  легкий доступ для контроля, ремонта и обслуживания;
  
\item изделие удовлетворяло требованиям технической эстетики;
\item габариты и масса изделия были минимальными.
\end{enumerate}

Существуют различные способы компоновки аппаратуры.
Необходимо выполнить аналитический расчет компоновочных параметров,
в основе которого лежит представление геометрических параметров ЭА
в виде чисел ~\cite{Kostukevich2012}.

Исходными данными для компоновочного расчета являются
~\cite{Kostukevich2012}:
\begin{itemize}
\item перечень элементов;  
\item габаритные и установочные размеры ИЭТ.
\end{itemize}

Методика расчёта заключается в следующем ~\cite{Kostukevich2012}:
\begin{enumerate}
\item Определяется суммарная площать $S_{\textrm{ИЭТ}}$, занимаемая всеми ИЭТ:
  \begin{equation}
    S_{\textrm{ИЭТ}} = \sum^N_{i=1}S_{yi}
  \end{equation}

  где $S_{yi}$ - значение установочной площади \textit{i}-го элемента;
  \textit{n} —  количество элементов.
  
\item Рассчитывается приблизительная площадь печатной платы с учетом
  способа монтажа (односторонний, двусторонний):
  \begin{equation}
    S_{\textrm{Пл}} = \frac{S_{\textrm{ИЭТ}}}{(k_{\textrm{ЗПл}} \cdot m)}
  \end{equation}
  где $k_{\textrm{ЗПл}}$ — коэффициент заполнения платы печатной, как правило,
  должен быть в пределах от 0,3 до 0,8;
  m — количество сторон монтажа (1,2).
  В данном случае количество сторон монтажа равно $m=1$.
\end{enumerate}


Исходя из рассчитанной площади платы и высоты ИЭТ определить
приблизительные габаритные размеры~\cite{Kostukevich2012}.

При оценке приблизительных габаритных размеров всего устройства два
размера из трех определяют по рассчитанным размерам платы печатной с
учетом допусков на зазоры между платой и корпусом, толщины корпуса,
особенностей дизайна устройства и т. п.
Третий размер определяется с учетом максимально высоких элементов,
размещаемых на плате, и размеров, обусловленных особенностью
разрабатываемой конструкции (способ крепления платы в корпусе,
толщина корпуса, наличие дополнительных деталей на корпусе и т. п.) ~\cite{Kostukevich2012}.

Допускается выполнять предварительный расчет габаритных размеров
электронной аппаратуры по следующей методике ~\cite{Kostukevich2012}:
\begin{enumerate}
\item Определяется суммарный объем, занимаемый всеми ИЭТ и деталями:
  \begin{equation}
    V_{\textit{ИЭТ}} = \sum^N_{i=1}\vartheta_i +   \sum^N_{i=1}\vartheta_j
  \end{equation}

где $\vartheta_i$ - значение объёма \textit{i}-го ИЭТ;
$\vartheta_j$ — значение объёма \textit{j}-й детали;
\textit{n} — количество ИЭТ;
\textit{m} — количество деталей.
\item Оценивается приблизительный объем всего устройства:
  \begin{equation}
    V_{\textrm{У}} = \frac{V_{\textrm{ИЭТ}}}{K_{\textrm{З}}}
  \end{equation}
\end{enumerate}

\subsection{Расчёт теплового режима.} 
  % ДЛЯ РЭС В ГЕРМЕТИЧНОМ КОРПУСЕ}
Расчёт теплового режима для РЭС выполняется в указанном порядке ~\cite{Rotkop1976}:
\begin{enumerate}
  
\item Рассчитывается поверхность корпуса блока

  \begin{equation}
    S\mathrm{_{К}} = 2 \cdot (l_1 l_2 + (l_1+ l_2)l_3) % (4.46)
  \end{equation}

\item Определяется условная поверхность нагретой зоны:
  \begin{equation}
    S\mathrm{_{з}} = 2 (l_1 l_2 + (l_1 + l_2) K\mathrm{_{з}} l_3 ) % (4.39)
  \end{equation}

\item Определяется удельная мощность корпуса по блоку:

\begin{equation}
  q\mathrm{_к} = P\mathrm{_з}/S\mathrm{_к} % (4.45)
\end{equation}


\item Рассчитывается удельная мощность нагретой зоны:
  
  \begin{equation}
      q\mathrm{_з} = P\mathrm{_з}/S\mathrm{_3} % (4.38)
    \end{equation}


\item Находится коэффициент $\vartheta_1$ в зависимости от удельной мощности корпуса блока:
    
\begin{equation}
\vartheta_1 = 0,1472q\mathrm{_к} - 0,2962 \cdot 10^{-3}q\mathrm{_к}^2 + 0,3127 \cdot 10^{-6}q\mathrm{_к}^2
\end{equation}

\item Находится коэффициент $\vartheta_2$ в зависимости от удельной мощности нагретой среды:

\begin{equation}
\vartheta_2 = 0,1390q\mathrm{_к} - 0,1223 \cdot 10^{-3}q\mathrm{_к}^2 + 0,0698 \cdot 10^{-6}q\mathrm{_к}^3
\end{equation}

\item Коэффициент $K\mathrm{_{Н1}}$ в зависмости от давления
  среды вне корпуса блока берётся из ГОСТ~\cite{GOST-15150-69}.

  $$K\mathrm{_{Н1}} = 0,999$$

  \item Коэффициент $K\mathrm{_{Н2}}$ в зависмости от давления
  среды внутри корпуса блока берётся из ГОСТ~\cite{GOST-15150-69}.

  $$K\mathrm{_{Н2}} = 0,996$$

\item Определяется перегрев корпуса блока:

  \begin{equation}
    \vartheta\mathrm{_к} = \vartheta_1 \cdot K\mathrm{_{Н1}}
  \end{equation}

\item Рассчитывается перегрев нагретой зоны:

  \begin{equation}
    \vartheta\mathrm{_з} = \vartheta\mathrm{_к} + (\vartheta_2 - \vartheta_1) \cdot K\mathrm{_{H2}}
  \end{equation}

\item Определяется средний перегрев воздуха в блоке:

  \begin{equation}
      \vartheta\mathrm{_в} = 0,5 \cdot (\vartheta\mathrm{_к} + \vartheta\mathrm{_з})
    \end{equation}

\item Определяется удельная мощность элемента:
    \begin{equation}
      q\mathrm{_{эл}} = \frac{P\mathrm{_{эл}}}{S\mathrm{_{эл}}}
    \end{equation}

\item Рассчитывается перегрев поверхности элемента:

    \begin{equation}
      \vartheta\mathrm{_{эл}} = \vartheta\mathrm{_{з}} \left(a + b \frac{q\mathrm{_{Эл}}}{q\mathrm{_{з}}}\right)
    \end{equation}

\item Рассчитывается перегрев окружающей элемент среды:

    \begin{equation}
      \vartheta\mathrm{_{эс}} = \vartheta\mathrm{_в}
      \left(0,75 + 0,25\frac{q\mathrm{_{эл}}}{q\mathrm{_{з}}}\right)
    \end{equation}


\item Определяется температура корпуса блока:
    \begin{equation}
      T\mathrm{_к} = \vartheta\mathrm{_{к}} + T\mathrm{_с}
    \end{equation}

\item Определяется температура нагретой зоны:
    \begin{equation}
      T\mathrm{_з} = \vartheta\mathrm{_з} + T\mathrm{_c}
    \end{equation}


\item Находится температура поверхности элемента:
    \begin{equation}
      T\mathrm{_{эл}} = \vartheta\mathrm{_{эл}} + T\mathrm{_c}
    \end{equation}



\item Находится средняя температура воздуха в блоке:
    \begin{equation}
      T\mathrm{_{в}} = \vartheta\mathrm{_{в}} + T\mathrm{_c}
    \end{equation}



\item Находится температура окружающей элемент среды:
    \begin{equation}
      T\mathrm{_{эс}} = \vartheta\mathrm{_{эс}} + T\mathrm{_c}
    \end{equation}

\end{enumerate}
 

 

\subsection{Проектирование печатного модуля}

% выбор типа конструкции печатной платы, класса точности и шага
% координатной сетки;
% 
% выбор и обоснование метода изготовления
% электронного модуля;
% 
% расчет конструктивно-технологических параметров
% электронного модуля: определение габаритных размеров, определение
% толщины печатной платы, расчет элементов проводящего рисунка, расчет
% электрических параметров

\subsection{Расчет механической прочности \\
  и системы виброударной защиты.}

\subsection{Полный расчет надежности}

\newpage

%%% Local Variables:
%%% mode: LaTeX
%%% TeX-master: "main"
%%% End:
