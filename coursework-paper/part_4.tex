
\section{Расчет \\
  конструктивно-технологических \\
  параметров проектируемого изделия}

\subsection{Расчет объемно-компоновочных характеристик устройства.}

Под компоновкой электронной аппаратуры понимается процесс размещения
комплектующих модулей, изделий электронной техники (ИЭТ) и деталей ЭА
на плоскости или в пространстве с определением основных геометрических
форм и размеров, а также ориентировочное определение массы изделия.
На практике задача компоновки чаще всего решается путем размещения
готовых элементов с заданными формами, размером и весом на плоскости с
учетом электрических, магнитных, механических, тепловых и других видов
связи. При компоновке нужно стремиться к тому, чтобы ~\cite{Kostukevich2012}:
\begin{enumerate}
\item обеспечивалось отсутствие заметных паразитных электрических и магнитных взаимосвязей,
  влияющих на технические характеристики изделия;
  
\item взаимное расположение элементов обеспечивало технологичность сборки и монтажа,
  легкий доступ для контроля, ремонта и обслуживания;
  
\item изделие удовлетворяло требованиям технической эстетики;
\item габариты и масса изделия были минимальными.
\end{enumerate}

Существуют различные способы компоновки аппаратуры.
Необходимо выполнить аналитический расчет компоновочных параметров,
в основе которого лежит представление геометрических параметров ЭА
в виде чисел ~\cite{Kostukevich2012}.

Исходными данными для компоновочного расчета являются
~\cite{Kostukevich2012}:
\begin{itemize}
\item перечень элементов;  
\item габаритные и установочные размеры ИЭТ.
\end{itemize}

Методика расчёта заключается в следующем ~\cite{Kostukevich2012}:
\begin{enumerate}
\item Определяется суммарная площадь $S_{\text{ИЭТ}}$, занимаемая всеми ИЭТ:
  \begin{equation}
    S_{\text{ИЭТ}} = \sum^N_{i=1}S_{yi}
  \end{equation}
  где $S_{yi}$ - значение установочной площади \textit{i}-го элемента;
  \textit{n} —  количество элементов.
%
  $$ S_{\text{ИЭТ}}= 2851 \text{ мм²}$$
%
\item Рассчитывается приблизительная площадь печатной платы с учетом
  способа монтажа (односторонний, двусторонний):
  \begin{equation}
    S_{\text{Пл}} = \frac{S_{\text{ИЭТ}}}{(k_{\text{ЗПл}} \cdot m)}
  \end{equation}
  %
  где $k_{\text{ЗПл}}$ — коэффициент заполнения платы печатной, как правило,
  должен быть в пределах от 0,3 до 0,8;
  m — количество сторон монтажа (1,2).
  В данном случае количество сторон монтажа равно $m=1$.
  %
  $$S_{\text{Пл}} = 3563.75 \text{ мм²}$$
\end{enumerate}


Исходя из рассчитанной площади платы и высоты ИЭТ определить
приблизительные габаритные размеры~\cite{Kostukevich2012}.

При оценке приблизительных габаритных размеров всего устройства два
размера из трех определяют по рассчитанным размерам платы печатной с
учетом допусков на зазоры между платой и корпусом, толщины корпуса,
особенностей дизайна устройства и т. п.
Третий размер определяется с учетом максимально высоких элементов,
размещаемых на плате, и размеров, обусловленных особенностью
разрабатываемой конструкции (способ крепления платы в корпусе,
толщина корпуса, наличие дополнительных деталей на корпусе и т. п.) ~\cite{Kostukevich2012}.

Допускается выполнять предварительный расчет габаритных размеров
электронной аппаратуры по следующей методике ~\cite{Kostukevich2012}:
\begin{enumerate}
\item Определяется суммарный объем, занимаемый всеми ИЭТ и деталями:
  \begin{equation}
    V_{\textit{ИЭТ}} = \sum^N_{i=1}\vartheta_i +   \sum^M_{i=1}\vartheta_j
  \end{equation}
%
где $\vartheta_i$ - значение объёма \textit{i}-го ИЭТ;
$\vartheta_j$ — значение объёма \textit{j}-й детали;
\textit{N} — количество ИЭТ;
\textit{M} — количество деталей.
%
$$V_{\textit{ИЭТ}} =  2694 \text{мм}^3$$
\item Оценивается приблизительный объем всего устройства:
  \begin{equation}
    V_{\text{У}} = \frac{V_{\text{ИЭТ}}}{K_{\text{З}}}
  \end{equation}
%
  $$V_{\text{У}} =3367.5 \text{мм}^3$$
%
\end{enumerate}
%
Таким образом был рассчитан объём печатной платы на основе данных о
имеющихся на ней компонентах и коэффициенте заполнения.
Реальный же объём может отличаться, по той причине, что при таком
подсчёте не было учтено место отведенное под дорожки между
компонентами.

\subsection{РАСЧЁТ ТЕПЛОВОГО РЕЖИМА ДЛЯ РЭС В ГЕРМЕТИЧНОМ КОРПУСЕ}
Расчёт теплового режима для РЭС выполняется в указанном порядке ~\cite{Rotkop1976}:
\begin{enumerate}
  
\item Рассчитывается поверхность корпуса блока
%
  \begin{equation}
    S\mathrm{_{К}} = 2 \cdot (l_1 l_2 + (l_1+ l_2)l_3) % (4.46)
  \end{equation}
%
  $$ S\mathrm{_{К}} = 19790 \text{мм}^2 $$
\item Определяется условная поверхность нагретой зоны:
  \begin{equation}
    S\mathrm{_{з}} = 2 (l_1 l_2 + (l_1 + l_2) K\mathrm{_{з}} l_3 ) % (4.39)
  \end{equation}
%
    $$ S\mathrm{_{К}} = 18704 \text{мм}^2 $$
\item Определяется удельная мощность корпуса по блоку:
%
\begin{equation}
  q\mathrm{_к} = P\mathrm{_з}/S\mathrm{_к} % (4.45)
\end{equation}
%
$$q\mathrm{_к} = 606 \text{Вт/м}^2 $$
%
\item Рассчитывается удельная мощность нагретой зоны:
 % 
  \begin{equation}
      q\mathrm{_з} = P\mathrm{_з}/S\mathrm{_3} % (4.38)
    \end{equation}
$$q\mathrm{_з} = 642 \text{Вт/м}^2 $$
%
\item Находится коэффициент $\vartheta_1$ в зависимости от удельной мощности корпуса блока:
 %   
\begin{equation}
\vartheta_1 = 0,1472q\mathrm{_к} - 0,2962 \cdot 10^{-3}q\mathrm{_к}^2 + 0,3127 \cdot 10^{-6}q\mathrm{_к}^2
\end{equation}
%
$$\vartheta_1 -19.6$$
\item Находится коэффициент $\vartheta_2$ в зависимости от удельной мощности нагретой среды:
%
\begin{equation}
\vartheta_2 = 0,1390q\mathrm{_к} - 0,1223 \cdot 10^{-3}q\mathrm{_к}^2 + 0,0698 \cdot 10^{-6}q\mathrm{_к}^3
\end{equation}
%
$$\vartheta_2 = 54.85$$
%
\item Коэффициент $K\mathrm{_{Н1}}$ в зависимости от давления
  среды вне корпуса блока берётся из ГОСТ~\cite{GOST-15150-69}.
%
  $$K\mathrm{_{Н1}} = 0,999$$
%
  \item Коэффициент $K\mathrm{_{Н2}}$ в зависимости от давления
  среды внутри корпуса блока берётся из ГОСТ~\cite{GOST-15150-69}.
%
  $$K\mathrm{_{Н2}} = 0,996$$
%
\item Определяется перегрев корпуса блока:
%
  \begin{equation}
    \vartheta\mathrm{_к} = \vartheta_1 \cdot K\mathrm{_{Н1}}
  \end{equation}
%
  Поскольку перегрев не может быть отрицательным, а значение
$\vartheta_1$ получилось отрицательным, здесь и далее будем считать
перегревы определяемые через этот коэффициент отсутствующими, то есть
исходя из этих расчётов утверждается, что корпус блока не
перегревается.
%
\item Рассчитывается перегрев нагретой зоны:
%
  \begin{equation}
    \vartheta\mathrm{_з} = \vartheta\mathrm{_к} + (\vartheta_2 - \vartheta_1) \cdot K\mathrm{_{H2}}
  \end{equation}
%
    $$\vartheta\mathrm{_з} = 74K$$
%
\item Определяется средний перегрев воздуха в блоке:
%
  \begin{equation}
      \vartheta\mathrm{_в} = 0,5 \cdot (\vartheta\mathrm{_к} + \vartheta\mathrm{_з})
    \end{equation}
% 
    $$\vartheta\mathrm{_з} = 37K$$
\item Определяется удельная мощность элемента:
    \begin{equation}
      q\mathrm{_{эл}} = \frac{P\mathrm{_{эл}}}{S\mathrm{_{эл}}}
    \end{equation}
    $$q\mathrm{_{эл}} = 705882\mathrm{ВТ/м^2} $$
\end{enumerate}
%  
Таким образом были найдены средний перегрев воздуха в блоке и перегрев
нагретой зоны. Уже этих данных достаточно для того, чтобы принять
конструкторское решение и приступить к выбору радиатора для самых
тепловыделяющих компонентов, или, что гораздо более предпочтительно с
точки зрения надёжности, ввести в конструкцию корпуса перфорацию для
того, чтобы свободно проходящий воздух охлаждал элементы.
%
Однако доверять только вышеприведенном расчёту не следует,
в том числе потому, что в нём перегрев корпуса был принят за ноль.
Данное предположение нельзя без оснований принимать на веру,
по этой причине требуется дальнейшее уточнение расчёта в \textit{CAE}
программах и вычисления значений перегрева платы методом конечных элементов.

% \item Рассчитывается перегрев поверхности элемента:

%     \begin{equation}
%       \vartheta\mathrm{_{эл}} = \vartheta\mathrm{_{з}} \left(a + b \frac{q\mathrm{_{Эл}}}{q\mathrm{_{з}}}\right)
%     \end{equation}

% \item Рассчитывается перегрев окружающей элемент среды:

%     \begin{equation}
%       \vartheta\mathrm{_{эс}} = \vartheta\mathrm{_в}
%       \left(0,75 + 0,25\frac{q\mathrm{_{эл}}}{q\mathrm{_{з}}}\right)
%     \end{equation}


% \item Определяется температура корпуса блока:
%     \begin{equation}
%       T\mathrm{_к} = \vartheta\mathrm{_{к}} + T\mathrm{_с}
%     \end{equation}

% \item Определяется температура нагретой зоны:
%     \begin{equation}
%       T\mathrm{_з} = \vartheta\mathrm{_з} + T\mathrm{_c}
%     \end{equation}


% \item Находится температура поверхности элемента:
%     \begin{equation}
%       T\mathrm{_{эл}} = \vartheta\mathrm{_{эл}} + T\mathrm{_c}
%     \end{equation}



% \item Находится средняя температура воздуха в блоке:
%     \begin{equation}
%       T\mathrm{_{в}} = \vartheta\mathrm{_{в}} + T\mathrm{_c}
%     \end{equation}



% \item Находится температура окружающей элемент среды:
%     \begin{equation}
%       T\mathrm{_{эс}} = \vartheta\mathrm{_{эс}} + T\mathrm{_c}
%     \end{equation}

% \end{enumerate}
 

 

\subsection{Проектирование печатного модуля}
% Пирогова Печатные Платы.
% Парфёнов «Сборник задач и упражнений по технологии РЭА»
% выбор типа конструкции печатной платы, класса точности и шага
% координатной сетки;
По ГОСТ 23751-86 предусмотрены следующие типы конструкции ПП
~\cite{PirogovaEngineering}:
\begin{enumerate}

\item Односторонние ПП. Применяются в бытовой технике, технике связи
  и в блоках питания на ЭРЭ. Имеют низкую стоимость,
  высокую надежность, низкую плотность компоновки;

\item двусторонние ПП. Применяются в измерительной, вычислительной
  технике, технике управления и автоматического регулирования,
  технике связи, высокочастотной технике;

\item многослойные ПП. Применяются в технике управления и
  автоматического регулирования,
  вычислительной и бортовой аппаратуры для
  коммутации ИМС, БИС, СБИС, МСБ, в ЭА с высокими требованиями по быстродействию,
  плотности монтажа,  волновому сопротивлению, времени задержки сигнала и т.д.;

\item гибкие ПП, ГЖП и ГПК. Применяются в ЭА и высокой надежности
  при реализации уникальных и сложных технических решений,
  конструкция которых исключает применение жестких ПП.
\end{enumerate}

Чем больше слоев, тем легче трассировать печатную плату.
Однако большее число слоёв затрудняет производство печатной платы
в условиях лаборатории, в которых может использоваться разрабатываемое
устройств.


В качестве типа конструкции была выбрана двухсторонняя печатная плата.
Печатные платы данной конструкции широко применяются в измерительной,
вычислительной технике, технике управления и автоматического регулирования,
технике связи, высокочастотной технике;

 Согласно ГОСТ Р 53429-2009 существует семь классов точности печатных
плат. При этом класс точности печатной платы определяется требованиями
к конструктивным элементам, т.е. если в конструкции присутствует хотя
бы один элемент, требующий точности 7 класса, при этом к остальным
элементам требования не выше 4, то мы имеем плату 7 класса точности
~\cite{rezonit-class}.

В качестве компромисса между требуемой точности и условиями в которых
может изготавливаться плата был выбран третий класс точности печатной
платы.


% 
% выбор и обоснование метода изготовления
% электронного модуля;
%
Был выбран механический метод изготовления двухсторонней ПП с
использованием фрезерования.

При этом контролируют глубину врезания фрезы
в заготовку и равномерность прижима заготовки к рабочему
столу.

Металлизацию переходных отверстий
осуществляют пустотелыми
заклепками.

Данный способ выбран из-за высокой оперативности простоты реализации.
% расчет конструктивно-технологических параметров
% электронного модуля: определение габаритных размеров, определение
% толщины печатной платы, расчет элементов проводящего рисунка, расчет
% электрических параметров

\subsection{Расчет механической прочности \\
  и системы виброударной защиты.}
% Каленкович мех. воздействия

Под механической прочностью понимают способность констуркции РЭС и его
элементов противостоять механическим воздействиями без разрушения.

При расчете конструкции и ее элементов на механическую прочность
определяют уровни ее запасов.  Обычно отсуствие достаточной
механической прочсности вдеёт к появлению обрывов, трещин сколов,
срывов элементов со своих посадочных мест, к поломкам корпусов ЭРЭ,
смятию, значительным уровням остаточных деформаций, что ведет к
электрическому отказу конструкции РЭС.
При этом под отказом РЭС здесь понимается также и уход его
электрических параметров за допустимые пределы. Вместе с тем
электрический отказ РЭС может наступать в результате механических
воздействий за счет значительных уровней упругой деформации некоторых
констурктивных элементов даже при их достаточной механической
прочности. При таких деформациях могут происходить короткие замыканиях
между элементмаи, приводящие к электрическому пробою или к ухудшению
электрических параметров РЭС. Все это в соответвии с теорией
сопротивления материалов будет свидетелсьтвовать о недостаточной
жесткости элементов конструкции~\cite{Kalenkovich1989}.

Аналогичные явления в конструкциях будут наблюдаться при механических
воздействиях, когда такие конструктивы, как платы, будут испытыать
механическую нагрузку вдоль их продольной оси.
При достижении данной нагрузкой некоторого критического значения может
происходить продольный изгиб платы~\cite{Kalenkovich1989}.

Целью расчета является определение действующих на элементы изделия
перегрузок при наличии вибрации, а также максимальных перемещений.
При необходимости производится выбор и расчет системы
аммортизации~\cite{Kalenkovich2012}.

Исходные данные для расчета~\cite{Kalenkovich2012}:
\begin{itemize}
\item a — длина,
  
\item b — ширина,
  
\item h — толщина печатной платы,
  
\item М - масса печатной платы с ЭРЭ.
\end{itemize}



Определяем частоту собственных колебаний печатной платы, как
частоту собственных колебаний, равномерно нагруженной пластины по
формуле ~\cite{Kalenkovich2012}:
\begin{equation}
  f_0 = \frac{1}{2\pi}\frac{K_{\text{б}}}{a^2}\sqrt{\frac{D}{M}ab}
\end{equation}
где
\begin{itemize}
\item $a$ и $b$ — длина и ширина пластины;

\item $D$ — цилиндрическая жесткость;
\item $M$ — масса пластины с элементами.
\item $K_{\alpha}$ — коэффициент, зависящий от способа закрепления сторон пластины.
\end{itemize}


\subsection{Расчет параметров лицевой панели. \\
  Анализ и учет требований эргономики и технической эстетики. }
Панель управления (ПУ) является неотъемлемой частью любого
радиоэлектронного средства (РЭС),
а также технического средства, управление
которым осуществляется с помощью радиоэлектроники
(СВЧ-печь, стиральная машина,
копировальный аппарат, автомобиль, самолет и т.п.) ~\cite{Alipherenko2007}.

Панель управления, являясь средством коммуникативной связи,
представляет собой несущую конструкцию, на которой расположены органы индикации,
управления, коммутации, надписи и другие компоненты, предназначенные для
выполнения соответствующих им функций и несущие оператору необходимую
информацию~\cite{Alipherenko2007}.

В данном устройстве ПУ состоит из одного \textit{OLED} дисплея и четырех
потенциометров. Расстояние от оператора, до ПУ будет принято за 1 метр.

Определим максимальную длину высоту и площадь ПУ по следующим
формулам~\cite{Alipherenko2007}:
\begin{equation}
  L_{\text{П.У.max}} = 2l \cdot tg\frac{a_{\text{г}}}{2}
\end{equation}

\begin{equation}
  H_{\text{П.У.max}} = 2l \cdot tg\frac{a_{\text{в}}}{2}
\end{equation}

\begin{equation}
  S_{\text{П.У.max}} =   L_{\text{П.У.max}} \cdot   H_{\text{П.У.max}}
\end{equation}

Таковы результаты расчетов максимальных размеров панели управления:

$$L_{\text{П.У.max}} \sim 0.5 \text{м}^2$$

$$H_{\text{П.У.max}} \sim 0.4 \text{м}^2$$

$$S_{\text{П.У.max}} \sim 0.2 \text{м}^2$$

Реальный размер панели в несколько раз меньше.

Минимально допустимые размеры ПУ определяются исходя из объема
оперативной памяти и оперативного (центрального) поля зрения
оператора. В соответствии с требованиями инженерной психологии в поле
зрения оператора, ограниченным углом оперативного поля зрения $\alpha_{\text{ПЗ}}$,
должно попадать $6\pm 2$ компонента ПУ ~\cite{Alipherenko2007}.


Тогда площадь оперативного поля зрения может быть определена как
\begin{equation}
  S_{\text{П.З}} = h \cdot h = \left( 2l \cdot tg\frac{\alpha_{\text{П.З.}}}{2} \right)
\end{equation}

$$S \sim 0.063\text{м}^2$$

Следовательно:
\begin{equation}
  S_{\text{П.У.MIN}} = \frac{N}{6 \pm 2} S_{\text{ПЗ}}
\end{equation}

На данной панели управления представлено всего 5 компонентов:
4 потенциометра и один дисплей.

$$S \sim 0.04 \text{м}^2$$
Фактическая площадь ПУ SПУф должна лежать в пределах~\cite{Alipherenko2007}:
\begin{equation}
  S_{\text{П.УMin}} \leq S_{\text{П.У.Ф}} \leq S_{\text{П.УMAx}}
\end{equation}

Получается по результатам расчета
$$  0.04 \text{м}^2 \leq S_{\text{П.У.Ф}} \leq 0.2 \text{м}^2$$

Таким образом была рассчитаны пределы, в которые должна укладываться
панель управления при проектировании данной РЭС.
Эти пределы удалось соблюсти.

\subsection{Уточнённый расчет надежности}

Под надёжностью понимают свойство изделия сохранять в течение
заданного времени в пределах установленных норм значения
функциональных параметров при определённых условиях (заданные режимы и
условия эксплуатации, техническокого обслуживания, хранения и
транспортирования)~\cite{Borovikov2010}.

В теории и практике надёжности технических изделий широко используют
понятие наработка, под которой понимают продолжительность работы
изделия, выраженную в часах, циклах переключения или других единиц в
зависимости от вида и функционального назначения изделия ~\cite{Borovikov2010}.

Под отказом понимают полную или частичную потерю изделием
работоспособности вследствие ухода одного или нескольких
функциональных параметров за пределы установленных норм, указанных в
технической документации~\cite{Borovikov2010}.

Проведём уточнённый расчёт показателей безотказности функционального модуля:
\begin{enumerate}
\item Находим коэффициент электрической нагрузки элементов, пользуюясь
картами электрических режимо и эксплуатационными электрическими
характеристиками используемыми в модуле.

Коэффициент электрической нагрузки элементов равен
\begin{equation}
  K_{\text{Н}} = \frac{F_{\text{раб}}}{F_{\text{ном}}}
\end{equation}

При этом исходя из того, что в данной работе расчёт производится на
раннем этапе проектирования, выходит возможным, на
основании данных полученных в результате САПР для симуляции
электронных схем, узнать показатели $F_{\text{раб}}$ и
на основании полученных данных подбирать компоненты из широкого
ассортимента представленных на рынке с таким показателем
$F_{\text{ном}}$ чтобы искуственно изменять коэффицент
$  K_{\text{Н}}$ в лучшую сторону.

Симуляция схемы не была произведена, по той причине, что на момент
написания работы не была готова прошивка для микроконтроллера в схеме,
однако, приняв во внимание,  вышеописанную возможность примем
$K_{\text{Н}}$ у элементов равным $0,8$.
\item Определим максимальную температуру элементов модуля при его
  работу в составе РЭУ.
  Для учета влияния температуры на эксплуатационную интенсивность
  отказов элементов  $\lambda_{\text{Э}}$ принято во внимание
  верхнее значение предельной рабочей температы
  ($t_{\text{раб.max}}= 40°C$), cоответствующей РЭУ
  исполнения УХЛ4.2 по ГОСТ 15150-69, и возможное увеличение предельной рабочей температуры на значение
  $\Delta t_C = 10°C$ за счёт нагрева РЭУ и, следовательно модуля в составе РЭУ.
  Предельная рабоачая температура $t_{эл.max}$ теплонагруженных
  элементов (ИМС, транзисторы, диоды, мощные резисторы) определена как \cite{Borovikov2010}:
  \begin{equation}
    t_{\text{эл.max}} = (t_{\text{раб.max}} + \Delta t_C) + \Delta t_{\text{з}} = (40 + 10) +15 = 65°С
  \end{equation}
  где $\Delta t_{\text{з}}$ — перегрев нагретой зоне конструкции РЭУ
  Нагретая зона — это гиптотетический объём, в котором условно
  рассеивается вся тепловая энергия, выделяемая РЭУ.

  Значение величины $t_{эл.max}$  для нетеплонагруженных элементов
  (конденсаторы, слабонагруженные резисторы, соединительс, кварцевый резонатор)
  подсчитано как ~\cite{Borovikov2010}:
  \begin{equation}
    t_{\text{эл.max}} = (t_{\text{раб.max}} + \Delta t_C) + \Delta t_{\text{з}} = (40 + 10) +10 = 60°С
  \end{equation}
%
\item Пользуясь таблице 5.3 из источника~\cite{Borovikov2010} находим
  справочные значения интенсивности отказов элементов модуля.

\item По таблице 5.1 из ~\cite{Borovikov2010} выбираем математические
модели расчёта эксплуатационной интеcивности отказов элементов
$\lambda_{\text{Э}}$
% ---

\item Определяем значения поправочных коэффициентов, входящих в
выбранные модели расчёта эксплутационной интесивности отказов
элементов $\lambda_{\text{Э}}$
\item Для каждого элемента находим произведение поправочных
коэффициентов, и значение эксплуатационной интесивности отказов
$\lambda_{\text{Э}}$.

\begin{sidewaystable}
  \centering
  \small
  \caption{Расчёт эксплуатационной безоткзоности элементов модуля}
  \begin{tabular}{|l |l |l |l |l |l |l |l |l |l |l |l |l |l |l |l |l |l |}
    \hline
Designator  &   λБ  & λe                 & Кис & Кр & Кt & Ккорп & Кv & Кф & Кд & Кu & Кс & Kr & Км & Кк & Кэ & Кп & λe   \\ \hline
R1...R4     & 0,044 & λб Kр Kr Км Kэ Kп  & 1 & 0,34 & 1 & 1 & 1 & 1 & 1 & 1 & 1 & 0,7 & 0,7 & 1 & 1 & 1 & 0,1615 \\ \hline
R5...R8     & 0,183 & λб Kр Kr Км Kэ Kп  & 1 & 0,99 & 1 & 1 & 1 & 1 & 1 & 1 & 1 & 0,9 & 0,7 & 1 & 1 & 1 & 0,6209 \\ \hline
R9...R10    & 0,044 & λб Kр Kr Км Kэ Kп  & 1 & 0,34 & 1 & 1 & 1 & 1 & 1 & 1 & 1 & 0,7 & 0,7 & 1 & 1 & 1 & 0,1615 \\ \hline
R13         & 0,044 & λб Kр Kr Км Kэ Kп  & 1 & 0,34 & 1 & 1 & 1 & 1 & 1 & 1 & 1 & 1 & 0,7 & 1 & 1 & 1 & 0,2395  \\ \hline
R14         & 0,044 & λб Kр Kr Км Kэ Kп  & 1 & 0,34 & 1 & 1 & 1 & 1 & 1 & 1 & 1 & 0,7 & 0,7 & 1 & 1 & 1 & 0,16 \\ \hline
R15         & 0,044 & λб Kр Kr Км Kэ Kп  & 1 & 0,34 & 1 & 1 & 1 & 1 & 1 & 1 & 1 & 1 & 0,7 & 1 & 1 & 1 & 0,24  \\ \hline
C1          & 0,173 & λб Кр Кс Кэ Кп  & 1 & 0,7 & 1 & 1 & 1 & 1 & 1 & 1 & 0,5768 & 1 & 1 & 1 & 1 & 1 & 0,41 \\ \hline
C2          & 0,173 & λб Кр Кс Кэ Кп  & 1 & 0,7 & 1 & 1 & 1 & 1 & 1 & 1 & 0,3344 & 1 & 1 & 1 & 1 & 1 & 0,24 \\ \hline
C3          & 0,173 & λб Кр Кс Кэ Кп  & 1 & 0,7 & 1 & 1 & 1 & 1 & 1 & 1 & 0,2 & 1 & 1 & 1 & 1 & 1 & 0,14 \\ \hline
C4, C5      & 0,022 & λб Кр Кс Кэ Кп  & 1 & 0,7 & 1 & 1 & 1 & 1 & 1 & 1 & 0,1412 & 1 & 1 & 1 & 1 & 1 & 0,099 \\ \hline
DA1         & 0,028 & λб Кt Кис Ккорп Кv Кэ Кп & 0,96 & 1 & 1,3703 & 1 & 1 & 1 & 1 & 1 & 1 & 1 & 1 & 1 & 1 & 1 & 1,320 \\ \hline
DD1         & 0,023 & λб Кt Кис Ккорп Кv Кэ Кп & 0,75 & 1 & 1,412 & 1 & 1 & 1 & 1 & 1 & 1 & 1 & 1 & 1 & 1 & 1 & 1,054 \\ \hline
HA1         & 0,034 & λбКэКп & 1 & 1 & 1 & 1 & 1 & 1 & 1 & 1 & 1 & 1 & 1 & 1 & 1 & 1 & 1 \\ \hline
HL1         & 0,034 & λбКрКэКп & 1 & 0,04985 & 1 & 1 & 1 & 1 & 1 & 1 & 1 & 1 & 1 & 1 & 1 & 8 & 0,3988 \\ \hline
SA1         & 0,058 & Λб Кк Кf КрКэКп & 1 & 1,96 & 1 & 1 & 1 & 1 & 1 & 1 & 1 & 1 & 1 & 1 & 1 & 1 & 1,960 \\ \hline
SA2         & 0,058 & Λб Кк Кf КрКэКп & 1 & 1,96 & 1 & 1 & 1 & 1 & 1 & 1 & 1 & 1 & 1 & 1 & 1 & 1 & 1,960 \\ \hline
VD1         & 0,091 & λб Кр Кф Кд Кu Кэ Кп    & 1 & 0,05 & 1 & 1 & 1 & 1,5 & 0,6 & 0,8183 & 1 & 1 & 1 & 1 & 1 & 1 & 0,037 \\ \hline
VD2         & 0,091 & λб Кр Кф Кд Кu Кэ Кп    & 1 & 0,05 & 1 & 1 & 1 & 1,5 & 0,6 & 0,8183 & 1 & 1 & 1 & 1 & 1 & 1 & 0,037 \\ \hline
VD3         & 0,091 & λб Кр Кф Кд Кu Кэ Кп    & 1 & 0,05 & 1 & 1 & 1 & 1,5 & 0,6 & 0,8183 & 1 & 1 & 1 & 1 & 1 & 1 & 0,037 \\ \hline
VT1         & 0,044 & λб Кр Кф Кд Кu Кэ Кп    & 1 & 0,18 & 1 & 1 & 1 & 0,7 & 0,5 & 2,2831 & 1 & 1 & 1 & 1 & 1 & 1 & 0,14 \\ \hline
X1,X5       & 0,0104 & λб Кр Кк Кn Кэ Кп  & 1 & 0,5 & 1 & 1 & 1 & 1 & 1 & 1 & 1 & 1 & 1 & 2,86 & 1 & 8 & 11,49 \\ \hline
X2          & 0,0104 & λб Кр Кк Кn Кэ Кп  & 1 & 0,5 & 1 & 1 & 1 & 1 & 1 & 1 & 1 & 1 & 1 & 1,36 & 1 & 8 & 5,47 \\ \hline
X3          & 0,0104 & λб Кр Кк Кn Кэ Кп  & 1 & 0,5 & 1 & 1 & 1 & 1 & 1 & 1 & 1 & 1 & 1 & 2,02 & 1 & 8 & 8,11 \\ \hline
X4          & 0,0104 & λб Кр Кк Кn Кэ Кп  & 1 & 0,5 & 1 & 1 & 1 & 1 & 1 & 1 & 1 & 1 & 1 & 2,02 & 1 & 1 & 1,02 \\ \hline
ZQ1         & 0,026 & λб Кt Кэ Кп         & 1 & 1 & 1,29 & 1 & 1 & 1 & 1 & 1 & 1 & 1 & 1 & 1 & 1 & 1 & 1,29 \\ \hline
    
  \end{tabular}
\end{sidewaystable}

\item Подсчитываем эксплуатационную интенсивность отказов модуля.
  Для этого проссумируем значения, приведенные в последнем столбце.
  $\lambda_{\text{М}} = 36,53 \cdot 10^{-6} \text{1/ч}$
  
\item Находим наработку на отказ:
  $T_0 = 1 / \Lambda_{\text{М}} = 27,377 \text{часов}$
\end{enumerate}


\newpage

%%% Local Variables:
%%% mode: LaTeX
%%% TeX-master: "main"
%%% End:
