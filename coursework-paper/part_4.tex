
\section{Расчет \\
  конструктивно-технологических \\
  параметров проектируемого изделия}

\subsection{Расчет объемно-компоновочных характеристик устройства.}

Под компоновкой электронной аппаратуры понимается процесс размещения
комплектующих модулей, изделий электронной техники (ИЭТ) и деталей ЭА
на плоскости или в пространстве с определением основных геометрических
форм и размеров, а также ориентировочное определение массы изделия.
На практике задача компоновки чаще всего решается путем размещения
готовых элементов с заданными формами, размером и весом на плоскости с
учетом электрических, магнитных, механических, тепловых и других видов
связи. При компоновке нужно стремиться к тому, чтобы ~\cite{Kostukevich2012}:
\begin{enumerate}
\item обеспечивалось отсутствие заметных паразитных электрических и магнитных взаимосвязей,
  влияющих на технические характеристики изделия;
  
\item взаимное расположение элементов обеспечивало технологичность сборки и монтажа,
  легкий доступ для контроля, ремонта и обслуживания;
  
\item изделие удовлетворяло требованиям технической эстетики;
\item габариты и масса изделия были минимальными.
\end{enumerate}

Существуют различные способы компоновки аппаратуры.
Необходимо выполнить аналитический расчет компоновочных параметров,
в основе которого лежит представление геометрических параметров ЭА
в виде чисел ~\cite{Kostukevich2012}.

Исходными данными для компоновочного расчета являются
~\cite{Kostukevich2012}:
\begin{itemize}
\item перечень элементов;  
\item габаритные и установочные размеры ИЭТ.
\end{itemize}

Методика расчёта заключается в следующем ~\cite{Kostukevich2012}:
\begin{enumerate}
\item Определяется суммарная площадь $S_{\textrm{ИЭТ}}$, занимаемая всеми ИЭТ:
  \begin{equation}
    S_{\textrm{ИЭТ}} = \sum^N_{i=1}S_{yi}
  \end{equation}

  где $S_{yi}$ - значение установочной площади \textit{i}-го элемента;
  \textit{n} —  количество элементов.

  $$ S_{\textrm{ИЭТ}}= 2851 \textrm{ мм²}$$

\item Рассчитывается приблизительная площадь печатной платы с учетом
  способа монтажа (односторонний, двусторонний):
  \begin{equation}
    S_{\textrm{Пл}} = \frac{S_{\textrm{ИЭТ}}}{(k_{\textrm{ЗПл}} \cdot m)}
  \end{equation}
  где $k_{\textrm{ЗПл}}$ — коэффициент заполнения платы печатной, как правило,
  должен быть в пределах от 0,3 до 0,8;
  m — количество сторон монтажа (1,2).
  В данном случае количество сторон монтажа равно $m=1$.
  $$S_{\textrm{Пл}} = 3563.75 \textrm{ мм²}$$
\end{enumerate}


Исходя из рассчитанной площади платы и высоты ИЭТ определить
приблизительные габаритные размеры~\cite{Kostukevich2012}.

При оценке приблизительных габаритных размеров всего устройства два
размера из трех определяют по рассчитанным размерам платы печатной с
учетом допусков на зазоры между платой и корпусом, толщины корпуса,
особенностей дизайна устройства и т. п.
Третий размер определяется с учетом максимально высоких элементов,
размещаемых на плате, и размеров, обусловленных особенностью
разрабатываемой конструкции (способ крепления платы в корпусе,
толщина корпуса, наличие дополнительных деталей на корпусе и т. п.) ~\cite{Kostukevich2012}.

Допускается выполнять предварительный расчет габаритных размеров
электронной аппаратуры по следующей методике ~\cite{Kostukevich2012}:
\begin{enumerate}
\item Определяется суммарный объем, занимаемый всеми ИЭТ и деталями:
  \begin{equation}
    V_{\textit{ИЭТ}} = \sum^N_{i=1}\vartheta_i +   \sum^M_{i=1}\vartheta_j
  \end{equation}

где $\vartheta_i$ - значение объёма \textit{i}-го ИЭТ;
$\vartheta_j$ — значение объёма \textit{j}-й детали;
\textit{N} — количество ИЭТ;
\textit{M} — количество деталей.

$$V_{\textit{ИЭТ}} =  2694 \textrm{мм}^3$$
\item Оценивается приблизительный объем всего устройства:
  \begin{equation}
    V_{\textrm{У}} = \frac{V_{\textrm{ИЭТ}}}{K_{\textrm{З}}}
  \end{equation}

  $$V_{\textrm{У}} =3367.5 \textrm{мм}^3$$

\end{enumerate}

Таким образом был рассчитан объём печатной платы на основе данных о
имеющихся на ней компонентах и коэффициенте заполнения.
Реальный же объём может отличаться, по той причине, что при таком
подсчёте не было учтено место отведенное под дорожки между
компонентами.

\subsection{Расчёт теплового режима.} 
  % ДЛЯ РЭС В ГЕРМЕТИЧНОМ КОРПУСЕ}
Расчёт теплового режима для РЭС выполняется в указанном порядке ~\cite{Rotkop1976}:
\begin{enumerate}
  
\item Рассчитывается поверхность корпуса блока

  \begin{equation}
    S\mathrm{_{К}} = 2 \cdot (l_1 l_2 + (l_1+ l_2)l_3) % (4.46)
  \end{equation}

  $$ S\mathrm{_{К}} = 19790 \textrm{мм}^2 $$
\item Определяется условная поверхность нагретой зоны:
  \begin{equation}
    S\mathrm{_{з}} = 2 (l_1 l_2 + (l_1 + l_2) K\mathrm{_{з}} l_3 ) % (4.39)
  \end{equation}

    $$ S\mathrm{_{К}} = 18704 \textrm{мм}^2 $$
\item Определяется удельная мощность корпуса по блоку:

\begin{equation}
  q\mathrm{_к} = P\mathrm{_з}/S\mathrm{_к} % (4.45)
\end{equation}

$$q\mathrm{_к} = 606 \textrm{Вт/м}^2 $$

\item Рассчитывается удельная мощность нагретой зоны:
  
  \begin{equation}
      q\mathrm{_з} = P\mathrm{_з}/S\mathrm{_3} % (4.38)
    \end{equation}
$$q\mathrm{_з} = 642 \textrm{Вт/м}^2 $$

\item Находится коэффициент $\vartheta_1$ в зависимости от удельной мощности корпуса блока:
    
\begin{equation}
\vartheta_1 = 0,1472q\mathrm{_к} - 0,2962 \cdot 10^{-3}q\mathrm{_к}^2 + 0,3127 \cdot 10^{-6}q\mathrm{_к}^2
\end{equation}

$$\vartheta_1 -19.6$$
\item Находится коэффициент $\vartheta_2$ в зависимости от удельной мощности нагретой среды:

\begin{equation}
\vartheta_2 = 0,1390q\mathrm{_к} - 0,1223 \cdot 10^{-3}q\mathrm{_к}^2 + 0,0698 \cdot 10^{-6}q\mathrm{_к}^3
\end{equation}

$$\vartheta_2 = 54.85$$

\item Коэффициент $K\mathrm{_{Н1}}$ в зависимости от давления
  среды вне корпуса блока берётся из ГОСТ~\cite{GOST-15150-69}.

  $$K\mathrm{_{Н1}} = 0,999$$

  \item Коэффициент $K\mathrm{_{Н2}}$ в зависимости от давления
  среды внутри корпуса блока берётся из ГОСТ~\cite{GOST-15150-69}.

  $$K\mathrm{_{Н2}} = 0,996$$

\item Определяется перегрев корпуса блока:

  \begin{equation}
    \vartheta\mathrm{_к} = \vartheta_1 \cdot K\mathrm{_{Н1}}
  \end{equation}

  Поскольку перегрев не может быть отрицательным, а значение
$\vartheta_1$ получилось отрицательным, здесь и далее будем считать
перегревы определяемые через этот коэффициент отсутствующими, то есть
исходя из этих расчётов утверждается, что корпус блока не
перегревается.

\item Рассчитывается перегрев нагретой зоны:

  \begin{equation}
    \vartheta\mathrm{_з} = \vartheta\mathrm{_к} + (\vartheta_2 - \vartheta_1) \cdot K\mathrm{_{H2}}
  \end{equation}

    $$\vartheta\mathrm{_з} = 74K$$

\item Определяется средний перегрев воздуха в блоке:

  \begin{equation}
      \vartheta\mathrm{_в} = 0,5 \cdot (\vartheta\mathrm{_к} + \vartheta\mathrm{_з})
    \end{equation}
    
    $$\vartheta\mathrm{_з} = 37K$$
\item Определяется удельная мощность элемента:
    \begin{equation}
      q\mathrm{_{эл}} = \frac{P\mathrm{_{эл}}}{S\mathrm{_{эл}}}
    \end{equation}
    $$q\mathrm{_{эл}} = 705882\mathrm{ВТ/м^2} $$
\end{enumerate}
  
Таким образом были найдены средний перегрев воздуха в блоке и перегрев
нагретой зоны. Уже этих данных достаточно для того, чтобы принять
конструкторское решение и приступить к выбору радиатора для самых
тепловыделяющих компонентов, или, что гораздо более предпочтительно с
точки зрения надёжности, ввести в конструкцию корпуса перфорацию для
того, чтобы свободно проходящий воздух охлаждал элементы.

Однако доверять только вышеприведенном расчёту не следует,
в том числе потому, что в нём перегрев корпуса был принят за ноль.
Данное предположение нельзя без оснований принимать на веру,
по этой причине требуется дальнейшее уточнение расчёта в \textit{CAE}
программах и вычисления значений перегрева платы методом конечных элементов.

% \item Рассчитывается перегрев поверхности элемента:

%     \begin{equation}
%       \vartheta\mathrm{_{эл}} = \vartheta\mathrm{_{з}} \left(a + b \frac{q\mathrm{_{Эл}}}{q\mathrm{_{з}}}\right)
%     \end{equation}

% \item Рассчитывается перегрев окружающей элемент среды:

%     \begin{equation}
%       \vartheta\mathrm{_{эс}} = \vartheta\mathrm{_в}
%       \left(0,75 + 0,25\frac{q\mathrm{_{эл}}}{q\mathrm{_{з}}}\right)
%     \end{equation}


% \item Определяется температура корпуса блока:
%     \begin{equation}
%       T\mathrm{_к} = \vartheta\mathrm{_{к}} + T\mathrm{_с}
%     \end{equation}

% \item Определяется температура нагретой зоны:
%     \begin{equation}
%       T\mathrm{_з} = \vartheta\mathrm{_з} + T\mathrm{_c}
%     \end{equation}


% \item Находится температура поверхности элемента:
%     \begin{equation}
%       T\mathrm{_{эл}} = \vartheta\mathrm{_{эл}} + T\mathrm{_c}
%     \end{equation}



% \item Находится средняя температура воздуха в блоке:
%     \begin{equation}
%       T\mathrm{_{в}} = \vartheta\mathrm{_{в}} + T\mathrm{_c}
%     \end{equation}



% \item Находится температура окружающей элемент среды:
%     \begin{equation}
%       T\mathrm{_{эс}} = \vartheta\mathrm{_{эс}} + T\mathrm{_c}
%     \end{equation}

% \end{enumerate}
 

 

\subsection{Проектирование печатного модуля}
% Пирогова Печатные Платы.
% Парфёнов «Сборник задач и упражнений по технологии РЭА»
% выбор типа конструкции печатной платы, класса точности и шага
% координатной сетки;
По ГОСТ 23751-86 предусмотрены следующие типы конструкции ПП
~\cite{PirogovaEngineering}:
\begin{enumerate}

\item Односторонние ПП. Применяются в бытовой технике, технике связи
  и в блоках питания на ЭРЭ. Имеют низкую стоимость,
  высокую надежность, низкую плотность компоновки;

\item двусторонние ПП. Применяются в измерительной, вычислительной
  технике, технике управления и автоматического регулирования,
  технике связи, высокочастотной технике;

\item многослойные ПП. Применяются в технике управления и
  автоматического регулирования,
  вычислительной и бортовой аппаратуры для
  коммутации ИМС, БИС, СБИС, МСБ, в ЭА с высокими требованиями по быстродействию,
  плотности монтажа,  волновому сопротивлению, времени задержки сигнала и т.д.;

\item гибкие ПП, ГЖП и ГПК. Применяются в ЭА и высокой надежности
  при реализации уникальных и сложных технических решений,
  конструкция которых исключает применение жестких ПП.
\end{enumerate}

Чем больше слоев, тем легче трассировать печатную плату.
Однако большее число слоёв затрудняет производство печатной платы
в условиях лаборатории, в которых может использоваться разрабатываемое
устройств.


В качестве типа конструкции была выбрана двухсторонняя печатная плата.
Печатные платы данной конструкции широко применяются в измерительной,
вычислительной технике, технике управления и автоматического регулирования,
технике связи, высокочастотной технике;

 Согласно ГОСТ Р 53429-2009 существует семь классов точности печатных
плат. При этом класс точности печатной платы определяется требованиями
к конструктивным элементам, т.е. если в конструкции присутствует хотя
бы один элемент, требующий точности 7 класса, при этом к остальным
элементам требования не выше 4, то мы имеем плату 7 класса точности
~\cite{rezonit-class}.

В качестве компромисса между требуемой точности и условиями в которых
может изготавливаться плата был выбран третий класс точности печатной
платы.


% 
% выбор и обоснование метода изготовления
% электронного модуля;
%
Был выбран механический метод изготовления двухсторонней ПП с
использованием фрезерования.

При этом контролируют глубину врезания фрезы
в заготовку и равномерность прижима заготовки к рабочему
столу.

Металлизацию переходных отверстий
осуществляют пустотелыми
заклепками.

Данный способ выбран из-за высокой оперативности простоты реализации.
% расчет конструктивно-технологических параметров
% электронного модуля: определение габаритных размеров, определение
% толщины печатной платы, расчет элементов проводящего рисунка, расчет
% электрических параметров

\subsection{Расчет механической прочности \\
  и системы виброударной защиты.}
%  Каленкович мех. воздействия

\subsection{Расчет параметров лицевой панели. \\
  Анализ и учет требований эргономики и технической эстетики. }
Панель управления (ПУ) является неотъемлемой частью любого
радиоэлектронного средства (РЭС),
а также технического средства, управление
которым осуществляется с помощью радиоэлектроники
(СВЧ-печь, стиральная машина,
копировальный аппарат, автомобиль, самолет и т.п.) ~\cite{Alipherenko2007}.

Панель управления, являясь средством коммуникативной связи,
представляет собой несущую конструкцию, на которой расположены органы индикации,
управления, коммутации, надписи и другие компоненты, предназначенные для
выполнения соответствующих им функций и несущие оператору необходимую
информацию~\cite{Alipherenko2007}.

В данном устройстве ПУ состоит из одного \textit{OLED} дисплея и четырех
потенциометров. Расстояние от оператора, до ПУ будет принято за 1 метр.

Определим максимальную длину высоту и площадь ПУ по следующим
формулам~\cite{Alipherenko2007}:
\begin{equation}
  L_{\textrm{П.У.max}} = 2l \cdot tg\frac{a_{\textrm{г}}}{2}
\end{equation}

\begin{equation}
  H_{\textrm{П.У.max}} = 2l \cdot tg\frac{a_{\textrm{в}}}{2}
\end{equation}

\begin{equation}
  S_{\textrm{П.У.max}} =   L_{\textrm{П.У.max}} \cdot   H_{\textrm{П.У.max}}
\end{equation}

Таковы результаты расчетов максимальных размеров панели управления:

$$L_{\textrm{П.У.max}} \sim 0.5 \textrm{м}^2$$

$$H_{\textrm{П.У.max}} \sim 0.4 \textrm{м}^2$$

$$S_{\textrm{П.У.max}} \sim 0.2 \textrm{м}^2$$

Реальный размер панели в несколько раз меньше.

Минимально допустимые размеры ПУ определяются исходя из объема
оперативной памяти и оперативного (центрального) поля зрения
оператора. В соответствии с требованиями инженерной психологии в поле
зрения оператора, ограниченным углом оперативного поля зрения $\alpha_{\textrm{ПЗ}}$,
должно попадать $6\pm 2$ компонента ПУ ~\cite{Alipherenko2007}.


Тогда площадь оперативного поля зрения может быть определена как
\begin{equation}
  S_{\textrm{П.З}} = h \cdot h = \left( 2l \cdot tg\frac{\alpha_{\textrm{П.З.}}}{2} \right)
\end{equation}

$$S \sim 0.063\textrm{м}^2$$

Следовательно:
\begin{equation}
  S_{\textrm{П.У.MIN}} = \frac{N}{6 \pm 2} S_{\textrm{ПЗ}}
\end{equation}

На данной панели управления представлено всего 5 компонентов:
4 потенциометра и один дисплей.

$$S \sim 0.04 \textrm{м}^2$$
Фактическая площадь ПУ SПУф должна лежать в пределах~\cite{Alipherenko2007}:
\begin{equation}
  S_{\textrm{П.УMin}} \leq S_{\textrm{П.У.Ф}} \leq S_{\textrm{П.УMAx}}
\end{equation}

Получается по результатам расчета
$$  0.04 \textrm{м}^2 \leq S_{\textrm{П.У.Ф}} \leq 0.2 \textrm{м}^2$$

Таким образом была рассчитаны пределы, в которые должна укладываться
панель управления при проектировании данной РЭС.
Эти пределы удалось соблюсти.

\subsection{Полный расчет надежности}

Под надёжностью понимают свойство изделия сохранять в течение
заданного времени в пределах установленных норм значения
функциональных параметров при определённых условиях (заданные режимы и
условия эксплуатации, техническокого обслуживания, хранения и
транспортирования)~\cite{Borovikov2010}.

В теории и практике надёжности технических изделий широко используют
понятие наработка, под которой понимают продолжительность работы
изделия, выраженную в часах, циклах переключения или других единиц в
зависимости от вида и функционального назначения изделия ~\cite{Borovikov2010}.

Под отказом понимают полную или частичную потерю изделием
работоспособности вследствие ухода одного или нескольких
функциональных параметров за пределы установленных норм, указанных в
технической документации~\cite{Borovikov2010}.

Проведём уточнённый расчёт показателей безотказности функционального модуля:
\begin{enumerate}
\item Находим коэффициент электрической нагрузки элементов, пользуюясь
картами электрических режимо и эксплуатационными электрическими
характеристиками используемыми в модуле.

Коэффициент электрической нагрузки элементов равен
\begin{equation}
  K_{\textrm{Н}} = \frac{F_{\textrm{раб}}}{F_{\textrm{ном}}}
\end{equation}

При этом исходя из того, что в данной работе расчёт производится на
раннем этапе проектирования, выходит возможным, на
основании данных полученных в результате САПР для симуляции
электронных схем, узнать показатели $F_{\textrm{раб}}$ и
на основании полученных данных подбирать компоненты из широкого
ассортимента представленных на рынке с таким показателем
$F_{\textrm{ном}}$ чтобы искуственно изменять коэффицент
$  K_{\textrm{Н}}$ в лучшую сторону.

Симуляция схемы не была произведена, по той причине, что на момент
написания работы не была готова прошивка для микроконтроллера в схеме,
однако, приняв во внимание,  вышеописанную возможность примем
$K_{\textrm{Н}}$ у элементов равным $0,8$.
\item Определим максимальную температуру элементов модуля при его
  работу в составе РЭУ.
  Для учета влияния температуры на эксплуатационную интенсивность
  отказов элементов  $\lambda_{\textrm{Э}}$ принято во внимание
  верхнее значение предельной рабочей температы
  ($t_{\textrm{раб.max}}= 40°C$), cоответствующей РЭУ
  исполнения УХЛ4.2 по ГОСТ 15150-69, и возможное увеличение предельной рабочей температуры на значение
  $\Delta t_C = 10°C$ за счёт нагрева РЭУ и, следовательно модуля в составе РЭУ.
  Предельная рабоачая температура $t_{эл.max}$ теплонагруженных
  элементов (ИМС, транзисторы, диоды, мощные резисторы) определена как \cite{Borovikov2010}:
  \begin{equation}
    t_{\textrm{эл.max}} = (t_{\textrm{раб.max}} + \Delta t_C) + \Delta t_{\textrm{з}} = (40 + 10) +15 = 65°С
  \end{equation}
  где $\Delta t_{\textrm{з}}$ — перегрев нагретой зоне конструкции РЭУ
  Нагретая зона — это гиптотетический объём, в котором условно
  рассеивается вся тепловая энергия, выделяемая РЭУ.

  Значение величины $t_{эл.max}$  для нетеплонагруженных элементов
  (конденсаторы, слабонагруженные резисторы, соединительс, кварцевый резонатор)
  подсчитано как ~\cite{Borovikov2010}:
  \begin{equation}
    t_{\textrm{эл.max}} = (t_{\textrm{раб.max}} + \Delta t_C) + \Delta t_{\textrm{з}} = (40 + 10) +10 = 60°С
  \end{equation}
  
\item Пользуясь таблице 5.3 из источника~\cite{Borovikov2010} находим
  справочные значения интенсивности отказов элементов модуля.
  \begin{table}[H]
    \centering
        \caption{Значения $\lambda_{\textrm{Б}}$ для каждого элемента}
    \begin{tabular}{|l|r|}
      \hline
      Комопонент & Значение $\lambda_{\textrm{Б}}$ \\ \hline
      R1...R4 & 0,044 \\ \hline
      R5...R8 & 0,183 \\ \hline
      R9...R10 & 0,044 \\ \hline
      R11, R12 & 0,044 \\ \hline
      R13 & 0,044 \\ \hline
      R14 & 0,044 \\ \hline
      R15 & 0,044 \\ \hline
      C1 & 0,173 \\ \hline
      C2 & 0,173 \\ \hline
      C3 & 0,173 \\ \hline
      C4, C5 & 0,022 \\ \hline
      DA1 & 0,023 \\ \hline
      DD1 & 0,023 \\ \hline
      HA1 & 0,034 \\ \hline
      HL1 & 0,034 \\ \hline
      SA1, SA2 & 0,0104 \\ \hline
      VD1...VD3 & 0,091 \\ \hline
      VT1 & 0,044 \\ \hline
      X1... X5 & 0,0041 \\ \hline
    \end{tabular}
  \end{table}
\end{enumerate}
\newpage

%%% Local Variables:
%%% mode: LaTeX
%%% TeX-master: "main"
%%% End:
