
\tableofcontents
\newpage
\begin{center}
\textbf{Перечень условных обозначений, символов и терминов}
\end{center}

БИС - большие интегральные схемы.

ГОСТ — государственный стандарт.

ГЖП - гибко-жесткие печатные платы.

ГПК - гибкий печатных кабель.

ДПП - двухсторонняя печатная плата.

ЭРИ - электрорадиоизделия.

МКЭ – метод конечных элементов.

МСБ - микросборки.

САПР — системы автоматизированного проектирования.

СБИС - cверх большие интегральные схемы.

ШИМ — широко импульсная модуляция.

ПП — печатная плата.

РЭС — радиоэлектронное средство.

ИМС - интегральные микросхемы.

ЭА - электронная аппаратура.

ЭРИ - электрорадиоизделие.

ЭРЭ - электрорадиоэлемент.

\textit{CAE} — \textit{Computer Aided Engineering}, система автоматизации инженерных расчетов.

\textit{DPDT} — \textit{Double Pole Double Throw}, переключатель два полюса, два направления.

\textit{DIP} — \textit{Dual In-line Package}, корпус с двумя рядами
прямогуольных выводов.

\textit{GPIO} — \textit{ General Purpose Input Output}, система ввода-вывода общего пользования

\textit{OLED} — \textit{Organic light emitted diode}, органический светодиод.

\textit{I2C} — \textit{Inter-Integrated Circuit}, интерфейс микрокнтроллера.

\textit{TDP} — \textit{Thermal Dessipation Power}, рассеиваемая тепловая мощность.

\textit{PCB} — \textit{Printed Circuti Board}, печатная плата.

\newpage

\begin{center}
\textbf{ВВЕДЕНИЕ}
\end{center}

Целью курсовой работы является проектирование конструкции
устройства системы тестирования сервоприводов.

Неотъемлимой частью системы тестирования является само устройство
тестирования — модуль печатной платы предназначенный для
проверки электрически и функциональных характеристик сервопривода.

Цель процесса проектирования состоит в том, чтобы на основании
априорной (исходной) и апостериорной (дополнительной) информации,
поступающей в процессе проектирования, получить полное описание
объекта проектирования в виде технической документации, необходимой
для его изготовления, удовлетворяющего заданным требованиям и
ограничениям.

Для достижения поставленной цели в работе решаются следующие задачи:
\begin{enumerate}
\item Анализ литературных и патентных исследований, включающий обзор
методов тестирования сервоприводов, а также анализ существующих патентов
в этой области.
\item Общетехнический анализ проектируемого устройства, в котором
будет проведен анализ исходных данных; схемотехнический анализ, в
котором рассмотрен принцип работы проектируемого устройства;
составлено техническое задание к устройству
\item Разработка конструкции проектируемого изделия состоит из выбора:
конструкторских решений, обеспечивающих удобство ремонта и
эксплуатации устройства; типа электрического монтажа, элементов
крепления и фиксации; способов защиты устройства от внешних
воздействий; способов обеспечения нормального теплового режима
устройства и выбора элементной базы, конструктивных элементов.

\item Рассмотрение применяемых средств автоматизированного проектирования
при разработке устройства, где будет обоснован выбор пакетов
прикладного программного обеспечения для моделирования и
проектирования устройства, а также технология применения средств
автоматизированного проектирования при разработке конструкторской
документации.
\end{enumerate}

Курсовой проект выполнен самостоятельно, проверен в системе «Атиплагиат».
Процент оригинальности составляет n\%. Цитирования обозначены
ссылками на публикации, указанными в
«Списке использованных источников».
\newpage
%%% Local Variables:
%%% mode: LaTeX
%%% TeX-master: "main"
%%% End:
