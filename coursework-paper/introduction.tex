\tableofcontents
\newpage
\begin{center}
\textbf{Перечень условных обозначений, символов и терминов}
\end{center}

ГОСТ — государственный стандарт.

ДПП - двухсторонняя печатная плата.

ЭРИ - электрорадиоизделия.

МКЭ – метод конечных элементов.

САПР — системы автоматизированного проектирования.

ШИМ — широко импульсная модуляция.

ПП — печатная плата.

РЭС — радиоэлектронное средство.



\textit{CAE} — \textit{Computer Aided Engineering}, система автоматизации инженерынх расчетов.

\textit{DPDT} — \textit{Double Pole Double Throw}, переключатель два полюса, два направления.

\textit{GPIO} — \textit{ General Purpose Input Output}, система ввода-вывода общего пользования

\textit{OLED} — \textit{Organic light emitted diode}, органический светодиод.

\textit{I2C} — \textit{Inter-Integrated Circuit}, интерфейс микрокнтроллера.

\textit{TDP} — \textit{Thermal Dessipation Power}, рассеиваемая тепловая мощность.

\textit{PCB} — \textit{Printed Circuti Board}, печатная плата.

\newpage

\begin{center}
\textbf{ВВЕДЕНИЕ}
\end{center}


В данной курсовой работе рассмотрены результаты проектривования 
печатной платы электронного модуля.

Проведены расчёты компоновки модуля, расчеты теплового перегрева
модуля.

Целью курсовой работы является общетехнический анализ устройства,
его проектирование моделерование, а также анализ полученных результатов.

\newpage
%%% Local Variables:
%%% mode: LaTeX
%%% TeX-master: "main"
%%% End:
