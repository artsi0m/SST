\begin{center}  
\textbf{РЕФЕРАТ}
\end{center}

БГУИР КР 1-39 02 03 013 ПЗ

\textbf{Корякин А.Л.} Устройство для тестирования
сервоприводов квадрокоптера:
пояснительная записка к курсовой работе/А.Л.Корякин.
— Минск, БГУИР: 2024, 46 с.
Пояснительная записка 46c. 5 рис., 1 табл., 33 источников.

УСТРОЙСТВО ДЛЯ ТЕСТИРОВАНИЯ СЕРВОПРИВОДОВ КВАДРОКОПТЕРА,
АНАЛИЗ-ЛИТЕРАТУРНО ПАТЕНТНЫХ
ИССЛЕДОВАНИЙ, ОБЩЕТЕХНИЧЕСКОЕ ОБОСНОВАНИЕ РАЗРАБОТКИ УСТРОЙСТВА,
СХЕМОТЕХНИЧЕСКИЙ АНАЛИЗ, УСЛОВИЯ ЭКСПЛУАТАЦИИ, РАЗРАБОТКА КОНСТРУКЦИИ
ПРОЕКТИРУЕМОГО ИЗДЕЛИЯ, РАСЧЕТ КОНСТРУКТИВНЫХ ТЕХНОЛОГИЧЕСКИХ
ПАРАМЕТРОВ УСТРОЙСТВА ПЛАВНОГО ПУСКА, СРЕДСТВА АВТОМАТИЗИРОВАННОГО
ПРОЕКТИРОВАНИЯ ДЛЯ РАЗРАБОТКИ УСТРОЙСТВА.

\textit{Цель проектирования:} проектирование конструкции устройства
тестирования сервоприводов квадропотера.

\textit{Методология проведения работы:}
в процессе решения поставленных задач
использованы принципы системного подхода, теория схемотехнического и
конструкторско-технологического проектирования РЭС, аналитические и
физико-математические методы, методы компьютерной обработки
экспериментальных данных и компьютерного моделирования.

\textit{Результаты работы:}
выполнен анализ литературно-патентных исследований;
рассмотрено общетехническое обоснование разработки устройства; сделан
схемотехнический анализ радиоэлектронного средства; разработана
конструкция проектируемого изделия; рассчитаны
конструктивные технологические параметры проектируемого изделия;
выполнено обоснование выбора пакетов прикладного программного
обеспечения для моделирования и проектирования устройства; разработана
графическая часть проекта.

\textit{Область применения результатов:}
могут быть использованы при
проектировании систем тестирования сервоприводов,
в качестве рабочей документации.

\newpage

%%% Local Variables:
%%% mode: LaTeX
%%% TeX-master: "main"
%%% End:
