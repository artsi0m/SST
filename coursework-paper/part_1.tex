
%%% Local Variables:
%%% mode: LaTeX
%%% TeX-master: "main"
%%% End:

\section{анализ литературно \\ патентных исследований}

\subsection{Обзор методов и средств тестирования сервоприводов}

Под сервоприводом в данной работе подразумевается вращающийся привод,
управляющий широтно-импульсной модуляцией, который за счёт обратной
связи позволяет точно контролировать вал сервопривода, поворачивая его
на заданный угол или поддерживая определенную скорость или ускорение.

Широтно-импульсная модуляция это метод модуляции сигнала, как
прямоугольной волны с переменной длиной между амлитудами.

Малогабаритные быстродействующие сервоприводы применяются в
современных высокоточных системах управления подвижными объектами:
рулевыми системами летательных аппаратов, автоматическими
манипуляторами, роботами с подвижными элементами конструкции и
др~\cite{dyakovSUBSTANTIATIONRELIABILITYSERVOMOTORS2023}.

В типичной системе контроля полета квадракоптера или дрона пять
парамеров подлежат отслеживанию: длина четырех управляющих
сервоприводами импульсов, которые контролируют тягу, крен, тангаж и
рысканье, и напряжение питания.

В данной работе рассматривается система тестирования сервоприводов
квадрокоптера.  Назначение этой системы — тестирование отдельных
частей квадрокоптера, таких как сервоприводы,
блок управления скоростью или контроллер полёта ~\cite{Elector521}.

Говоря кратко, разрабатываемое изделие,
можно назвать одним общепринятым словом — сервотестер.

Сервотестер выполнен в виде платы, на которую, с помощью монтажа в
отверстия помещаются компоненты. Такой способ монтажа выбран для того,
чтобы облегчить сборку данной схемы, установку и смену тестируемых
компонентов или незначительного изменения схемы, необходимого для
взаимодействия с определенными компонентами, а также возможный ремонт.



В случае сложной конструкции, например при разработке полётного
контроллера, данный тестовый стенд позволяет наблюдать получаемый
сигнал ~\cite{Elector521}.
