\begin{center}
\textbf{Заключение}
\end{center}

В результате выполнения курсового проекта была спроектирована конструкция
системы тестирования сервоприводов квадрокоптера,
проведен общетехнический анализ проектируемого устройства,
который включает: анализ исходных
данных, описание принципа работы анализируемого устройства, анализ
электрической принципиальной схемы устройства,
определение параметров компоновки, расчёт надёжности и эргономичности.

Также был проведён анализ литературно-патентных исследований с
целью выявления схожих устройств.

Проведено моделирование физических процессов, воздействующих на
устройство, а именно его нагрев.
Основываясь на вышеперечисленном, был проведен анализ полученных
результатов, а также дана краткая оценка.

Разработана конструкция проектируемого изделия с учётом
конструкторских решений, обеспечивающих удобство ремонта и
эксплуатации устройства, обеспечения нормального теплового режима
устройства.

Был произведён расчет конструктивно-технологических параметров
проектируемого изделия, результаты которого показали, что устройство
спроектировано в соответствии с ТЗ.

Устройство обладает высокой вероятностью безотказной работы за 1000.

Для моделирования и проектирования устройства были использованы
следующие САПР: \textit{KiCAD}, Компас3D.

Цели работы выполнены в полном объёме. Данные могут быть применены для
разработки системы тестирования сервоприводов квадрокоптера.

\newpage

\begin{center}
\textbf{Приложение А}\\
\textbf{Обязательное}\\
\textbf{Параметры компонентов}
\end{center}

Техническое описание ПП:
\begin{enumerate}[label={\arabic*.}]
\item Габариты ПП 80 на 84,5 мм.
\item Толщина ПП 1,6 мм.
\end{enumerate}

Данные необходимые для тепловго анализа:
\begin{enumerate}[label={\arabic*.}]
\item Температура окружающей среды 315 К.
\item Коэффициент конвекции — 25 Вт/К * м².
\end{enumerate}


\newpage
%%% Local Variables:
%%% mode: LaTeX
%%% TeX-master: "main"
%%% End:
