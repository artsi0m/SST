\documentclass[a4paper]{bsuir-std}
\usepackage{pdflscape} % for album orientation

\begin{document}
\thispagestyle{empty} % no page number
\vspace*{\fill}
\begin{center}
  СПРАВКА\\

  ОБ ИССЛЕДОВАНИИ ПАТЕНТНОЙ И НАУЧНОЙ ТЕХНИЧЕСКОЙ ЛИТЕРАТУРЫ ПО ТЕМЕ ДИПЛОМНОГО ПРОЕКТА\\
  \rule{17cm}{1pt}
  Система автоматического управления беспилотным летательным аппаратом мультироторного типа
  \rule{17cm}{1pt}
\end{center}
\vspace{\fill}

\newpage
\thispagestyle{empty} % no page number
\begin{landscape}
  
\captionsetup[table]{
    format=bsuirtable,
    singlelinecheck=false,
    labelsep=endash,
    skip=1mm,
    position=above,
    parindent=0pt, % Убираем стандартный отступ
    labelformat=empty, % Не нумеруем таблицы автоматически, подписываем вручную.
  }

\begin{center}
    ПАТЕНТНЫЕ ИССЛЕДОВАНИЯ ПО ТЕМЕ ДИПЛОМНОЙ РАБОТЫ 
\end{center}
\begin{table}[H]
  \centering
  \caption{Таблица Б.1 - Патентные исследования}
  \small
  \begin{tabular}{|p{0.15\linewidth}|p{0.1\linewidth}|p{0.1\linewidth}|p{0.15\linewidth}|p{0.15\linewidth}|p{0.2\linewidth}|}
    \hline
    Основные технические данные для поиска & Страны & Класс МПК
    & Что и за какой период просмотрено                                                  
    & №, название выявленных аналогов
    & Анализ уровня исследуемой темы
      (тенденции развития). Выводы и рекомендации \\ \hline
    1 & 2 & 3 & 4 & 5& 6 \\ \hline
    % №1
    Полётный контроллер & USA &  B64U30/20
    & Интернет сервис patents.google.com
    & US10144527B2, Полётный контроллер с управляемым генератором.
    & Изобретение представляет собой полётный контроллер,
      который включает в себя отдельный вход для
      сигнала поступающего с электронного контроллера скорости
      и контроллер определяющий входной сигнал для генератора,
      для удовлетворения ожидаемого запроса на энергию питания \\ \cline{2-3} \cline{5-6}
    % №2
    & Taiwan &  G01C21/18 &
    & TW202504822A,
      Мультикоптер на базе смарт-устройства в качестве полётного
      контроллера.
    & В патенте рассмотрен беспилотный летательный аппарат
      мультироторного типа, использующий умное устройство в качестве
      полётного контроллера.\\ \hline
  \end{tabular}
\end{table}
\newpage
\thispagestyle{empty} % no page number

\begin{table}[H]
  \centering
  \caption{Продолжение таблицы Б.1}
  \small
  \begin{tabular}{|p{0.15\linewidth}|p{0.1\linewidth}|p{0.1\linewidth}|p{0.15\linewidth}|p{0.15\linewidth}|p{0.2\linewidth}|}
    \hline
    Основные технические данные для поиска & Страны & Класс МПК
    & Что и за какой период просмотрено                                                  
    & №, название выявленных аналогов
    & Анализ уровня исследуемой темы
      (тенденции развития). Выводы и рекомендации \\ \hline
    1 & 2 & 3 & 4 & 5& 6 \\ \hline
    % № 3
    Полётный контроллер & USA &  B64C39/02
    & Интернет сервис patents.google.com
    & US20210245877A1,  Синхронизированный конвейерный контроллер полета.
    &  Представлено устройство полётного контроллера с синхронизированным алгоритмом взаимодействия между сенсорами и
      актуаторами. \\ \cline{2-3} \cline{5-6}
    % № 4
    & USA & G05D1/0038
    && US10551834B2, Метод и электронное устройство для управления беспилотным летательным аппаратом.
    & В патенте рассмотрено устройство для управления беспилотным летательным аппаратом
      и соответствующий метод управления. \\ \cline{2-3} \cline{5-6}
      % № 5
    & China &  G05B19/0423 &
    & CN113341830A, Четырехроторный полётный контроллер.
    & В патенте представлен полётный контроллер четырёхроторного
      беспилотного летательного аппарата.
       Патент включает множество принципиальных схем отдельных каскадов. \\ \hline
                                                      
  \end{tabular}
\end{table}

\end{landscape}

\thispagestyle{empty} % no page number
\begin{center}
  НАУЧНОЕ-ТЕХНИЧЕСКАЯ ЛИТЕРАТУРА И ТЕХНИЧЕСКАЯ ДОКУМЕНТАЦИЯ 
\end{center}
\begin{enumerate}
\item Интернет сервис https://patents.google.com
\end{enumerate}
\vspace{\fill}

Достоверность сведений удостоверяю:\\

Руководитель: \underline{\hspace*{5.6cm}}
\end{document}
%%% Local Variables:
%%% mode: LaTeX
%%% TeX-master: t
%%% End:
