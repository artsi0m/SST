\section{Расчет параметров проектируемого изделия}

\subsection{Расчет теплового режима (выбор способа охлаждения;
  описание тепловых моделей;  оценка теплового режима). }

% Будет приведены аргументы в пользу выбора пассивного охлаждения в
% данном изделии, касающиеся простоты изделия и отсутствия каких-либо
% устройств выделяющих большой объём теплоты.

% Приведены расчёты соответствующие пассивному охлаждению в герметичном
% корпусе и в перфорированном корпусе.

\subsection{Расчёт на механические воздействия. }

% По толщине, длине, ширине и массе печатной платы будет определена
% собственная частота печатной платы согласно ~\cite{Kalenkovich2012}
% года издания.

\subsection{Расчет конструктивно-технологических параметров печатных плат. }

% Основываясь на~\cite{Kostukevich2012}, приведен расчёт
% объёмно-компоновочных характеристик устройства на основе данных о
% имеющихся на ней компонентах и коэффициенте заполнения.

\subsection{Расчёт электромагнитной совместимости}

% Согласно одному из актуальных пособий будет выбрана методика расчёта
% электромагнитной совместимости о осуществлён расчёт на основании этой
% методики.

\subsection{Полный расчёт надёжности. }

% Будет приведен полный расчёт надёжности на основании данных о
% компонентах из перечня элементов и данных полученных в ходе расчетов
% электрических параметров из главы 3.

Под надёжностью понимают свойство изделия сохранять в течение
заданного времени в пределах установленных норм значения
функциональных параметров при определённых условиях (заданные режимы и
условия эксплуатации, техническокого обслуживания, хранения и
транспортирования)~\cite{Borovikov2010}.

В теории и практике надёжности технических изделий широко используют
понятие наработка, под которой понимают продолжительность работы
изделия, выраженную в часах, циклах переключения или других единиц в
зависимости от вида и функционального назначения изделия ~\cite{Borovikov2010}.

В теории и практике надёжности технических изделий широко используют
понятие наработка, под которой понимают продолжительность работы
изделия, выраженную в часах, циклах переключения или других единиц в
зависимости от вида и функционального назначения изделия ~\cite{Borovikov2010}.

Проведём уточнённый расчёт показателей безотказности функционального модуля:
\begin{enumerate}
\item Находим коэффициент электрической нагрузки элементов, пользуюясь
картами электрических режимов и эксплуатационными электрическими
характеристиками используемыми в модуле.

Коэффициент электрической нагрузки элементов равен
\begin{equation}
  K_{\text{Н}} = \frac{F_{\text{раб}}}{F_{\text{ном}}}
\end{equation}

При этом исходя из того, что в данной работе расчёт производится на
раннем этапе проектирования, выходит возможным, на
основании данных полученных в результате САПР для симуляции
электронных схем, узнать показатели $F_{\text{раб}}$ и
на основании полученных данных подбирать компоненты из широкого
ассортимента представленных на рынке с таким показателем
$F_{\text{ном}}$ чтобы искуственно изменять коэффицент
$K_{\text{Н}}$ в лучшую сторону. 

Приняв во внимание,  вышеописанную возможность примем
$K_{\text{Н}}$ у элементов равным $0,8$.


\item Определим максимальную температуру элементов модуля при его
работу в составе РЭУ.  Для учета влияния температуры на
эксплуатационную интенсивность отказов элементов $\lambda_{\text{Э}}$
принято во внимание верхнее значение предельной рабочей температы
($t_{\text{раб.max}}= 40°C$), cоответствующей РЭУ исполнения УХЛ4.2 по
ГОСТ 15150-69, и возможное увеличение предельной рабочей температуры
на значение $\Delta t_C = 10°C$ за счёт нагрева РЭУ и, следовательно
модуля в составе РЭУ.  Предельная рабоачая температура $t_{эл.max}$
теплонагруженных элементов (ИМС, транзисторы, диоды, мощные резисторы)
определена как \cite{Borovikov2010}:
  \begin{equation}
    t_{\text{эл.max}} = (t_{\text{раб.max}} + \Delta t_C) + \Delta t_{\text{з}} = (40 + 10) +15 = 65°С
  \end{equation}
  где $\Delta t_{\text{з}}$ — перегрев нагретой зоне конструкции РЭУ
Нагретая зона — это гиптотетический объём, в котором условно
рассеивается вся тепловая энергия, выделяемая РЭУ.

  Значение величины $t_{эл.max}$ для нетеплонагруженных элементов
(конденсаторы, слабонагруженные резисторы, соединительс, кварцевый
резонатор) подсчитано как ~\cite{Borovikov2010}:
  \begin{equation}
    t_{\text{эл.max}} = (t_{\text{раб.max}} + \Delta t_C) + \Delta t_{\text{з}} = (40 + 10) +10 = 60°С
  \end{equation}
  %
\item Пользуясь таблице 5.3 из источника~\cite{Borovikov2010} находим
  справочные значения интенсивности отказов элементов модуля.

\item По таблице 5.1 из ~\cite{Borovikov2010} выбираем математические
модели расчёта эксплуатационной интеcивности отказов элементов
$\lambda_{\text{Э}}$
% ---

\item Определяем значения поправочных коэффициентов, входящих в
выбранные модели расчёта эксплутационной интесивности отказов
элементов $\lambda_{\text{Э}}$
\item Для каждого элемента находим произведение поправочных
коэффициентов, и значение эксплуатационной интесивности отказов
$\lambda_{\text{Э}}$.

% СЮДА ВСТАВЛЯТЬ ТАБЛИЦЫ

\item Подсчитываем эксплуатационную интенсивность отказов модуля.
  Для этого проссумируем значения, приведенные в последнем столбце.
  $\lambda_{\text{М}} = 36,53 \cdot 10^{-6} \text{ 1/ч}$
  
\item Находим наработку на отказ:
  $T_0 = 1 / \Lambda_{\text{М}} = 27,377 \text{ часов}$

  
\end{enumerate}
\newpage

%%% Local Variables:
%%% mode: LaTeX
%%% TeX-master: "main"
%%% End:
