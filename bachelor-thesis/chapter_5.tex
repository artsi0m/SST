\section{Расчет параметров проектируемого изделия}

\subsection{Расчет теплового режима \\
  (выбор способа охлаждения; \\
  описание тепловых моделей; \\
  оценка теплового режима). }

% Будет приведены аргументы в пользу выбора пассивного охлаждения в
% данном изделии, касающиеся простоты изделия и отсутствия каких-либо
% устройств выделяющих большой объём теплоты.

% Приведены расчёты соответствующие пассивному охлаждению в герметичном
% корпусе и в перфорированном корпусе.

\subsection{Расчёт на механические воздействия. }

% По толщине, длине, ширине и массе печатной платы будет определена
% собственная частота печатной платы согласно ~\cite{Kalenkovich2012}
% года издания.

\subsection{Расчет конструктивно-технологических параметров\\
  печатных плат. }

% Основываясь на~\cite{Kostukevich2012}, приведен расчёт
% объёмно-компоновочных характеристик устройства на основе данных о
% имеющихся на ней компонентах и коэффициенте заполнения.

\subsection{Расчёт электромагнитной совместимости}

% Согласно одному из актуальных пособий будет выбрана методика расчёта
% электромагнитной совместимости о осуществлён расчёт на основании этой
% методики.

\subsection{Полный расчёт надёжности. }

% Будет приведен полный расчёт надёжности на основании данных о
% компонентах из перечня элементов и данных полученных в ходе расчетов
% электрических параметров из главы 3.

\newpage

%%% Local Variables:
%%% mode: LaTeX
%%% TeX-master: "main"
%%% End:
