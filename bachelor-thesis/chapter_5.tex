\section{Расчет параметров проектируемого изделия}

\subsection{Расчет теплового режима (выбор способа охлаждения;
  описание тепловых моделей;  оценка теплового режима). }

% Будет приведены аргументы в пользу выбора пассивного охлаждения в
% данном изделии, касающиеся простоты изделия и отсутствия каких-либо
% устройств выделяющих большой объём теплоты.

% Приведены расчёты соответствующие пассивному охлаждению в герметичном
% корпусе и в перфорированном корпусе.

В процессе эксплуатации РЭС на их элементы воздействуют температуры,
источниками которых выступают не только окружающая среда, но и сама
работающая техника. Любой блок РЭС с точки зрения теплофизики
представляет собой систему множества тел, в которых тепловая энергия
распределена в пространстве конкретного блока сложным
образом. Тепловую энергию выделяют активные элементы схемы за счет
потребления электрической энергии. Внутреннее тепловыделение,
совмещенное с воздействием температуры окружающей среды, приводит к
изменению электрических характеристик РЭС. Эти изменения могут быть
как обратимыми, так и необратимыми, а их величина варьируется от
незначительных до критических, способных вызвать отказ устройства.

Тепловой режим радиоэлемента определяется его температурным
состоянием, то есть пространственно-временным распределением
температуры внутри элемента. Существенные отклонения температуры
устройства от номинального значения, особенно в сторону повышения,
резко снижают его надежность из-за перегрева. Расчет теплового режима
проводится для сравнения полученных температурных характеристик с
предельно допустимыми температурами, на которые рассчитаны компоненты
РЭС, чтобы избежать превышения допустимых значений.

% Данные для расчёта

Из исходных данных был сделан вывод, что расчет теплового режима
должен подразумевать использование РЭС в герметичном корпусе.

Расчёт теплового режима для РЭС выполняется в указанном порядке
~\cite{Rotkop1976}:
\begin{enumerate}
\item Рассчитывается поверхность корпуса блока
  \begin{equation}
    S_{К} = 2 \cdot (l_1 l_2 + (l_1+ l_2)l_3) % (4.46)
  \end{equation}

\item Определяется условная поверхность нагретой зоны:
  \begin{equation}
    S_{з} = 2 (l_1 l_2 + (l_1 + l_2) K_{з} l_3 ) % (4.39)
  \end{equation}

\item Определяется удельная мощность корпуса по блоку:
%
\begin{equation}
  q_к = P_з/S_к % (4.45)
\end{equation}
%
\item Рассчитывается удельная мощность нагретой зоны:
 % 
  \begin{equation}
      q_з = P_з/S_3 % (4.38)
    \end{equation}

\item Находится коэффициент $\vartheta_1$ в зависимости от удельной мощности корпуса блока:
 %   
\begin{equation}
\vartheta_1 = 0,1472q_к - 0,2962 \cdot 10^{-3}q_к^2 + 0,3127 \cdot 10^{-6}q_к^2
\end{equation}

\item Находится коэффициент $\vartheta_2$ в зависимости от удельной мощности нагретой среды:
%
\begin{equation}
\vartheta_2 = 0,1390q_к - 0,1223 \cdot 10^{-3}q_к^2 + 0,0698 \cdot 10^{-6}q_к^3
\end{equation}

\item Коэффициент $K_{Н1}$ в зависимости от давления
  среды вне корпуса блока берётся из ГОСТ~\cite{GOST-15150-69}.

\item Коэффициент $K_{Н2}$ в зависимости от давления
  среды внутри корпуса блока берётся из ГОСТ~\cite{GOST-15150-69}.

\item Определяется перегрев корпуса блока:
%
  \begin{equation}
    \vartheta_к = \vartheta_1 \cdot K_{Н1}
  \end{equation}

\item Рассчитывается перегрев нагретой зоны:
%
\begin{equation}
  \vartheta_з = \vartheta_к + (\vartheta_2 - \vartheta_1) \cdot K_{H2}
\end{equation}

\item Определяется средний перегрев воздуха в блоке:
%
\begin{equation}
  \vartheta_в = 0,5 \cdot (\vartheta_к + \vartheta_з)
\end{equation}

\item Определяется удельная мощность элемента:
  \begin{equation}
    q_{эл} = \frac{P_{эл}}{S_{эл}}
  \end{equation}

\end{enumerate}

Таким образом были найдены средний перегрев воздуха в блоке и перегрев
нагретой зоны. Уже этих данных достаточно для того, чтобы принять
конструкторское решение и приступить к выбору радиатора для самых
тепловыделяющих компонентов, или, что гораздо более предпочтительно с
точки зрения надёжности, ввести в конструкцию корпуса перфорацию для
того, чтобы свободно проходящий воздух охлаждал элементы.

\subsection{Расчёт на механические воздействия. }

% По толщине, длине, ширине и массе печатной платы будет определена
% собственная частота печатной платы согласно ~\cite{Kalenkovich2012}
% года издания.

\subsection{Расчет конструктивно-технологических параметров печатных плат. }

% Основываясь на~\cite{Kostukevich2012}, приведен расчёт
% объёмно-компоновочных характеристик устройства на основе данных о
% имеющихся на ней компонентах и коэффициенте заполнения.

\subsection{Расчёт электромагнитной совместимости}

% Согласно одному из актуальных пособий будет выбрана методика расчёта
% электромагнитной совместимости о осуществлён расчёт на основании этой
% методики.

\subsection{Полный расчёт надёжности. }

% Будет приведен полный расчёт надёжности на основании данных о
% компонентах из перечня элементов и данных полученных в ходе расчетов
% электрических параметров из главы 3.

Под надёжностью понимают свойство изделия сохранять в течение
заданного времени в пределах установленных норм значения
функциональных параметров при определённых условиях (заданные режимы и
условия эксплуатации, техническокого обслуживания, хранения и
транспортирования)~\cite{Borovikov2010}.

В теории и практике надёжности технических изделий широко используют
понятие наработка, под которой понимают продолжительность работы
изделия, выраженную в часах, циклах переключения или других единиц в
зависимости от вида и функционального назначения изделия ~\cite{Borovikov2010}.

В теории и практике надёжности технических изделий широко используют
понятие наработка, под которой понимают продолжительность работы
изделия, выраженную в часах, циклах переключения или других единиц в
зависимости от вида и функционального назначения изделия ~\cite{Borovikov2010}.

Проведём уточнённый расчёт показателей безотказности функционального модуля:
\begin{enumerate}
\item Находим коэффициент электрической нагрузки элементов, пользуюясь
картами электрических режимов и эксплуатационными электрическими
характеристиками используемыми в модуле.

Коэффициент электрической нагрузки элементов равен
\begin{equation}
  K_{\text{Н}} = \frac{F_{\text{раб}}}{F_{\text{ном}}}
\end{equation}

При этом исходя из того, что в данной работе расчёт производится на
раннем этапе проектирования, выходит возможным, на
основании данных полученных в результате САПР для симуляции
электронных схем, узнать показатели $F_{\text{раб}}$ и
на основании полученных данных подбирать компоненты из широкого
ассортимента представленных на рынке с таким показателем
$F_{\text{ном}}$ чтобы искуственно изменять коэффицент
$K_{\text{Н}}$ в лучшую сторону. 

Приняв во внимание,  вышеописанную возможность примем
$K_{\text{Н}}$ у элементов равным $0,8$.


\item Определим максимальную температуру элементов модуля при его
работу в составе РЭУ.  Для учета влияния температуры на
эксплуатационную интенсивность отказов элементов $\lambda_{\text{Э}}$
принято во внимание верхнее значение предельной рабочей температы
($t_{\text{раб.max}}= 40°C$), cоответствующей РЭУ исполнения УХЛ4.2 по
ГОСТ 15150-69, и возможное увеличение предельной рабочей температуры
на значение $\Delta t_C = 10°C$ за счёт нагрева РЭУ и, следовательно
модуля в составе РЭУ.  Предельная рабоачая температура $t_{эл.max}$
теплонагруженных элементов (ИМС, транзисторы, диоды, мощные резисторы)
определена как \cite{Borovikov2010}:
  \begin{equation}
    t_{\text{эл.max}} = (t_{\text{раб.max}} + \Delta t_C) + \Delta t_{\text{з}} = (40 + 10) +15 = 65°С
  \end{equation}
  где $\Delta t_{\text{з}}$ — перегрев нагретой зоне конструкции РЭУ
Нагретая зона — это гиптотетический объём, в котором условно
рассеивается вся тепловая энергия, выделяемая РЭУ.

  Значение величины $t_{эл.max}$ для нетеплонагруженных элементов
(конденсаторы, слабонагруженные резисторы, соединительс, кварцевый
резонатор) подсчитано как ~\cite{Borovikov2010}:
  \begin{equation}
    t_{\text{эл.max}} = (t_{\text{раб.max}} + \Delta t_C) + \Delta t_{\text{з}} = (40 + 10) +10 = 60°С
  \end{equation}
  %
\item Пользуясь таблице 5.3 из источника~\cite{Borovikov2010} находим
  справочные значения интенсивности отказов элементов модуля.

\item По таблице 5.1 из ~\cite{Borovikov2010} выбираем математические
модели расчёта эксплуатационной интеcивности отказов элементов
$\lambda_{\text{Э}}$
% ---

\item Определяем значения поправочных коэффициентов, входящих в
выбранные модели расчёта эксплутационной интесивности отказов
элементов $\lambda_{\text{Э}}$
\item Для каждого элемента находим произведение поправочных
коэффициентов, и значение эксплуатационной интесивности отказов
$\lambda_{\text{Э}}$.

% СЮДА ВСТАВЛЯТЬ ТАБЛИЦЫ

\item Подсчитываем эксплуатационную интенсивность отказов модуля.
  Для этого проссумируем значения, приведенные в последнем столбце.
  $\lambda_{\text{М}} = 36,53 \cdot 10^{-6} \text{ 1/ч}$
  
\item Находим наработку на отказ:
  $T_0 = 1 / \Lambda_{\text{М}} = 27,377 \text{ часов}$

  
\end{enumerate}
\newpage

%%% Local Variables:
%%% mode: LaTeX
%%% TeX-master: "main"
%%% End:
