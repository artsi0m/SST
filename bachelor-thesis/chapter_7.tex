%\section{Экономическое обоснование}

\section{Технико экономическое\\
  обоснование \\
  разработки и производства\\
  системы автоматического управления\\
  беспилотным летательным аппаратом\\
  мультироторного типа}
% В этой главе будет приведён расчёт экономического обоснование
% внедрения в эксплуатацию программно-аппаратного комплекса.

% Пользователем данной разработки будут разработчики и производители
% беспилотных летательных аппаратов мультироторного типа.

% Данная разработка решает проблему полуавтоматического управления
% двигателями мультироторного летательного аппарата.

% Разработка окажется для пользователя более предпочтительных из-за
% простоты производства и эксплуатации.

% обосновать (предельно кратко) актуальность технического решения с
% точки зрения целевого рынка, его конкурентные преимущества, область
% применения и функциональное назначение, способ коммерциализации. Могут
% также рассматриваться вопросы патентования результатов разработки, а
% также проводиться оценка значимости разработки для импортозамещения.

% Необходимо описать изделие, область его применения, чем оно отличается
% от существующих аналогов, какие преимущества оно даёт пользователям
% либо предприятиям, которые используют его в качестве оборудования,
% устройства или прибора для производства продукции, оказания услуг,
% выполнения работ, либо конечным потребителям (населению). Кроме того
% обязательно описываются все преимущества разработанного изделия.

%\subsection{Характеристика проектного решения}
\subsection{Характеристика нового изделия}

% Описание изделия
Изделие представляет собой двухслойную печатную плату с распаянным на
ней микроконтроллером в \textit{DIP} корпусе, компонентами и
разъёмами.

% Область применения
Область применения данного изделия это беспилотная авиация, а именно
беспилотные мультироторные летательные аппараты, которые в силу своей
специфики, не обладают аэродинамической устойчивостью и потому,
нуждаются в постоянной стабилизации в полёте.

% Отличие от существующих аналогов
% Тут сравнить с тем, что я выбрал в патентах.

% Преимущества которые оно даёт
Преимущество этого изделия является простота его устройства и
изготовления, ремонтнопригодность и заменяемость.

\subsection{Расчёт себестоимости и отпускной цены нового изделия}

\begin{enumerate}
\item Расчёт затрат по статье «Сырьё и материалы».

  В данную статью включается стоимость основных и вспомогательных
  материалов, необходимых для изготовления единицы продукции по
  установленным нормам. Исходные данные взяты из констукторской
  документации.
  
  \begin{table}[H]
  \small
  \caption{Расчёт затрат на материалы}
  \begin{tabular}{|p{0.2\linewidth} | p{0.2\linewidth}| p{0.2\linewidth} |p{0.2\linewidth}| p{0.09\linewidth} |}
    \hline
Наименование материала & Еденица измерения & Норма расхода & Цена за единицу, BYN & Сумма\\[0pt]
\hline
    Сталь 10 & кг & 0,9 & 4,45 / 1,2 & 3,34\\[0pt]
    \hline
    Фторопласт фольгированный двухсторонний & мм² & 6800 & 1,08 за 1000  мм² & 7,34\\[0pt]
    \hline
    Всего &  &  &  & 10,68 \\[0pt]
    \hline
    C учётом транспортных расходов &  &  &  & 12,82 \\[0pt]
    \hline
\end{tabular}
\end{table}

%%% Local Variables:
%%% mode: LaTeX
%%% TeX-master: "main"
%%% End:


\item Расчёт затрат по статье «Покупные комплектующие изделия,
  полуфабрикаты, и услуги производственного характера»

  В данную статью включаются затраты на приобретение в поредке
  производственной кооперации готовых покупных изделий и полуфабрикатов,
  используемых для комплектования изделий или подвергающихся
  дополнительной обработке на данном предприятии для получения готовой
  продукции (радиоэлементы, микросхемы и прочее и пр.)

  \begin{longtable}{| p{0.25\linewidth} |  p{0.2\linewidth} |  p{0.15\linewidth} |  p{0.15\linewidth} |}
  \caption{Расчёт затрат на комплектующие изделия и полуфабрикаты} \\
\hline
Наименование комплектующего или полуфабриката & Кол-во на единицу, шт & Цена, BYN & Сумма, р\\%[0pt]
\hline
  Резистор 10 кОм & 7 & 0,05 & 0,35 \\
  \hline
  Резистор 22 Ом  & 2 & 0,04 & 0,08 \\[0pt]
  \hline
  Резистор 1 кОм  & 2 & 0,05 & 0,1\\[0pt]
  \hline
  Переменный резистор & 4 & 21 & 84\\[0pt]
  \hline
  Конденсатор 100 мкФ & 1 & 0,14 & 0,14\\[0pt]
  \hline
  Конденсатор 10 мкФ  & 1 & 0,10 & 0,1\\[0pt]
  \hline
  Конденсатор 100нФ & 1 & 0,75 & 0,75 \\[0pt]
  \hline
  Конденсатор 22пФ SMD0805 & 2 & 3,30 & 6,6\\[0pt]
  \hline
  Микросхема LM7805 & 1 & 2,35 & 2,35 \\[0pt]
  \hline
  Микроконтроллер ATMega328PU & 1 & 15,50 & 15,5 \\[0pt]
  \hline
  Зуммер & 1 & 4,10 & 4,1 \\[0pt]
  \hline
  Светодиод & 1 & 0,10 & 0,1 \\[0pt]
  \hline
  Переключатель DPDT & 1 & 11 & 11 \\[0pt]
  \hline
  Переключатель SPDT & 1 & 6 & 6\\[0pt]
  \hline
  Диод Шотки 1N5817 & 1 & 0,13 & 0,13 \\[0pt]
  \hline
  Стабилитрон BZX79C5V1 & 1 & 0,08 & 0,08 \\[0pt]
  \hline
  Микросхема LM385-BZ2.5 & 1 & 8,90 & 8,9\\[0pt]
  \hline
  Транзистор 2N7000 & 1 & 0,18 & 0,18 \\[0pt]
  \hline
  Разъём pld-6 & 1 & 0,22 & 0,22 \\[0pt]
  \hline
  Разъём KLDX-0202-AC & 1 & 4,90 & 4,9 \\[0pt]
  \hline
  Разъём SCT3001MH-2x6P & 2 & 1,45 & 2,9 \\[0pt]
  \hline
  Разъем PBD1.27-2-4 & 1 & 2,70 & 2,7 \\[0pt]
  \hline
  Кварцевый резонатор & 1 & 1,90 & 1,9\\[0pt]
  \hline
  Заклепки BN572 & 25 & 0,3Br за 200 штук & 0,0375\\[0pt]
  \hline
  Всего &  &  & 153,12\\[0pt]
  \hline
  Всего, с транспортнозаготовительными расходами &  &  & 168,43\\[0pt]
  \hline
\end{longtable}
    
%%% Local Variables:
%%% mode: LaTeX
%%% TeX-master: "main"
%%% End:

  
\end{enumerate}



\subsection{Расчёт инвестиций в производство нового изделия}

\subsection{Расчёт показателей экономической эффективности проекта}




\newpage

%%% Local Variables:
%%% mode: LaTeX
%%% TeX-master: "main"
%%% End:
