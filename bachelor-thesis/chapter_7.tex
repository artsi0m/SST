%\section{Экономическое обоснование}

\section{Технико экономическое\\
  обоснование \\
  разработки и производства\\
  системы автоматического управления\\
  беспилотным летательным аппаратом\\
  мультироторного типа}
% В этой главе будет приведён расчёт экономического обоснование
% внедрения в эксплуатацию программно-аппаратного комплекса.

% Пользователем данной разработки будут разработчики и производители
% беспилотных летательных аппаратов мультироторного типа.

% Данная разработка решает проблему полуавтоматического управления
% двигателями мультироторного летательного аппарата.

% Разработка окажется для пользователя более предпочтительных из-за
% простоты производства и эксплуатации.

% обосновать (предельно кратко) актуальность технического решения с
% точки зрения целевого рынка, его конкурентные преимущества, область
% применения и функциональное назначение, способ коммерциализации. Могут
% также рассматриваться вопросы патентования результатов разработки, а
% также проводиться оценка значимости разработки для импортозамещения.

% Необходимо описать изделие, область его применения, чем оно отличается
% от существующих аналогов, какие преимущества оно даёт пользователям
% либо предприятиям, которые используют его в качестве оборудования,
% устройства или прибора для производства продукции, оказания услуг,
% выполнения работ, либо конечным потребителям (населению). Кроме того
% обязательно описываются все преимущества разработанного изделия.

%\subsection{Характеристика проектного решения}
\subsection{Характеристика нового изделия}

% Описание изделия
Изделие представляет собой двухслойную печатную плату с распаянным на
ней микроконтроллером в \textit{DIP} корпусе, компонентами и
разъёмами.

% Область применения
Область применения данного изделия это беспилотная авиация, а именно
беспилотные мультироторные летательные аппараты, которые в силу своей
специфики, не обладают аэродинамической устойчивостью и потому,
нуждаются в постоянной стабилизации в полёте.

% Отличие от существующих аналогов
% Тут сравнить с тем, что я выбрал в патентах.

% Преимущества которые оно даёт
Преимущество этого изделия является простота его устройства и
изготовления, ремонтнопригодность и заменяемость.

\subsection{Расчёт себестоимости и отпускной цены нового изделия}

\begin{table}[H]
  \small
  \caption{Расчёт затрат на материалы}
  \begin{tabular}{|p{0.2\linewidth} | p{0.2\linewidth}| p{0.2\linewidth} |p{0.2\linewidth}| p{0.09\linewidth} |}
    \hline
Наименование материала & Еденица измерения & Норма расхода & Цена за единицу, BYN & Сумма\\[0pt]
\hline
    Сталь 10 & кг & 0,9 & 4,45 / 1,2 & 3,34\\[0pt]
    \hline
    Фторопласт фольгированный двухсторонний & мм² & 6800 & 1,08 за 1000  мм² & 7,34\\[0pt]
    \hline
    Всего &  &  &  & 10,68 \\[0pt]
    \hline
    C учётом транспортных расходов &  &  &  & 12,82 \\[0pt]
    \hline
\end{tabular}
\end{table}

%%% Local Variables:
%%% mode: LaTeX
%%% TeX-master: "main"
%%% End:


\subsection{Расчёт инвестиций в производство нового изделия}

\subsection{Расчёт показателей экономической эффективности проекта}




\newpage

%%% Local Variables:
%%% mode: LaTeX
%%% TeX-master: "main"
%%% End:
