\section{Экономическое обоснование}

% В этой главе будет приведён расчёт экономического обоснование
% внедрения в эксплуатацию программно-аппаратного комплекса.

% Пользователем данной разработки будут разработчики и производители
% беспилотных летательных аппаратов мультироторного типа.

% Данная разработка решает проблему полуавтоматического управления
% двигателями мультироторного летательного аппарата.

% Разработка окажется для пользователя более предпочтительных из-за
% простоты производства и эксплуатации.

% обосновать (предельно кратко) актуальность технического решения с
% точки зрения целевого рынка, его конкурентные преимущества, область
% применения и функциональное назначение, способ коммерциализации. Могут
% также рассматриваться вопросы патентования результатов разработки, а
% также проводиться оценка значимости разработки для импортозамещения.

% Необходимо описать изделие, область его применения, чем оно отличается
% от существующих аналогов, какие преимущества оно даёт пользователям
% либо предприятиям, которые используют его в качестве оборудования,
% устройства или прибора для производства продукции, оказания услуг,
% выполнения работ, либо конечным потребителям (населению). Кроме того
% обязательно описываются все преимущества разработанного изделия.


\subsection{Автоматическая система управления \\
  беспилотным летательным аппаратом\\
мультироторного типа}

% Описание изделия
Изделие представляет собой двухслойную печатную плату с распаянным на
ней микроконтроллером в DIP корпусе, компонентами и
разъёмами.

% Область применения
Область применения данного изделия это беспилотная авиация, а именно
беспилотные мультироторные летательные аппараты, которые в силу своей
специфики, не обладают аэродинамической устойчивостью и потому,
нуждаются в постоянной стабилизации в полёте.

Из-за этого устройство существенно отличается от аналогов
используемых в пилотируемой авиации и летательных аппаратов обладающих
аэродинамической устойчивостью, а также тем классом устройств, что называют
автопилотами. Летательный аппарат не обладающий аэродинамической
стабильностью, может управляться оператором удаленно и быть
беспилотным, но ему необходим полётный контроллер, который получает
данные с внешних датчиков, а затем управляет двигателями с учётом
того, как на полёт влияет инерция, аэродинамика и внешняя среда.

% Данный полётный контроллер призван управлять пятью параметрами:
% \begin{itemize}
% \item Тяга,
% \item крен,
% \item тангаж,
% \item рысканье,
% \item напряжение питания двигателя.
% \end{itemize}

Управление двигателями осуществляется импульсами ШИМ, совместимыми по
формату с теми, которые используются при управлении
сервоприводами. При том, в той же форме импульса ШИМ на полётный
контроллер поступают команды управления.  Они могут быть отправлены в
зависимости от конфигурации:
\begin{enumerate}
\item Оператором с земли. То есть будут приходить через приёмник сигнала;
\item С бортового компьютера на беспилотном летательном аппарате, в случае автономного использования;
\item С компьютера с земли, управляющего одним или несколькими
беспилотными летательными аппаратами. В таком случае сигнал также
может быть принят с земли через отдельный приёмник.
\end{enumerate}
Кроме того возможен автономный режим работы, когда полётный контроллер
не считывает внешние импульсы ШИМ, а только отправляет им.

Для взаимодействия с прочими устройствами, установленными на
мультироторный летательный аппарат используется интерфейс TWI.
% Отличие от существующих аналогов
% Тут сравнить с тем, что я выбрал в патентах.


% Преимущества которые оно даёт
Преимущество этого изделия является простота его устройства и
изготовления, ремонтопригодность и заменяемость.
Также этот полётный контроллер может в работать в автономной
конфигурации, в том числе когда на него не поступает внешнее питание.
Использование интерфейса TWI позволяет подключить до сотни
устройств к одному полётному контроллеру, не считая управляемых им же
двигателей и приёмник.


\subsection{Расчёт себестоимости и отпускной цены нового изделия}

Формирование отпускной цены изделия ведётся в соответствии с
~\cite{bsuir-project-economics}.

\begin{enumerate}
\item Расчёт затрат по статье «Сырьё и материалы».

  В данную статью включается стоимость основных и вспомогательных
  материалов, необходимых для изготовления единицы продукции по
  установленным нормам. Исходные данные взяты из конструкторской
  документации.
  
  \begin{table}[H]
  \small
  \caption{Расчёт затрат на материалы}
  \begin{tabular}{|p{0.2\linewidth} | p{0.2\linewidth}| p{0.2\linewidth} |p{0.2\linewidth}| p{0.09\linewidth} |}
    \hline
Наименование материала & Еденица измерения & Норма расхода & Цена за единицу, BYN & Сумма\\[0pt]
\hline
    Сталь 10 & кг & 0,9 & 4,45 / 1,2 & 3,34\\[0pt]
    \hline
    Фторопласт фольгированный двухсторонний & мм² & 6800 & 1,08 за 1000  мм² & 7,34\\[0pt]
    \hline
    Всего &  &  &  & 10,68 \\[0pt]
    \hline
    C учётом транспортных расходов &  &  &  & 12,82 \\[0pt]
    \hline
\end{tabular}
\end{table}

%%% Local Variables:
%%% mode: LaTeX
%%% TeX-master: "main"
%%% End:


\item Расчёт затрат по статье «Покупные комплектующие изделия,
  полуфабрикаты, и услуги производственного характера»

  В данную статью включаются затраты на приобретение в порядке
  производственной кооперации готовых покупных изделий и полуфабрикатов,
  используемых для комплектования изделий или подвергающихся
  дополнительной обработке на данном предприятии для получения готовой
  продукции (радиоэлементы, микросхемы и прочее и пр.)

  \begin{longtable}{| p{0.25\linewidth} |  p{0.2\linewidth} |  p{0.15\linewidth} |  p{0.15\linewidth} |}
  \caption{Расчёт затрат на комплектующие изделия и полуфабрикаты} \\
\hline
Наименование комплектующего или полуфабриката & Кол-во на единицу, шт & Цена, BYN & Сумма, р\\%[0pt]
\hline
  Резистор 10 кОм & 7 & 0,05 & 0,35 \\
  \hline
  Резистор 22 Ом  & 2 & 0,04 & 0,08 \\[0pt]
  \hline
  Резистор 1 кОм  & 2 & 0,05 & 0,1\\[0pt]
  \hline
  Переменный резистор & 4 & 21 & 84\\[0pt]
  \hline
  Конденсатор 100 мкФ & 1 & 0,14 & 0,14\\[0pt]
  \hline
  Конденсатор 10 мкФ  & 1 & 0,10 & 0,1\\[0pt]
  \hline
  Конденсатор 100нФ & 1 & 0,75 & 0,75 \\[0pt]
  \hline
  Конденсатор 22пФ SMD0805 & 2 & 3,30 & 6,6\\[0pt]
  \hline
  Микросхема LM7805 & 1 & 2,35 & 2,35 \\[0pt]
  \hline
  Микроконтроллер ATMega328PU & 1 & 15,50 & 15,5 \\[0pt]
  \hline
  Зуммер & 1 & 4,10 & 4,1 \\[0pt]
  \hline
  Светодиод & 1 & 0,10 & 0,1 \\[0pt]
  \hline
  Переключатель DPDT & 1 & 11 & 11 \\[0pt]
  \hline
  Переключатель SPDT & 1 & 6 & 6\\[0pt]
  \hline
  Диод Шотки 1N5817 & 1 & 0,13 & 0,13 \\[0pt]
  \hline
  Стабилитрон BZX79C5V1 & 1 & 0,08 & 0,08 \\[0pt]
  \hline
  Микросхема LM385-BZ2.5 & 1 & 8,90 & 8,9\\[0pt]
  \hline
  Транзистор 2N7000 & 1 & 0,18 & 0,18 \\[0pt]
  \hline
  Разъём pld-6 & 1 & 0,22 & 0,22 \\[0pt]
  \hline
  Разъём KLDX-0202-AC & 1 & 4,90 & 4,9 \\[0pt]
  \hline
  Разъём SCT3001MH-2x6P & 2 & 1,45 & 2,9 \\[0pt]
  \hline
  Разъем PBD1.27-2-4 & 1 & 2,70 & 2,7 \\[0pt]
  \hline
  Кварцевый резонатор & 1 & 1,90 & 1,9\\[0pt]
  \hline
  Заклепки BN572 & 25 & 0,3Br за 200 штук & 0,0375\\[0pt]
  \hline
  Всего &  &  & 153,12\\[0pt]
  \hline
  Всего, с транспортнозаготовительными расходами &  &  & 168,43\\[0pt]
  \hline
\end{longtable}
    
%%% Local Variables:
%%% mode: LaTeX
%%% TeX-master: "main"
%%% End:




  % \begin{table}[H]
  \small
  \caption{Формирование отпускной цены изделия на основе полной себестоимости}
\begin{tabular}{|p{0.25\linewidth} | p{0.40\linewidth}|}
  \hline
  Показатель & Формула / таблица для расчёта\\[0pt]
  \hline
  Материалы & Таблица 7.1\\[0pt]
  \hline
  Компоненты & Таблица 7.2\\[0pt]
  \hline
  Накладные расходы & \begin{equation}  Р_{накл}=\frac{(Р_м+P_к) \cdot H_{накл}}{100}    \end{equation} \\[0pt]
  \hline
  Полная себестоимость & \begin{equation} С_п =Р_м + Р_к + Р_{накл}  \end{equation} \\[0pt]
  \hline
  Плановая прибыль & \begin{equation}  П_{ед}= \frac{С_п \cdot Р_{пр}}{100} \end{equation}\\[0pt]
  \hline
  Отпускная цена изделия & \begin{equation}  Ц_{ОТП}=С_п+П_{ед}\end{equation}\\[0pt]
  \hline
\end{tabular}
\end{table}


%%% Local Variables:
%%% mode: LaTeX
%%% TeX-master: "main"
%%% End:

  
\item Накладные расходы рассчитываются по формуле ~\cite{bsuir-project-economics}:
\begin{equation}
  Р_{накл}=\frac{(Р_м+P_к) \cdot H_{накл}}{100}
\end{equation}

Где  $Н_{накл}$ - норматив накладных расходов, 54\%, а $Р_м$, $Р_к$ - расходы на материалы и комплектующие изделия, BYN.
Таким образом
$$  Р_{накл}=(12,82+168,43) \cdot 54\%  = 97,87 Br$$

\item Полная себестоимость рассчитывается как сумма накладных расходов и
расходов на материалы и комплектующие ~\cite{bsuir-project-economics}:
\begin{equation}
  С_п =Р_м + Р_к + Р_{накл}
\end{equation} 

$$С_п = 12,82 + 168,43 + 97,87 = 279,12 Br$$ 

\item Плановая прибыль считается как произведение полной себестоимости и рентабельности продукции:
  \begin{equation}
    П_{ед}= \frac{С_п \cdot Р_{пр}}{100}
  \end{equation}
  Где  $P_{пр}$ - рентабельность продукции, взята равной 40\%.
  Таким образом:
$$ П_{ед} = 279,12 \cdot 40\% = 111,65 Br$$

\item Отпускная цена изделия считается как сумма полной себестоимости
  и плановой прибыли~\cite{bsuir-project-economics}:
  \begin{equation}
    Ц_{ОТП}=С_п+П_{ед}
  \end{equation}
  И равна: $$Ц_{ОТП}=279,12+111,65= 390,77Br$$ 
\end{enumerate}

Так как в данном случае предполагается автоматизированное производство
нового изделия, прямые расходы на оплату труда не выделяются в
отдельные статьи.

\subsection{Расчет экономического эффекта}

Экономическим эффектом от производства и реализации новых изделий
является прирост чистой прибыли, полученной от их реализации.  Расчет
прироста чистой прибыли у предприятия-производителя от реализации
новых изделий осуществляется по
формуле~\cite{bsuir-project-economics}:

\begin{equation}
  \Delta П_ч=N_П \cdot П_{ед} ( 1- \frac{Н_п}{100} )
\end{equation}

Где $N_П$ — прогнозируемый годовой объём производства и реализации изделий, 1000 шт;
$П_{ед}$ — плановая прибыль на единицу изделия, 83,74 BYN;
$Н_п$ — ставка налога на прибыль согласно действующему законодательству в процентах,
(по состоянию на май 2025 20\%).

Таким образом чистая прибыль составит
$$  \Delta П_ч = 1000 \cdot 111,65 \cdot 80\% = 89 318,55Br$$


\subsection{Расчёт инвестиций в производство нового изделия}
Инвестиции в прирост основного капитала не требуются,
т. к. производство нового изделия планируется осуществлять на
действующем оборудовании в связи с наличием на
предприятии-производителе свободных производственных мощностей.

Инвестиции в производство нового изделия включают~\cite{bsuir-project-economics}:
\begin{enumerate}
\item Инвестиции в разработку нового изделия.
\item Инвестиции в основной оборотный капитал.
\end{enumerate}

Производство продукции предполагается осуществлять на действующем
оборудовании на свободных производственных мощностях, поэтому
инвестиции в основной капитал не требуются.


Инвестиции в разработку нового изделия $Ир$ могут быть оценены двумя
альтернативными способами~\cite{bsuir-project-economics}:
\begin{enumerate}
\item По договорной цене разработчика, если разработка нового изделия
  осуществляется сторонней организацией (по смете разработчика);
\item По затратам на разработку нового изделия инженерами
  предприятия-производителя.
\end{enumerate}

\begin{table}[H]
  \small
  \caption{Расчёт основной заработной платы
    разработчиков нового изделия}
  \begin{tabular}{|p{0.15\linewidth}|p{0.15\linewidth}|p{0.15\linewidth}|p{0.15\linewidth}|p{0.15\linewidth}|p{0.1\linewidth}|}
    \hline
Категория исполнителя & Численность & Средняя месячная З/П & Дневной оклад & Время участия, в днях & Cумма\\[0pt]
\hline
    1, Руководитель проекта & 1 & 2580,96 & 122,90 & 20 & 2458,06\\[0pt]
    \hline
    2, Инженер схемотехник & 1 & 2580,96 & 122,90 & 20 & 2458,06\\[0pt]
    \hline
    3, Инженер конструктор & 1 & 2580,96 & 122,90 & 20 & 2458,06\\[0pt]
    \hline
    4, Техник проектировщик & 1 & 2580,96 & 122,90 & 20 & 2458,06\\[0pt]
    \hline
    Итого &  &  & 0 & 100 & 9832,23\\[0pt]
    \hline
    Премия (принята 25\%) &  & 645,24 & 30,73 &  & 3073 \\[0pt]
    \hline
    Всего &  &  &   &  & 12905,23\\[0pt]
    \hline
\end{tabular}
\end{table}

%%% Local Variables:
%%% mode: LaTeX
%%% TeX-master: "main"
%%% End:


Месячный оклад был взят из данных по о начислении средней заработной
платы работников данной области в Марте 2025
года~\cite{belstat-salary}.

% \input{table_economics_investment_in_development}

Расчёт инвестиций в разработку нового изделия велся на основании данных о освноной
заработной платы и согласно ~\cite{bsuir-project-economics}:
\begin{enumerate}
\item Производится расчёт дополнительной заработной платы по формуле~\cite{bsuir-project-economics}:
  \begin{equation} З_Д = З_О \cdot Н_д \end{equation}
  Где $Н_д$ норматив дополнительной заработной платы, принят за 5\%;
  $$З_Д = 12904,8 \cdot 5\% = 645,24Br $$
\item Рассчитываются отчисления на социальные нужды ~\cite{bsuir-project-economics}:
  \begin{equation}
    Р_{соц} = \frac{(З_О + З_Д) \cdot Н_{соц}}{100}
  \end{equation}
  $Н_{соц}$ - норматив отчислений в ФСЗН и Белгосстрах (На Май 2025 34,6\%).
  Таким образом:
  $$Р_{соц} = (12905,23 + 645,26) \cdot 34,6\% = 13128,49$$
\item Инветиции на разработку нового изделия рассчитываются как сумма
основной, дополнительной заработных плат и отчислений на социальные
нужды ~\cite{bsuir-project-economics}:
\begin{equation}
  З_р = З_О + З_Д +Р_{cоц}
\end{equation}
 $$ З_р = 12905,23 + 645,26 + 13128,49 = 26678,98Br$$
\end{enumerate}


Расчёт инвестиций в прирост собственного оборотного капитала:
~\cite{bsuir-project-economics}:
\begin{enumerate}
\item Годовая потребность в материалах
  \begin{equation}
    П_м = Р_м \cdot N
  \end{equation}
Где N, планируемое число выпускамых за год изделий.
  $$  П_м = 12,82 \cdot 1000 = 128200$$
\item Годовая потребность в комплектующих:
  \begin{equation}
    П_к = Р_к \cdot N
  \end{equation}
  $$П_к = 168,43 \cdot 100 = 1684300$$
Где N, планируемое число выпускамых за год изделий.

\item Инвестиции в прирост собственного оборотного капитала:
\begin{equation}
  И_{c.о.к} = \beta \cdot (П_м + П_к)
\end{equation}
Где $\beta$ взята за 30\%.
$$   И_{c.о.к} = 30\% \cdot (12820 + 168430) = 54375Br$$
\end{enumerate}

\subsection{Расчет показателей экономической эффективности инвестиций \\
  в производство нового изделия}
 Оценка экономической эффективности инвестиций в производство нового
изделия осуществляется на основе расчета рентабельности инвестиций.
(затрат) (Return on Investment, ROI) ~\cite{bsuir-project-economics}.
По формуле:
\begin{equation}
  ROI= \frac{\Delta Пч - (И_р + И_{c.о.к})}{И_р + И_{с.о.к}} \cdot 100 \%
\end{equation}

Таким образом

$$ROI=\frac{89 318,55 - (26678,98 + 54375)}{26678,98 + 54375}= 10,20\%$$

Рентабельность инвестиций превышает ставку по долгосрочным отзывным
депозитам~\cite{belarusbank-vklady}.
На основании этого инвестиции в производство нового изделия можно
считать экономически эффективными и разработка целесообразна.

\newpage

%%% Local Variables:
%%% mode: LaTeX
%%% TeX-master: "main"
%%% End:
