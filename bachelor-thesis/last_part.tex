\nsection{Заключение}

В результате выполнения дипломного проекта была спроектирована система
автоматического управления беспилотного летательного аппарата
мультироторного типа.

В ходе выполнения дипломного проекта, был проведен анализ
литературно-патентных исследований, обзор методов и средств управления
беспилотными летательными аппаратами мультироторного типа;
Проведено общетехническое обоснование разработки устройства, которое
включает в себя анализ исходных данных, формирование основных
технических требований к разрабатываемой конструкции, схемотехнический
анализ проектируемого средства.

% В рамках проектной части была разработана конструкция автоматической
% системы управления беспилотными летательными аппаратами
% мультироторного типа.

Был осуществлен выбор конструкторских решений, обеспечивающих удобство
ремонта и эксплуатации устройства, выбор типа электрического монтажа,
элементов крепления и фиксации, выбор способов защиты устройства от
внешних воздействий, выбор способов обеспечения нормального теплового
режима устройства (выбор способа охлаждения на ранней стадии
проектирования; выбор наименее теплостойких элементов, для которых
необходимо проведение теплового расчета), выбор и обоснование
элементной базы, конструктивных элементов, установочных изделий,
материалов конструкции и защитных покрытий, маркировки деталей и
сборочных единиц.

% Также были проведены: расчет конструктивно-технологических параметров
% проектируемого изделия, включающий расчет объемно-компоновочных
% характеристик устройства; расчет теплового режима; проектирование
% печатного модуля; расчет механической прочности и системы виброударной
% защиты; а также был выполнен полный расчет надежности.

% В ходе проведения расчета надежности, было установлено, что устройство
% обладает высокой вероятностью безотказной работы за 1000 часов и
% хорошей гамма-процентной наработкой до отказа при γ=95%.

Для проектирования печатной платы использовалась система
автоматизированного проектирования – KiCAD.

Для проектирования корпуса изделия, а также чертежа нестандартной
детали и сборочного чертежа изделия использовался программный комплекс
AutoCAD.

Для моделирования процессов, протекающих в проектируемом устройстве,
использовался программный комплекс SolidWorks.

В результате проведенного частотного анализа установлено, что частота
печатной платы, полученная при моделировании, выше диапазона
дестабилизирующего фактора, а значит устройство устойчиво к
колебаниям.

Спроектированная модель является адекватной и пригодной для
использования при условии эксплуатации устройства в пределах
допустимых значений воздействующих факторов.

% Было проведено экономическое обоснование разработки и производства
% дешифратора дискретных сигналов для многоканальной системы
% радиоуправления. Исходя из полученных результатов можно сделать вывод
% о том, что инвестиции в производство нового изделия экономически
% эффективны, так как рентабельность инвестиций составляет 10,2\%, что
% превышает ставку по банковским долгосрочным депозитам, и,
% следовательно, разработка нового изделия является целесообразной.

В результате выполнения дипломного проекта была разработана
графическая часть к дипломному проекту в виде 6 чертежей А1 и
соответствующей к ним документации.

Конструкторская документация, включающая перечень изделий,
спецификацию, электрическую схему, схему электрическую принципиальную,
чертежи нестандартных деталей, чертежи сборочных единиц, сборочный
чертеж изделия, была подготовлена с использованием программных пакетов
KiCAD и AutoCAD.

Цели дипломного проекта были достигнуты в полном объеме. Полученные
данные могут быть использованы при проектировании системы
автоматического управления беспилотным летательным аппаратом
мультироторного типа.
\newpage

% Список использованных источников
\renewcommand{\refname}{\textbf{Cписок использованных источников}}
\DeclareFieldFormat{url}{Режим доступа\addcolon\space\url{#1}}
\DeclareFieldFormat{title}{{#1}}
\DeclareFieldFormat{labelnumberwidth}{[{#1}]\adddot  }
\printbibliography[heading=bibintoc, title={Cписок использованных источников}]

\newpage

\captionsetup[table]{
    format=bsuirtable,
    singlelinecheck=false,
    labelsep=endash,
    skip=1mm,
    position=above,
    parindent=0pt, % Убираем стандартный отступ
    labelformat=empty, % Не нумеруем таблицы автоматически, подписываем вручную.
  }
  
% Приложение А
\begin{center}
\textbf{Приложение А}\\
\textbf{(обязательное)}\\
\textbf{Отчет о проверке на заимствования в системе «Антиплагиат»}
\end{center}

\addcontentsline{toc}{section}{Приложение А (обязательное) Отчет о проверке на заимствования в системе «Антиплагиат»}

\newpage


% Приложение Б
\begin{center}
\textbf{Приложение Б}\\
\textbf{(обязательное)}\\
\textbf{справка о результатах патентных исследований}
\end{center}
\addcontentsline{toc}{section}{Приложение Б (обязательное) справка о результатах патентных исследований}
\newpage

% Приложение В
\begin{center}
  \textbf{Приложение В}\\
  \textbf{(обязательное)}\\
  \textbf{техническое задание}
\end{center}
\addcontentsline{toc}{section}{Приложение В (обязательное) техническое задание}
\newpage

% Приложение Г
\begin{center}
  \textbf{Приложение Г}\\
  \textbf{(обязательное)}\\
  \textbf{перечень элементов}
\end{center}
\addcontentsline{toc}{section}{Приложение Г (обязательное) перечень элементов}
\newpage

% Приложение Д
\begin{center}
  \textbf{Приложение Д}\\
  \textbf{(обязательное)}\\
  \textbf{спецификации}
\end{center}
\addcontentsline{toc}{section}{Приложение Д (обязательное) спецификации}
\newpage

% Приложение Е
\begin{center}
  \textbf{Приложение E}\\
  \textbf{(обязательное)}\\
  \textbf{листинги результатов моделирования и расчётов}
\end{center}
\addcontentsline{toc}{section}{Приложение Е (обязательное) листинги результатов моделирования и расчетов}

  

 
\begin{table}[H]
  \centering
  \caption{Таблица E.1 – Значения частот полученные в результате моделирования в SolidWorks Simulation}
  \begin{tabular}{|l | r |}
    \hline
    Вариант моделирования & Значение частоты \\ \hline
    Вариант 1 & 176,24 Гц\\ \hline
    Вариант 2 & 325,16 Гц\\ \hline
    Вариант 3 & 352,94 Гц \\ \hline
    Вариант 4 & 378,08 Гц \\ \hline
%    Вариант 5 & 650,15 Гц \\ \hline
  \end{tabular}
\end{table}

\newpage


% Приложение Ж
\begin{center}
  \textbf{Приложение Ж}\\
  \textbf{(обязательное)}\\
  \textbf{ведомость дипломного проекта}
\end{center}
\addcontentsline{toc}{section}{Приложение Ж (обязательное) ведомость дипломного проекта}
\newpage

%%% Local Variables:
%%% mode: LaTeX
%%% TeX-master: "main"
%%% LaTeX-biblatex-use-Biber: t
%%% End:
