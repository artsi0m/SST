\section{Моделирование 
  физических процессов,
  протекающих в проектируемом
  радиоэлектронном средстве}

% В данном разделе будет приведены данные о моделировании устройства в
% следующих программах:
% \begin{itemize}
% \item \textit{Ansys},
  
% \item \textit{COMSOL Multiphysics},
% \item \textit{Solidworks Simulation},
  
% \item \textit{Solidworks Flow Simulation.}
% \end{itemize}

% Будет приведена инфомарция касательно результатов моделирования данных
% программах.

\subsection{Обоснование выбора пакетов 
  прикладного программного обеспечения 
  ANSYS, COMSOL Multiphysics, SolidWorks Simulation}

Популярность таких программ как \textit{COMSOL Multiphysics} и
\textit{ANSYS} в кругах инженеров позволяет рассчитывать на наличие
большого количетства обучающих материалов по ним, в том числе
созданных не только самой компанией осуществляющей разработку
программы.

Кроме того обе программы сами по себе обладают огромным набором
возможностей для проведения симуляции и зачастую имеют некоторый
аналог интерфейса трёхмерных параметрических САПР типа
\textit{SOLIDWORKS}, который позволяет интерактивно взаимодествовать с
геометрией модели, материалом и сеткой модели.

Также особенно хочется отметить доступность и качество документации
доступной на официальном сайте \textit{COMSOL}.

Совокупноcть вышеописанных качеств делает эти программы достойным
выбором для проведения исследования.

\newpage

%%% Local Variables:
%%% mode: LaTeX
%%% TeX-master: "main"
%%% End:
