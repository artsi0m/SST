\section{Общетехническое обоснование разработки устройства}

% Приведена информация про режимы работы устройства, то для чего именно
% оно будет использоваться, его соответствие категории из ГОСТ 151150-69.

\subsection{Анализ исходных данных}

Рассматриваемым для проектированием устройстовом является полётный
контроллер.

Изделие представляет собой двухслойную печатную плату с распаянным на
ней микроконтроллером в DIP корпусе, компонентами и
разъёмами.


Данный изделие может работать в двух режимах, выбираемых
переключателем:

\begin{enumerate} 
  
\item Ручной режим. В нём полётный контроллер генерирует импульсы для
четырех сервоприводов. Длина импульсов контролируется четырьмя
потенциометрами. В этом режиме тестовый стенд обеспечивает питание
сервоприводов или полетного контроллера и питание от квадрокоптера не
должно быть подключено к ним. Напряжение питания тестового стенда
должно быть между 7,5 или 12 Вольт.
  
\item Режим ввода. В нём измеряются длины имульсов поступающих от
приёмника сигнала. Сигналы потом поступают на выводы, подключенные к
сервпоприводам.  В этом режиме тестовый стенд и сервоприводы получают
питание от источника питания от квадрокоптера. Получаемое питание не
должно превышать 7.49 Вольт и тестовый стенд не должен быть подключен
к своему истчонику питания. Также четыре канада должны быть
подключены, иначе светодиод и зуммер просигналируют об ошибке.

\end{enumerate}

Полётный контроллер также измеряет длину контрольных импульсов и даёт
информацию о качестве источника питания.

Данный полётный контроллер призван управлять пятью параметрами:
\begin{itemize}
\item Тяга,
\item крен,
\item тангаж,
\item рысканье,
\item напряжение питания двигателя.
\end{itemize}

Информацию о сигналах выводимых к сервоприводам полётный контроллер
выводит на OLED дисплей подключенный, через I2C интерфейс.  Дисплей
показывает продолжительсность импульсов графиком из четырёх кривых
вмести с их численным значением в микросекундах
~\cite{Elector521}.

Данное устройство может быть подключено между приёмником
дистанционного управления и сервоприводами.

\subsection{Формирование основных технических требований
  к разрабатываемой конструкции}

% Вот это уточнить
Основным электрическим параметром устройства является входное
напряжение схемы равно 7,5 В. Входное напряжение микроконтроллера,
равное 5 В и частота  кварцевого резонатора
тактирующего микроконтроллер 16МГц.

Частота кварцевого резонатора тактирующая микроконтроллер является
очень важным парамтером, так как влияет и на энергопотребление батареи
подключенной к полётному контроллеру в одном из его режимов работы,
так и на те возможности, которые представлены при программировании
микроконтроллера.

По нормам надежности сервотетстер должен удовлетворять требованиям
ГОСТ 27.003-2016. Время наработки на отказ должно быть не менее 10000
часов. Среднее время восстановления устройства должно быть не более 60
минут ~\cite{GOST-27.003-2016}.

После восстановления работоспособности, по окончании ремонтных работ
при его отказе, изделие должно сохранять показатели назначения,
изложенные в техническом задании.

Трудоемкость изготовления устройства не должна
составлять больше 12 часов.

Параметры устройства должны контролироваться с помощью стандартных
измерительных приборов обслуживающим персоналом средней квалификации.

Конструкция устройства должна быть разработана согласно требованиям к
технологичисности, таким как габариты и масса устройсва.  Согласно им
габаритные размеры устройства, измерненные с погрешностью плюс-минус 1
мм, должны быть не более, мм: высота - 95 мм, ширина 200 мм, длина
200мм. Масса устройства, измеренная с погрешностью плюс-минус 0,1 кг,
должна быть не более 1 кг.

Показатели надежности по ГОСТ 27.003-2016 должны соответствовать
заданным значениям при нормальных климатических условиях (температура
окружающей среды плюс-минус 20ºC, относительная влажность 60 \%,
атмосферное давление $(958...1037) \cdot 10^2$ ПА; c отклонениями
напряжения сети 220В от плюс 10\% до минут 15\% от номинального
значения, частотой (49...51) Гц.

Средний срок службы блока должен быть не менее 30 лет.

По нормам надежности блок должен иметь время наработки на отказ
не менее 10000 часов. Среднее время восстановления
устройства должно быть не более 60 минут.

Трудоемкость изготовления блока не должна привышать 12 часов.

Конструкция устройства в целом и отдельных сложных узлов должна
обеспечивать сборку при изготовлении без создания и применения
специального оборудования.При изготовлении устройства должны
применяться стандартные методы и универсальные средства измерений,
серийное испытательное оборудование. В качестве комплектующих единиц и
деталей (коммутационные, изделия электроники, крепежные, установочные)
должны применяться серийно выпускаемые изделия.

\newpage
%%% Local Variables:
%%% mode: LaTeX
%%% TeX-master: "main"
%%% End:
