\tableofcontents
\newpage


\nsection{Перечень условных обозначений, символов и терминов}


БПЛА – беспилотный летательный аппарат.

БИС - большие интегральные схемы.

ГОСТ — государственный стандарт.

ГЖП - гибко-жесткие печатные платы.

ГПК - гибкий печатных кабель.

ДПП - двухсторонняя печатная плата.

ЭРИ - электрорадиоизделия.

МКЭ – метод конечных элементов.

МСБ - микросборки.

САПР — системы автоматизированного проектирования.

СБИС - cверх большие интегральные схемы.

ШИМ — широко импульсная модуляция.

ПП — печатная плата.

РЭС — радиоэлектронное средство.

ИМС - интегральные микросхемы.

ЭА - электронная аппаратура.

ЭРИ - электрорадиоизделие.

ЭРЭ - электрорадиоэлемент.

CAE — Computer Aided Engineering, система автоматизации инженерных расчетов.

DPDT — Double Pole Double Throw, переключатель два полюса, два направления.

DIP — Dual In-line Package, корпус с двумя рядами
прямогуольных выводов.

GPIO — General Purpose Input Output, система ввода-вывода общего пользования

OLED — Organic light emitted diode, органический светодиод.

I2C — Inter-Integrated Circuit, интерфейс микрокнтроллера.

TDP — Thermal Dessipation Power, рассеиваемая тепловая мощность.

PCB — Printed Circuti Board, печатная плата.

\newpage

\nsection{Введение}

Целью дипломной работы является проектирование конструкции
устройства системы автоматического управления
беспилотного летательного аппарата мультироторного типа сервоприводов.

Неотъемлимой частью системы тестирования является само устройство
тестирования — модуль печатной платы предназначенный для
управления беспилотным летательным аппаратом мультироторного типа.

Цель процесса проектирования состоит в том, чтобы на основании
априорной (исходной) и апостериорной (дополнительной) информации,
поступающей в процессе проектирования, получить полное описание
объекта проектирования в виде технической документации, необходимой
для его изготовления, удовлетворяющего заданным требованиям и
ограничениям.

Для достижения поставленной цели разрабатывается техническое задание
(ТЗ), проводится анализ литературно-патентных исследований;
общетехническое обоснование разработки устройства, которое включает в
себя анализ исходных данных, формирование основных технических
требований к разрабатываемой конструкции, схемотехнический анализ
проектируемого средства; разработка конструкции проектируемого
изделия, включающая выбор конструкторских решений, обеспечивающих
удобство ремонта и эксплуатации устройства, выбор типа электрического
монтажа, элементов крепления и фиксации, выбор способов защиты от
внешних воздействий, выбор способов обеспечения нормального теплового
режима устройства, выбор и обоснование элементной базы, конструктивных
элементов, расчет теплового режима, проектирование печатного модуля,
расчет механической прочности, расчет электромагнитной совместимости,
полный расчет надежности, моделирование физических процессов,
протекающих в проектируемом радиоэлектронном изделии.

Для моделирования и компьютерного проектирования применяются средства
автоматизированного проектирования (САПР).

Результатом проектирования станет прототип портативного устройства,
соответствующим современным требованием эргономики, надежности и
технологичности.

Дипломная работа выполнена самостоятельно, проверена в системе
«Атиплагиат».  Процент оригинальности составляет 58\%. Цитирования
обозначены ссылками на публикации, указанными в «Списке использованных
источников».

\newpage
%%% Local Variables:
%%% mode: LaTeX
%%% TeX-master: "main"
%%% End:
