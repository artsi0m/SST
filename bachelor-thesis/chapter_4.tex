\section{Разработка конструкции \\
  проектируемого изделия} % 4

\subsection{Выбор и обоснование элементной базы,\\
  конструктивных элементов, установочных изделий, \\
  материалов корпуса и защитных покрытий,\\
  маркировки деталей и сборочных единиц. }

% Приведены виды электрорадиоизделий, согласно ~\cite{Belyanin2008}.
% Приведены основные параметры электрорадиоэлементов, согласно
% ~\cite{Alexeev2011}.
% Рассказано на основании каких соображения осуществлялся выбор тех или
% иных элементов.

\subsection{Выбор типа электрического монтажа,\\
  элементов крепления и фиксации. }

% В этом подразделе рассказано то почему выбран именно THT монтаж в
% данном изделии.

\subsection{Выбор способов обеспечения \\
  нормального теплового режима устройства \\
  (выбор способа охлаждения \\
  на ранней стадии проектирования;\\
  выбор наименее теплостойких элементов, \\
  для которых необходимо проведение расчёта). }

% Говорится, что выбрано пассивное охлаждение из-за своей простоты.

% Рассказано почему в качестве наименее теплостойкого элемента будет
% выбран единственный транзистор на плате, либо микроконтроллер. На
% данный момент склоняюсь к выбору транзистора.

\subsection{Выбор и обоснование \\
  метода изготовления печатной платы. }

% В данном разделе будет приведена информация про несколько оптимальных
% с разных точек зрения методов изготовления печатный платы. Будет
% указано почему выбран один конкретный метода. На данный момент таким
% методом выбран метод фрезерования ~\cite{PirogovaEngineering}.

\subsection{Выбор конструкторских решений, \\
  обеспечивающих удобство ремонта \\
  и эксплуатации устройства.}

% Рассказано о том, как  может быть осуществлено перепрограммирование устройства с
% помощью разъёма ISP, без выпаивания микроконтроллера из печатной
% платы. Обосновывается выбор THT компонентов, как облегающих ремонт.
% Возможно, обосновывается добавление резистора, находящегося
% между транзистором и микроконтроллером, как меры безопасности.

\subsection{Технология разработки чертежа детали в среде KiCAD.}

% Рассказано, как разработан чертёж принципиальной схемы и печатные
% платы в САПР KiCAD. Обоснован выбор именно свободного ПО, типа САПР
% KiCAD.  Рассказано как проходит разработка в данной САПР, а также как
% может осуществляться контроль разрабатываемых файлов с помощью \textit{git}.

\subsection{Обеспечение требований стандартизации, \\
  унификации и технологичности конструкции устройства.}

% Приведены ГОСТы из листа задания. Рассказано о том, что именно
% предпринималось, чтобы изделия соответствовало данным нормативным
% документам.
\newpage

%%% Local Variables:
%%% mode: LaTeX
%%% TeX-master: "main"
%%% End:
